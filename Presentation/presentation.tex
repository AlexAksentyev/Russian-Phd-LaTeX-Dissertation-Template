\documentclass[14pt]{beamer}
\usepackage[T2A]{fontenc}
\usepackage[utf8]{inputenc}
\usepackage[english,russian]{babel}
\usepackage{amssymb,amsfonts,amsmath,mathtext}
\usepackage{cite,enumerate,float,indentfirst}

\graphicspath{{../images/}{images/}} 

\usepackage{../Dissertation/phdstyle}
\usepackage[edges]{forest}
\newcommand{\ntrn}{n_{turn}}
\newcommand{\gef}{\gamma_{eff}}

\usepackage{animate}
\usepackage{tcolorbox}

% \usetheme[secheader]{Boadilla}
% \usecolortheme{seahorse}

%\usetheme{Pittsburgh}
%\usecolortheme{whale}

\usetheme{Boadilla}

\beamertemplatenavigationsymbolsempty

\newcommand{\todo}{\alert}
%%% Основные сведения %%%
\newcommand{\thesisAuthorLastName}{\todo{Аксентьев}}
\newcommand{\thesisAuthorOtherNames}{\todo{Александр Евгеньевич}}
\newcommand{\thesisAuthorInitials}{\todo{А.\,Е.}}
\newcommand{\thesisAuthor}             % Диссертация, ФИО автора
{%
    \texorpdfstring{% \texorpdfstring takes two arguments and uses the first for (La)TeX and the second for pdf
        \thesisAuthorLastName~\thesisAuthorOtherNames% так будет отображаться на титульном листе или в тексте, где будет использоваться переменная
    }{%
        \thesisAuthorLastName, \thesisAuthorOtherNames% эта запись для свойств pdf-файла. В таком виде, если pdf будет обработан программами для сбора библиографических сведений, будет правильно представлена фамилия.
    }
}
\newcommand{\thesisAuthorShort}        % Диссертация, ФИО автора инициалами
{\thesisAuthorInitials~\thesisAuthorLastName}
%\newcommand{\thesisUdk}                % Диссертация, УДК
%{\todo{xxx.xxx}}
\newcommand{\thesisTitle}              % Диссертация, название
{\todo{Метод замороженного спина  для поиска электрического дипольного момента дейтрона в накопительном кольце}}
\newcommand{\thesisSpecialtyNumber}    % Диссертация, специальность, номер
{\todo{01.04.20}}
\newcommand{\thesisSpecialtyTitle}     % Диссертация, специальность, название
{\todo{Физика пучков заряженных частиц и ускорительная техника}}
\newcommand{\thesisDegree}             % Диссертация, ученая степень
{\todo{кандидата физико-математических наук}}
\newcommand{\thesisDegreeShort}        % Диссертация, ученая степень, краткая запись
{\todo{канд. физ.-мат. наук}}
\newcommand{\thesisCity}               % Диссертация, город написания диссертации
{\todo{Москва}}
\newcommand{\thesisYear}               % Диссертация, год написания диссертации
{\todo{2020}}
\newcommand{\thesisOrganization}       % Диссертация, организация
{\todo{Национальный исследовательский ядерный университет ``МИФИ'' \\ (НИЯУ МИФИ)}}
\newcommand{\thesisOrganizationShort}  % Диссертация, краткое название организации для доклада
{\todo{НИЯУ ``МИФИ''}}

\newcommand{\thesisInOrganization}     % Диссертация, организация в предложном падеже: Работа выполнена в ...
{\todo{Национальном исследовательском ядерном университете ``МИФИ''}}

\newcommand{\supervisorAFio}            % Научный руководитель, ФИО
{\todo{Сеничев Юрий Валерьевич}}
\newcommand{\supervisorARegalia}        % Научный руководитель, регалии
{\todo{доктор~физ.-мат.~наук,~профессор}}
\newcommand{\supervisorAFioShort}       % Научный руководитель, ФИО
{\todo{Ю.\,В.~Сеничев}}
\newcommand{\supervisorARegaliaShort}   % Научный руководитель, регалии
{\todo{д-р.~физ.-мат.~наук,~проф.}}

\newcommand{\supervisorBFio}            % Научный руководитель, ФИО
{\todo{Полозов Сергей Маркович}}
\newcommand{\supervisorBRegalia}        % Научный руководитель, регалии
{\todo{кандидат~физ.-.мат.~наук,~доцент}}
\newcommand{\supervisorBFioShort}       % Научный руководитель, ФИО
{\todo{С.\,М.~Полозов}}
\newcommand{\supervisorBRegaliaShort}   % Научный руководитель, регалии
{\todo{канд. физ-мат. наук,~доц.}}


\newcommand{\opponentOneFio}           % Оппонент 1, ФИО
{\todo{Андрианов Сегрей Николаевич}}
\newcommand{\opponentOneRegalia}       % Оппонент 1, регалии
{\todo{доктор физ.-мат. наук, профессор}}
\newcommand{\opponentOneJobPlace}      % Оппонент 1, место работы
{\todo{Санкт-Петербургский государственный университет}}
\newcommand{\opponentOneJobPost}       % Оппонент 1, должность
{\todo{заведующий кафедрой компьютерного моделирования и многопроцессорных систем}}

\newcommand{\opponentTwoFio}           % Оппонент 2, ФИО
{\todo{Филатов Юрий Николаевич}}
\newcommand{\opponentTwoRegalia}       % Оппонент 2, регалии
{\todo{кандидат физ.-мат. наук}}
\newcommand{\opponentTwoJobPlace}      % Оппонент 2, место работы
{\todo{Московский физико-технический институт (национальный исследовательский университет)}}
\newcommand{\opponentTwoJobPost}       % Оппонент 2, должность
{\todo{преподаватель кафедры общей физики}}

\newcommand{\leadingOrganizationTitle} % Ведущая организация, дополнительные строки
{\todo{НИЦ ``Курчатовский институт'' --- ИТЭФ}}

\newcommand{\defenseDate}              % Защита, дата
{\todo{DD mmmmmmmm YYYY~г.~в~XX часов}}
\newcommand{\defenseCouncilNumber}     % Защита, номер диссертационного совета
{\todo{Д\,201.004.01}}
\newcommand{\defenseCouncilTitle}      % Защита, учреждение диссертационного совета
{\todo{НИЦ ``Курчатовский институт'' --- ИФВЭ}}
\newcommand{\defenseCouncilAddress}    % Защита, адрес учреждение диссертационного совета
{\todo{142281, Московская область, г. Протвино, площадь науки, дом 1}}
\newcommand{\defenseCouncilPhone}      % Телефон для справок
{\todo{+7~(0000)~00-00-00}}

\newcommand{\defenseSecretaryFio}      % Секретарь диссертационного совета, ФИО
{\todo{Рябов Юрий Григорьевич}}
\newcommand{\defenseSecretaryRegalia}  % Секретарь диссертационного совета, регалии
{\todo{канд.~физ.-мат. наук}}            % Для сокращений есть ГОСТы, например: ГОСТ Р 7.0.12-2011 + http://base.garant.ru/179724/#block_30000

\newcommand{\synopsisLibrary}          % Автореферат, название библиотеки
{\todo{Название библиотеки}}
\newcommand{\synopsisWebsite}        % Автореферат, адрес сайта
{\url{http://www.ihep.su/pages/main/6582/6745/index.shtml}}
\newcommand{\synopsisDate}             % Автореферат, дата рассылки
{\todo{DD mmmmmmmm YYYY года}}

% To avoid conflict with beamer class use \providecommand
\providecommand{\keywords}%            % Ключевые слова для метаданных PDF диссертации и автореферата
{}
      % Основные сведения

\setbeamercolor{footline}{fg=blue}
\setbeamertemplate{footline}{
  \leavevmode%
  \hbox{%
  \begin{beamercolorbox}[wd=.333333\paperwidth,ht=2.25ex,dp=1ex,center]{}%
    % И. О. Фамилия, Организация кратко
    \thesisAuthorShort, \thesisOrganizationShort
  \end{beamercolorbox}%
  \begin{beamercolorbox}[wd=.333333\paperwidth,ht=2.25ex,dp=1ex,center]{}%
    % Город, 20XX
    \thesisCity, \thesisYear
  \end{beamercolorbox}%
  \begin{beamercolorbox}[wd=.333333\paperwidth,ht=2.25ex,dp=1ex,right]{}%
  Стр. \insertframenumber{} из \inserttotalframenumber \hspace*{2ex}
  \end{beamercolorbox}}%
  \vskip0pt%
}

\newcommand{\itemi}{\item[\checkmark]}

%%\title{\small{Название презентации}}
%\title{\small{\thesisTitle}}
%\author{\small{%
%\emph{Выступающий:}~\thesisAuthorShort\\%
%\emph{Руководитель:}~\supervisorARegaliaShort~\supervisorAFioShort
%}\\%
%\vspace{30pt}%
%\thesisOrganization%
%\vspace{20pt}%
%}
%\date{\small{\thesisCity, \thesisYear}}

\begin{document}
	\title{\small{\thesisTitle}}
	\author{\small{%
			\begin{tabular}{lll}
				\emph{Выступающий:} & & \thesisAuthorShort\\
				\emph{Руководитель:} & \supervisorARegaliaShort & \supervisorAFioShort \\
				& \supervisorBRegaliaShort & \supervisorBFioShort
			\end{tabular}
		}\\%
		\vspace{30pt}%
		\thesisOrganization%
		\vspace{20pt}%
	}
	\date{\small{\thesisCity, \thesisYear}}

\maketitle

\section{Формальные вещи}

\begin{frame}{Актуальность}
	\framesubtitle{Зачем нужно искать ЭДМ?}
	\begin{itemize}
		\item Барионная асимметрия вселенной
		\item Нарушение СР-симметрии, как одно из условий бариогенеза
		\item Существование перманентного ЭДМ нарушает СР-симметрию
	\end{itemize}
\end{frame}
\begin{frame}{Цель и задачи исследования}
	\textbf{Цель:} Оценка возможности детектирования ЭДМ на уровне $10^{-29}$\ecm~предложенным методом\\
	\textbf{Задачи:} 
	\begin{itemize}
		\item влияние бетатронных колебаний на валидность ЭДМ-статистики
		\item спин-декогеренция вблизи нулевого резонанса
		\item свойства МДМ-прецессии, связанной с неидеальностью машины (фэйк-сигнал)
		\item её кабилровка и исключение из ЭДМ-статистики
		\item оценка статистической точности
	\end{itemize}
\end{frame}

\begin{frame}{Научная новизна}
	\begin{enumerate}
		\item Конкретно моего исследования: 
			\begin{itemize}
				\item исследовано влияние бетатронных колебаний
				\item промоделирована процедура калибровки МДМ-прецессии
			\end{itemize}
		\item Вообще поиска ЭДМ (на COSY): 
		\begin{itemize}
			\item научились держать поляризацию продольно-поляризорванного пучка в течении 1,000 секунд
			\item научились измерять (относительную) частоту прецессии спина (спин-тюн) с точностью $10^{-10}$
			\item научились юстировать квадруполи при помощи самого пучка (Beam Based Alignment)
		\end{itemize}
	\end{enumerate}
\end{frame}

\begin{frame}{Практическая значимость}
	По результатам моего исследования 
	\begin{itemize}
		\item сформулированы аргументы в пользу частотного подхода к поиску ЭДМ в накопительном кольце
		\item исследованы систематические эффекты работы с поляризацией пучка в режиме нулевого спинового резонанса
		\item проведена оценка статистической точности (и оптимальных параметров) предполагаемого эксперимента
	\end{itemize}
\end{frame}

\begin{frame}{Положения выносимые на защиту}
	\begin{itemize}
		\item ЭДМ-статистика частотного метода измерения не чувствительна к возмущениям со стороны бетатронного движения частиц
		\item Возможно достичь времени жизни поляризации пучка на уровне 1,000 секунд
		\item Свойства угловой скорости МДМ-прецессии 
			\begin{itemize}
				\item вынуждают использование частотных методов измерения ЭДМ
				\item оставляют возможность исключения этой систематической ошибки из конечной статистики
			\end{itemize}
	\end{itemize}
\end{frame}
\begin{frame}
	\begin{itemize}
		\item Зависимость частоты прецессии спина частицы может быть выражена как функция одной переменной, называемой эффективным Лорнец-фактором, и отражающей величину продольного эмиттанса частицы
		\item Эффективный Лоренц-фактор поддаётся калибровке
		\item Возможно достичь величины стандартной ошибки среднего значения ЭДМ-статистики на уровне $10^{-29}$\ecm~за год измерений
	\end{itemize}
\end{frame}

\begin{frame}{Апробация}
	\begin{itemize}
		\item На COSY проводились исследования оптимизации времени когерентности спина при помощи секступольных полей
		\item Результаты моего исследования пошли в поддготавливаемый коллаборацией JEDI для CERN отчёта, под названием ``Feasibility study for an EDM Storage Ring''
	\end{itemize}
\end{frame}

\begin{frame}[allowframebreaks]
	\frametitle{Основные публикации по теме диссертации}
	\begin{thebibliography}{9}
		\bibitem{Stats-LaPlas}
		A.E. Aksentev, Y.V. Senichev, ``Statistical precision in charged particle {EDM} search in storage rings.'' J. Phys.: Conf. Ser. \textbf{941} 012083 (2017)
		\bibitem{Stats-IPAC}
		A.E. Aksentyev, Y.V. Senichev,
		``Model of Statistical Errors in the Search for the Deuteron EDM in the Storage Ring,''
		in \emph{Proc. IPAC'17}, Copenhagen, Denmark, May 2017, pp. 2258--2260,
		\url{https://doi.org/10.18429/JACoW-IPAC2017-TUPVA079}.
		\bibitem{SOD-modeling-LaPlas}
		A. Aksentev ``Modeling of spin-orbital dynamics in a storage ring.'' J. Phys.: Conf. Ser. \textbf{1238} 012079 (2019)		
		\bibitem{SD-LaPlas}
		A. E. Aksentyev, Y. V. Senichev, ``Spin Decoherence in a Frozen Spin Lattice, Its Suppression and Effect on the Frequency Domain Edm Statistic,''  в процессе публикации.
		\bibitem{GFF-IPAC}
		A.E. Aksentyev, Y.V. Senichev, ``Simulation of the Guide Field Flipping Procedure for the Frequency Domain Method,'' in \emph{Proc. IPAC'19}, Melbourne, Australia, May 2019, pp. 858--860, \url{doi:10.18429/JACoW-IPAC2019-MOPTS010}
		\bibitem{SMP-IPAC}
		A.E. Aksentyev, Y.V. Senichev, ``Spin Motion Perturbation Effect on the EDM Statistic in the Frequency Domain Method,'' in \emph{Proc. IPAC'19}, Melbourne, Australia, May 2019, pp. 861--863,
		\url{doi:10.18429/JACoW-IPAC2019-MOPTS011}
		\bibitem{SD-IPAC}
		A.E. Aksentyev, Y.V. Senichev, ``Spin Decoherence in the Frozen Spin Storage Ring Method of Search for a Particle EDM,'' in \emph{Proc. IPAC'19}, Melbourne, Australia, May 2019, pp. 864--866,
		\url{doi:10.18429/JACoW-IPAC2019-MOPTS012}
	\end{thebibliography}
\end{frame}

\section{Интерес для ускорительщиков}

\section{Мясо исследования}

\begin{frame}{Результаты работы}
	content...
\end{frame}

\begin{frame}
\begin{center}
Спасибо за внимание!
\end{center}
\end{frame}

\end{document} 
