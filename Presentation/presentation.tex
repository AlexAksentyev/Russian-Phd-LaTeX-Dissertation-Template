\documentclass[14pt]{beamer}
\usepackage[T2A]{fontenc}
\usepackage[utf8]{inputenc}
\usepackage[english,russian]{babel}
\usepackage{amssymb,amsfonts,amsmath,mathtext}
\usepackage{cite,enumerate,float,indentfirst}

\graphicspath{{../images/}{images/}} 



% \usetheme[secheader]{Boadilla}
% \usecolortheme{seahorse}

%\usetheme{Pittsburgh}
%\usecolortheme{whale}

\beamertemplatenavigationsymbolsempty

\newcommand{\todo}{\alert}
%%% Основные сведения %%%
\newcommand{\thesisAuthorLastName}{\todo{Аксентьев}}
\newcommand{\thesisAuthorOtherNames}{\todo{Александр Евгеньевич}}
\newcommand{\thesisAuthorInitials}{\todo{А.\,Е.}}
\newcommand{\thesisAuthor}             % Диссертация, ФИО автора
{%
    \texorpdfstring{% \texorpdfstring takes two arguments and uses the first for (La)TeX and the second for pdf
        \thesisAuthorLastName~\thesisAuthorOtherNames% так будет отображаться на титульном листе или в тексте, где будет использоваться переменная
    }{%
        \thesisAuthorLastName, \thesisAuthorOtherNames% эта запись для свойств pdf-файла. В таком виде, если pdf будет обработан программами для сбора библиографических сведений, будет правильно представлена фамилия.
    }
}
\newcommand{\thesisAuthorShort}        % Диссертация, ФИО автора инициалами
{\thesisAuthorInitials~\thesisAuthorLastName}
%\newcommand{\thesisUdk}                % Диссертация, УДК
%{\todo{xxx.xxx}}
\newcommand{\thesisTitle}              % Диссертация, название
{\todo{Метод замороженного спина  для поиска электрического дипольного момента дейтрона в накопительном кольце}}
\newcommand{\thesisSpecialtyNumber}    % Диссертация, специальность, номер
{\todo{01.04.01}}
\newcommand{\thesisSpecialtyTitle}     % Диссертация, специальность, название
{\todo{Приборы и методы экспериментальной физики}}
\newcommand{\thesisDegree}             % Диссертация, ученая степень
{\todo{кандидата физико-математических наук}}
\newcommand{\thesisDegreeShort}        % Диссертация, ученая степень, краткая запись
{\todo{канд. физ.-мат. наук}}
\newcommand{\thesisCity}               % Диссертация, город написания диссертации
{\todo{Москва}}
\newcommand{\thesisYear}               % Диссертация, год написания диссертации
{\todo{2019}}
\newcommand{\thesisOrganization}       % Диссертация, организация
{\todo{Национальный Ядерный Исследовательский Университет ``МИФИ'' \\ (НИЯУ МИФИ)}}
\newcommand{\thesisOrganizationShort}  % Диссертация, краткое название организации для доклада
{\todo{НазУчДисРаб}}

\newcommand{\thesisInOrganization}     % Диссертация, организация в предложном падеже: Работа выполнена в ...
{\todo{учреждении, в~котором выполнялась данная диссертационная работа}}

\newcommand{\supervisorFio}            % Научный руководитель, ФИО
{\todo{Сеничев Юрий Валериевич}}
\newcommand{\supervisorRegalia}        % Научный руководитель, регалии
{\todo{д.ф.-.м.н., проф.}}
\newcommand{\supervisorFioShort}       % Научный руководитель, ФИО
{\todo{Ю.\,В.~Сеничев}}
\newcommand{\supervisorRegaliaShort}   % Научный руководитель, регалии
{\todo{уч.~ст.,~уч.~зв.}}


\newcommand{\opponentOneFio}           % Оппонент 1, ФИО
{\todo{Фамилия Имя Отчество}}
\newcommand{\opponentOneRegalia}       % Оппонент 1, регалии
{\todo{доктор физико-математических наук, профессор}}
\newcommand{\opponentOneJobPlace}      % Оппонент 1, место работы
{\todo{Не очень длинное название для места работы}}
\newcommand{\opponentOneJobPost}       % Оппонент 1, должность
{\todo{старший научный сотрудник}}

\newcommand{\opponentTwoFio}           % Оппонент 2, ФИО
{\todo{Фамилия Имя Отчество}}
\newcommand{\opponentTwoRegalia}       % Оппонент 2, регалии
{\todo{кандидат физико-математических наук}}
\newcommand{\opponentTwoJobPlace}      % Оппонент 2, место работы
{\todo{Основное место работы c длинным длинным длинным длинным названием}}
\newcommand{\opponentTwoJobPost}       % Оппонент 2, должность
{\todo{старший научный сотрудник}}

\newcommand{\leadingOrganizationTitle} % Ведущая организация, дополнительные строки
{\todo{Федеральное государственное бюджетное образовательное учреждение высшего профессионального образования с~длинным длинным длинным длинным названием}}

\newcommand{\defenseDate}              % Защита, дата
{\todo{DD mmmmmmmm YYYY~г.~в~XX часов}}
\newcommand{\defenseCouncilNumber}     % Защита, номер диссертационного совета
{\todo{Д\,123.456.78}}
\newcommand{\defenseCouncilTitle}      % Защита, учреждение диссертационного совета
{\todo{Название учреждения}}
\newcommand{\defenseCouncilAddress}    % Защита, адрес учреждение диссертационного совета
{\todo{Адрес}}
\newcommand{\defenseCouncilPhone}      % Телефон для справок
{\todo{+7~(0000)~00-00-00}}

\newcommand{\defenseSecretaryFio}      % Секретарь диссертационного совета, ФИО
{\todo{Фамилия Имя Отчество}}
\newcommand{\defenseSecretaryRegalia}  % Секретарь диссертационного совета, регалии
{\todo{д-р~физ.-мат. наук}}            % Для сокращений есть ГОСТы, например: ГОСТ Р 7.0.12-2011 + http://base.garant.ru/179724/#block_30000

\newcommand{\synopsisLibrary}          % Автореферат, название библиотеки
{\todo{Название библиотеки}}
\newcommand{\synopsisDate}             % Автореферат, дата рассылки
{\todo{DD mmmmmmmm YYYY года}}

% To avoid conflict with beamer class use \providecommand
\providecommand{\keywords}%            % Ключевые слова для метаданных PDF диссертации и автореферата
{}
      % Основные сведения

\setbeamercolor{footline}{fg=blue}
\setbeamertemplate{footline}{
  \leavevmode%
  \hbox{%
  \begin{beamercolorbox}[wd=.333333\paperwidth,ht=2.25ex,dp=1ex,center]{}%
    % И. О. Фамилия, Организация кратко
    \thesisAuthorShort, \thesisOrganizationShort
  \end{beamercolorbox}%
  \begin{beamercolorbox}[wd=.333333\paperwidth,ht=2.25ex,dp=1ex,center]{}%
    % Город, 20XX
    \thesisCity, \thesisYear
  \end{beamercolorbox}%
  \begin{beamercolorbox}[wd=.333333\paperwidth,ht=2.25ex,dp=1ex,right]{}%
  Стр. \insertframenumber{} из \inserttotalframenumber \hspace*{2ex}
  \end{beamercolorbox}}%
  \vskip0pt%
}

\newcommand{\itemi}{\item[\checkmark]}

%\title{\small{Название презентации}}
\title{\small{\thesisTitle}}
\author{\small{%
\emph{Выступающий:}~\thesisAuthorShort\\%
\emph{Руководитель:}~\supervisorARegaliaShort~\supervisorAFioShort
}\\%
\vspace{30pt}%
\thesisOrganization%
\vspace{20pt}%
}
\date{\small{\thesisCity, \thesisYear}}

\begin{document}

\maketitle

\begin{frame}
\frametitle{Цели и задачи}
\begin{itemize}
  \item \textbf{Предмет исследования:} методология частотной области для детектирования ЭДМ дейтрона в накопительном кольце с замороженным спином
  \item \textbf{Исследуемые характеристики:} 
  \begin{itemize}
  	\item устойчивость к систематическим ошибкам
  	\item статистическая точность
  \end{itemize}
  \item \textbf{Цель исследования:} оценка возможности детектирования ЭДМ дейтрона с точностью $10^{-29}~e\cdot$см предложенным методом
  \item \textbf{Актуальность:} исследование велось в рамках проекта, посвящённого поиску ЭДМ элементарных частиц
\end{itemize}
\end{frame}

\begin{frame}
\frametitle{Проблемы}
\begin{itemize}
  \item Возмущения спиновой динамики
  \item Декогеренция спинов частиц пучка
  \item Поля неидеальности машины
  \item Смена полярности ведущего поля ускорителя
\end{itemize}
\end{frame}

\begin{frame}
\frametitle{План работ}
\begin{enumerate}
  \item \textbf{Возмущения спиновой динамики}
  \begin{itemize}
  	\item Постановка проблемы
  	\item Результаты симуляции
  \end{itemize}
  \item \textbf{Декогеренция спинов}
  \begin{itemize}
    \item Симуляция подавления декогеренции в идеальном ускорителе
    \item Симуляция подавления декогеренции в неидеальном ускорителе
    \item Анализ механизма подавления декогеренции
  \end{itemize}
\end{enumerate}
\end{frame}
\begin{frame}
	\begin{enumerate}	\setcounter{enumi}{2}
		\item \textbf{Поля неидеальности ускорителя}
		\begin{itemize}
			\item Исследование зависимости от распределения неидеальностей вдоль кольца
			\item Сравнение систематической ошибки при движении пучка в прямом и обратном направлениях в кольце
		\end{itemize}
		\item \textbf{Смена полярности ведущего поля}
		\begin{itemize}
			\item Алгоритм калибровки
			\item Результаты симуляции
		\end{itemize}
	\end{enumerate}
\end{frame}
\begin{frame}
	\begin{enumerate}  \setcounter{enumi}{4}
		\item \textbf{Спин-тюн эквивалентность частиц с одинаковыми эффективными Лоренц-факторами}
		\begin{itemize}
			\item Формулировка A
			\item Формулировка B
		\end{itemize}
		\item \textbf{Структуры колец для поиска ЭДМ методом замороденного спина}
		\begin{itemize}
			\item BNL FS
			\item QFS 6.3
			\item QFS E+B
		\end{itemize}
	\end{enumerate}
\end{frame}
%%%%%%%%%%%%%%%%%%%%%%%%%%%%%
%%%%%%%% MAIN CONTENT %%%%%%%

%%%%%%%%%%%%%%%%%%%%%%%%%%%%%%
\begin{frame}
\frametitle{Перспективы развития проекта}
\begin{itemize}
  \item Поляризованная программа на ускорительном комплексе НИКА, Дубна
\end{itemize}
\end{frame}

\begin{frame}
\frametitle{Результаты работы}
\begin{itemize}
  \item Изучены эффекты спиновой динамики, составляющие систематические ошибки эксперимента:
  \begin{itemize}
  	\item возмущения спиновой динамики, вызванные бетатронным движением
  	\item декогеренция спинов
  	\item МДМ прецессия, связанная с неидеальностью машины
  \end{itemize}
  \item Описаны средства борьбы с каждым из эффектов, проведено численное моделирование
\end{itemize}
\end{frame}
\begin{frame}
	\begin{itemize}
		  \item Сформулированы понятия:
		\begin{itemize}
			\item методов пространственной и частотной областей
			\item двумерно-замороженного спина
			\item необходимые условия успешного измерения ЭДМ в накопительном кольце
			\item методология, удовлетворяющая этим условиям
		\end{itemize}
		\item Описаны структуры с замороженным и квази-замороженным спином
	\end{itemize}
\end{frame}

\begin{frame}
\begin{center}
Спасибо за внимание!
\end{center}
\end{frame}

\end{document} 
