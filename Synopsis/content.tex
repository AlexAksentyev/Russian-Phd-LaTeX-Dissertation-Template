
\newcommand{\geff}{\gamma_{eff}}

\section*{Общая характеристика работы}

\newcommand{\actuality}{\underline{\textbf{\actualityTXT}}}
\newcommand{\progress}{\underline{\textbf{\progressTXT}}}
\newcommand{\aim}{\underline{{\textbf\aimTXT}}}
\newcommand{\tasks}{\underline{\textbf{\tasksTXT}}}
\newcommand{\novelty}{\underline{\textbf{\noveltyTXT}}}
\newcommand{\influence}{\underline{\textbf{\influenceTXT}}}
\newcommand{\methods}{\underline{\textbf{\methodsTXT}}}
\newcommand{\defpositions}{\underline{\textbf{\defpositionsTXT}}}
\newcommand{\reliability}{\underline{\textbf{\reliabilityTXT}}}
\newcommand{\probation}{\underline{\textbf{\probationTXT}}}
\newcommand{\contribution}{\underline{\textbf{\contributionTXT}}}
\newcommand{\publications}{\underline{\textbf{\publicationsTXT}}}

Одной из альтернатив Стандартной Модели (СМ) элементарных частиц являются теории суперсимметрии 
(так называемые  SUSY-теории). Электрический дипольный момент (ЭДМ) элементарных частиц может служить
отличным инстрментом для выбора между этими двумя моделями; к примеру: в случае нейтрона, 
электрический дипольный момент, совместный со стандартной моделью, 
находится в диапазоне от $10^{-33}$ до $10^{-30}$~\ecm.~\cite{Harris:Neutron2007}, в то время как SUSY-теории
предсказывает наличие ЭДМ гораздо большей величины -- на уровне ${10^{-29} - 10^{-24}}$~\ecm. 

Поиск ЭДМ частиц был начат более 50-ти лет назад. Первый эксперимент по измерению ЭДМ нейтрона был проведён д-р. Н.Ф. Рэмзи (Dr. N.F. Ramsey) в конце 1950-х годов. По результатам эксперимента, верхняя граница ЭДМ нейтрона была ограничена величиной ${5\cdot 10^{-20}}$~\ecm.~\cite{Ramsey:Neutron1957} С тех пор было проведено множество более точных экспериментов, и на данный момент, верхняя граница на ЭДМ нейтрона находится на уровне ${2.9\cdot 10^{-26}}$~\ecm.~\cite{Baker:nEDM:Main, Baker:nEDM:Reply}

Большинство экспериментов проводятся на зарядово-неитральных частицах, таких как нейтрон или атомы. ЭДМ заряженных частиц, таких как протон или дейтрон, можно измерить в накопительном кольце, на основе прецессии поляризации пучка в электрическом поле в системе центра масс пучка.

Идея использования накопительного кольца для детектирования ЭДМ заряженный частиц появилась в процессе разработки $g-2$ эксперимента~\cite{BNL:g-2:2001} в Брукхейвенской Национальной Лаборатории (BNL, США). По результатам экспериментов в BNL, верхняя граница электрического дипольного момента мюона была установлена на уровне $10^{-19}$~\ecm.~\cite{BNL:muon_ANA:2009} В 1990-х годах, дискуссия преимущественно велась вокруг мюонного эксперимента~\cite{Farley:SREDM:Muon}, однако также рассматривался и дейтрон, у которого похожее отношение аномального магнитного момента к массе.

В 2004 году, коллаборацией srEDM (Storage Ring EDM Collaboration)~\cite{BNL:SREDM} в BNL был предложен эксперимент 970 по детектированию ЭДМ дейтрона на уровне $10^{-27}~e\cdot$см в накопительном кольце. 

Тогда же была предложена идея ``замороженного спина'' (``frozen spin'' method~\cite{Farley:SREDM:Muon}), в котором направления векторов спина и импульса (референсной) частицы совпадают в каждый момент времени. Это условие должно обеспечивать максимальный рост сигнала электрического дипольного момента при его наличии. Теретически, условие ``замороженного спина'' означает нулевой спиновый резонанс, при котором ориентация спин-вектора остается  пространственно-неизменной при отсутствии электрического дипольного момента. Тогда, любой рост вертикальной компоненты поляризациии пучка детектирует наличие электрического дипольного момента. Таким образом, измеряя  амплитуду вертикальной компоненты спина, мы определяем величину электрического дипольного момента. Реализация этой концепции потребует специальное накопительное кольцо и определенных параметров пучка.

Однако, в последствии выяснилось, что ``замороженный спин'' -- лишь одно из условий успешного детектирования электрического дипольного момента. В частности, для измерения ЭДМ с требуемой точностью необходимо накопление большой статистики, которое возможно при сохранении поляризации, то есть максимальной однонаправленности спина всех частиц в пучке, в течение достаточно длительного времени, порядка 1~000~секунд. 

Другим важным условием является требование исключения примешивания к сигналу ЭДМ сигнала магнитного дипольного момента (МДМ), возникающего из-за различного рода несовершенств элементов кольца, и соизмеримого с ЭДМ. 
Классический метод ``замороженного спина'' (когда спин-прецессия, связанная с МДМ, исключена полностью) проблематичен в этом отношении по двум причинам: 
во-первых, при приближении к состоянию ``замороженности'' спина малейшие возмущения со стороны магнитного и электрического полей приводят к нарушению ориентации оси стабильного спина, что сразу же вносит неопределенность в измерение вертикальной компоненты спина; 
во-вторых, для уменьшения скорости МДМ спин-прецессии вокруг радиальной оси до уровня, позволяющего измерить ЭДМ с точностью $10^{-29}$~\ecm, необходимо устанавливать оптические элементы ускорителя с точностью, значительно превышающей технологические возможности современных геодезических методов.

Начиная с 2005 года, на циклотроне AGOR KVI-центра передовых радиационных технологий (KVI-Center for Advanced Radiation Technology) в университете Гронингена была проведена серия тестов по технико-экономическому обоснованию эксперимента.

В 2008 году начались исследования на накопительном кольце COSY в Исследовательском центре ``Юлих'' (Forschungszentrum J\"ulich GmbH, Германия). В период с 2015 по 2019 автор принимал непосредственное участие в этих работах. Исследования велись по трем направлениям.

Первое: экспериментальное изучение декогеренции спина частиц в пучке. Поскольку кольцо COSY не отвечает требованиям реализации условия ``замороженного спина,''  декогеренция изучалась по времени исчезновения средней по пучку ассиметрии сечения взаимодействия в реакции рассеяния дейтронного пучка на углеродной мишени~\cite{COSY:SCT:IPAC15}. Вектор поляризации пучка при этом быстро прецессировал в плоскости замкнутой орбиты, что, однако, не влияет на сделанные выводы. 

Второе направление --- экспериментальное детектирование сигнала электрического дипольного момента с помощью возбуждения параметрического резонанса прецессии спина. Сила резонанса при этом пропорциональна величине детектируемого ЭДМ. Резонансный метод не требует условия ``замороженного спина,'' но его чувствительность на четыре-пять порядков ниже; в лучшем случае, его достижимый предел измерения ЭДМ находится на уровне $10^{-24}$~\ecm. 

Третье направление --- разработка метода измерения ЭДМ, и его полномасштабное моделирование с целью его совершенствования, а также  разработки новых подходов к измерению электрического дипольного момента заряженной частицы с использованием накопительного кольца.

Впоследствии, эти тесты развились в программу по изучению спин-орбитаьной динамики пучка для разработки технологий, требуемых для эксперимента по поиску ЭДМ. В этом же году было сделано второе предложение~\cite{BNL:Deuteron2008} эксперимента по поиску ЭДМ дейтрона на основе концепции ``замороженного спина''; в этот раз --- на уровне $10^{-29}$~\ecm~ при условии накопления результатов измерения в течение года.

В то же время было решено, что эксперимент по детектированию ЭДМ протона, поскольку его можно измерить в полностью электростатическом кольце, обладает некоторыми техническими достоинствами. Среди таковых предполагается возможность одновременной инжекции противоположно-циркулирующих пучков, что позволяет уменьшить систематические ошибки измерения ЭДМ протона, вызванные несовершенством элементов накопительного кольца. Тем не менее, на COSY была продолжена работа над экспериментом с дейтроном, ввиду того, что результаты, полученные для дейтрона, распространяются и на протон.

В 2011 году была сформирована коллаборация JEDI (J\"ulich Elecric Dipoe moment Investigations).~\cite{JEDI:Website} Целью коллаборации является не только разработка ключевых технологий для srEDM, но также и проведение предварительного эксперимента прямого наблюдения ЭДМ дейтрона. 

В 2018 году, JEDI-коллаборация выполнила первое измерение дейтронного ЭДМ на COSY на основе резонансного метода~\cite{COSY:Partially-Frozen-Spin, COSY:SpinTuneMapping} с использованием специально разработанного для этой цели RF Wien filter~\cite{JSlim:RFWF:Design, JSlim:RFWF:Commisioning}. В кольце с незамороженным спином 
ЭДМ генерирует мало-амплитудные осцилляции вертикакльной компоненты поляризации пучка; например, 
при импульсе дейтронов 970 МэВ/с, как на COSY, амплитуда колебаний ожидается на уровне 
$3\cdot10^{-10}$ при величине ЭДМ $d = 10^{-24}$~\ecm. В связи с малостью амплитуды колебаний, установленный в данном эксперименте предел измерения ЭДМ оценивается на уровне $d=10^{-24}$~\ecm.

% {\progress} 
% Этот раздел должен быть отдельным структурным элементом по
% ГОСТ, но он, как правило, включается в описание актуальности
% темы. Нужен он отдельным структурынм элемементом или нет ---
% смотрите другие диссертации вашего совета, скорее всего не нужен.

{\aim} данной работы является развитие метода поиска электрического дипольного момента дейтрона с использованием накопительного кольца на основе измерения частоты прецессии спина (frequency domain method) с экспериментально подтвержденной точностью.

Для~достижения поставленной цели необходимо было решить следующие {\tasks}:
\begin{enumerate}
%  \item Исследовать явление декогеренции спина пучка в окрестности нулевой спиновой частоты, а также секступольный метод её подавления. 
%  \item Исследовать влияние возмущений спиновой динамики на ЭДМ-статистику.
%  \item Исследовать влияние неточности установки E+B спин-ротаторов на систематическую ошибку ЭДМ-статистики.
%  \item Промоделировать процесс калибровки спин-тюна пучка при смене полярности ведущего поля.
	\item Формулирование метода измерения электрического дипольного момента дейтрона на основе измерения частоты прецессии спина (frequency domain method).
	\item Анализ требований к магнитооптической структуре кольца-накопителя, ориентированного на поиск электрического дипольного момента дейтрона.
	\item Исследование явления декогеренции спина пучка дейтронов в окрестности состояния ``замороженного'' спина. 
	\item Разработка метода подавления декогеренции спина с помощью нелинейных элементов.
	\item Исследование влияния различного рода несовершенств элементов кольца на спин-орбитальную динамику. 
	\item Математическое моделирование процесса калибровки нормализованной частоты прецессии спина (спин-тюн) при помеременной смене полярности ведущего поля.
	\item Анализ систематических ошибок в различных предложениях по проведению эксперимента по поиску электрического дипольного момента, и их сравнение с методом frequency domain. 
	\item Изучение накопления необходимой статистики измерения электрического дипольного момента.
\end{enumerate}

{\novelty}
\begin{enumerate}
	\item Впервые предложен метод измерения электрического дипольного момента дейтрона основанный исключительно на измерении частоты прецессии спина в накопительном кольце (frequency domain method), с оцениваемым ограничением по точности на уровне $10^{-29}$~\ecm.
	\item Изучена спин-орбитальная динамика дейтронного пучка в окрестности состояния ``замороженного'' спина в накопительном кольце для поиска электрического дипольного момента. 
	\item Предложен метод калибровки средней по пучку нормированной частоты прецессии спина, позволяющий уменьшить вклад систематических ошибок.
	\item Введено определение эффективного значения фактора Лоренца, необходимое для определения зависимости частоты прецессии спина от координат в 3-х мерном пространстве. 
	\item Сделаны статистические оценки предельной чувствительности измерения ЭДМ предложенным методом в накопительном кольце. 
%	\item Исследована систематическая ошибка эксперимента по поиску ЭДМ в накопительном кольце, связанная с бетатронными колебаниями.
	\item Проведена общая классификация методов поиска ЭДМ в накопительном кольце; систематизированы их общие проблемы.
\end{enumerate}

{\influence}. Результаты исследования вошли в Yellow Report под названием ``Feasibility Study for an EDM Storage Ring,'' подготавливаемый для CERN коллаборацией CPEDM, в которую входит коллаборация JEDI.

% здесь я пытаюсь подтянуть свои статьи по ТРИКу

Целью экспериментов по поиску ЭДМ является проверка CP-инвариантности. При этом ЭДМ элементарных частиц нарушают одновременно и P-, и T-симметрию, а следовательно требуют дополнительных модельных предположений, для того, чтобы связать их существование с CP-нарушением.~\cite[стр.~1926]{Aksentev:TRIC}

Альтернативой является эксперимент TRIC (Time Reversal Invariance at Cosy~\cite{Aksentev:TRIC}), 
в котором используется T-нечётное, P-чётное взаимодействие, а значит нарушается \emph{только} T-симметрия, 
в связи с чем никаких дополнительных предположений не требуется.
%
TRIC входит в физическую программу PAX (Polarised Antiproton eXperiments~\cite{Aksentev:PAX}), для которой требуются высокоинтенсивные поляризованные пучки. Существует два подхода к получению поляризованных пучков: спин-флиппинг и спин-филтеринг. Спин-флиппинг позволяет получать более интенсивные пучки, однако на данный момент не существует стабильно-работающих методов спин-флиппинга.

{\methods} Основными методами исследования являются математическое и компьютерное моделирование, и численный эксперимент.

{\defpositions}
\begin{enumerate}
	\item Метод измерения электрического дипольного момента дейтрона, основанный исключительно на измерении частоты прецессии спина.
	\item Принцип построения магнитооптической структуры кольца-накопителя, ориентированного на поиск электрического дипольного момента дейтрона.
	\item Результаты исследования декогеренции спина пучка дейтронов в окрестности состояния ``замороженного'' спина, и метод подавления декогеренции спина с помощью нелинейных элементов.
	\item Исследование влияния различного рода несовершенств элементов накопительного кольца 
	на спин-орбитальную динамику пучка. 
	\item Метод калибровки нормализованной частоты прецессии спина (спин-тьюн) при попеременной смене полярности ведущего поля и его численное моделирование.
	\item Результаты исследования систематических ошибок в различных предложениях по проведению эксперимента по поиску электрического дипольного момента, и их сравнение с методом Frequency Domain. 
	\item Оценка статистических свойств Frequency Domain метода измерения электрического дипольного момента 
	в накопительном кольце.
%	\item Подтверждена теория механизма секступольного подавления декогеренции. % section 2.2.7
%  	\item Подтверждено утверждение о равенстве спин-тюнов частиц с одинаковыми эффективными Лоренц-факторами; найдена интерпретация эффективного Лоренц-фактора как меры продольного эмиттанса частицы. % section 2.5.2
%  	\item Показано, что калибровка ведущего магнитного поля ускорителя посредством наблюдения частоты прецессии поляризации пучка в горизонтальной плоскости --- потенциально работающая методика.
%  	\item Доказано, что возмущения спиновой динамики пучка, вызванные бетатронными колебаниями --- пренебрежимо малый систематический эффект, поддающийся контролю в методологии частотной области. % pretty well-founded
%  % statistics
%  	\item Доказано, что эффективная длительность цикла измерения поляризации находится в диапазоне от двух до трёх постоянных времени жизни поляризации. % this is a fairly well-founded claim
%  	\item Показана принципиальная возможность получения верхнего предела оценки ЭДМ на уровне $10^{-29}~e\cdot$см за полное время измерений длительностью один год. % if beam 1e11, sampling rate 375 Hz etc
%  	\item Доказано, что угловая скорость паразитного МДМ вращения линейно зависит от среднего угла наклона спин-ротаторов, и не зависит от конкретной реализации распределения наклонов. % multiple random distributions
%  	\item Доказано, что точность установки оптических элементов ускорителя не позволяет измерять ЭДМ частицы методами пространственной области. % b/c spin precession frequency is on the order of 50 rad/sec
\end{enumerate}

{\reliability} полученных результатов обеспечивается согласованием аналитических вычислений с результатами численных экспериментов. Результаты компьютерных симуляций находятся в соответствии с результатами, полученными другими авторами, и результатами, полученными в экспериментах на ускорителе COSY (Исследовательский центр ``Юлих,'' Германия).


{\probation}
Основные результаты работы докладывались~на:
\begin{itemize}
\item IIX международной концеренции по ускорителям заряженных частиц IPAC'17, Копенгаген, Дания.
\item X международной конференции по ускорителям заряженных частиц IPAC'19, Мельбурн, Австралия.
\item конференциях коллаборации JEDI, Юлих, Германия, 2017--2019.
\item III международной конференции ``Лазерные, плазменные исследования и технологии,'' (LaPlas) Москва, Россия. 
\item IV междунарожной конференции LaPlas, Москва, Россия.
\item V международной конференции LaPlas, Москва, Россия.
\item студенческих семинарах Института Ядерных Исследований, Исследовательский Центр ``Юлих,'' Германия.
\end{itemize}

{\contribution} Все положения, выносимые на защиту, получены автором лично. Результаты аналитического и численного исследования спин-орбитальной динамики пучка для моделирования метода измерения электрического дипольного момента дейтрона с помощью измерения прецессии спина в накопительном кольце получены автором лично либо при участии научного руководителя. Вклад соавторов в результаты, полученные совместно, оговаривается в тексте диссертации для каждого случая.
%Автор принимал активное участие в коллаборации JEDI, а также подготовке Yellow Report для CERN.

%\publications\ Основные результаты по теме диссертации изложены в ХХ печатных изданиях~\cite{Sokolov,Gaidaenko,Lermontov,Management},
%Х из которых изданы в журналах, рекомендованных ВАК~\cite{Sokolov,Gaidaenko}, 
%ХХ --- в тезисах докладов~\cite{Lermontov,Management}.

\ifnumequal{\value{bibliosel}}{0}{% Встроенная реализация с загрузкой файла через движок bibtex8
    \publications\ Основные результаты по теме диссертации изложены в 7 печатных изданиях, 
    \hl{X из которых изданы в журналах}, рекомендованных ВАК, 
    7 "--- в тезисах докладов.%
}{% Реализация пакетом biblatex через движок biber
%Сделана отдельная секция, чтобы не отображались в списке цитированных материалов
    \begin{refsection}[vak,papers,conf]% Подсчет и нумерация авторских работ. Засчитываются только те, которые были прописаны внутри \nocite{}.
        %Чтобы сменить порядок разделов в сгрупированном списке литературы необходимо перетасовать следующие три строчки, а также команды в разделе \newcommand*{\insertbiblioauthorgrouped} в файле biblio/biblatex.tex
        \printbibliography[heading=countauthorvak, env=countauthorvak, keyword=biblioauthorvak, section=1]%
        \printbibliography[heading=countauthornotvak, env=countauthornotvak, keyword=biblioauthornotvak, section=1]%
        \printbibliography[heading=countauthorconf, env=countauthorconf, keyword=biblioauthorconf, section=1]%
        \printbibliography[heading=countauthor, env=countauthor, keyword=biblioauthor, section=1]%
        \nocite{%Порядок перечисления в этом блоке определяет порядок вывода в списке публикаций автора
	        	TRIC,% VAK, Scopus, WoS
	        	PAX,% Scopus, WoS
	        	Stats,% Scopus, WoS
	        	Modeling, % Scopus, WoS
                Aksentev:IPAC17,% Scopus
                Aksentev:LaPlas17,Aksentev:LaPlas18,Aksentev:LaPlas19,% РИНЦ
                Aksentev:IPAC19:GFF,Aksentev:IPAC19:SMP,Aksentev:IPAC19:DECOH% Scopus
        }%
        \publications\ Основные результаты по теме диссертации изложены в~\arabic{citeauthor}~печатных изданиях:
        \arabic{citeauthornotvak} изданы в журналах, индексируемых в международных базах цитирования Scopus и Web of Science,
        %из которых \arabic{citeauthorvak} в журнале, рекомендованном ВАК, 
        а \arabic{citeauthorconf} "--- в~трудах докладов на международных конференциях. Из последних, 4 работы входят в базу Scopus, 3 в РИНЦ.
    \end{refsection}
    \begin{refsection}[vak,papers,conf]%Блок, позволяющий отобрать из всех работ автора наиболее значимые, и только их вывести в автореферате, но считать в блоке выше общее число работ
        \printbibliography[heading=countauthorvak, env=countauthorvak, keyword=biblioauthorvak, section=2]%
        \printbibliography[heading=countauthornotvak, env=countauthornotvak, keyword=biblioauthornotvak, section=2]%
        \printbibliography[heading=countauthorconf, env=countauthorconf, keyword=biblioauthorconf, section=2]%
        \printbibliography[heading=countauthor, env=countauthor, keyword=biblioauthor, section=2]%
        \nocite{TRIC}%vak
        \nocite{Stats, PAX, Modeling}%notvak
        \nocite{Aksentev:LaPlas17,Aksentev:LaPlas18,Aksentev:LaPlas19,Aksentev:IPAC17,%
	        	Aksentev:IPAC19:GFF,Aksentev:IPAC19:SMP,Aksentev:IPAC19:DECOH}%conf
    \end{refsection}
}
%При использовании пакета \verb!biblatex! для автоматического подсчёта
%количества публикаций автора по теме диссертации, необходимо
%их~здесь перечислить с использованием команды \verb!\nocite!.
 % Сеничев сказал упростить введение для автореферата
% Характеристика работы по структуре во введении и в автореферате не отличается (ГОСТ Р 7.0.11, пункты 5.3.1 и 9.2.1), потому её загружаем из одного и того же внешнего файла, предварительно задав форму выделения некоторым параметрам

%Диссертационная работа была выполнена при поддержке грантов ...

%\underline{\textbf{Объем и структура работы.}} Диссертация состоит из~введения,
%четырех глав, заключения и~приложения. Полный объем диссертации
%\textbf{ХХХ}~страниц текста с~\textbf{ХХ}~рисунками и~5~таблицами. Список
%литературы содержит \textbf{ХХX}~наименование.

\section*{Содержание работы}
Во \underline{\textbf{введении}} обосновывается актуальность
исследований, проводимых в~рамках данной диссертационной работы,
приводится обзор научной литературы по изучаемой проблеме,
формулируется цель, ставятся задачи работы, излагается научная новизна
и практическая значимость представляемой работы. Содержание следующих глав
такого:

% Здесь можно расписать поподробнее, потому что большинство людей не будут
% читать сам диссер, а ознакомятся только с авторефератом
В \textbf{первой главе}: вводится понятие ``замороженного спина''; проводится сравнительный анализ методов поиска ЭДМ в накопительном кольце с ``замороженным спином''; классифицируются проблемы, общие для всех методов поиска ЭДМ в накопительном кольце; описывается метод измерения ЭДМ в накопительном кольце, позволяющий решить поставленные проблемы; представлена магнитооптическая структура накопительного кольца, в котором возможно детектировать ЭДМ дейтрона предлагаемым методом.
%\begin{enumerate}
%	\item Вводит понятие замороженного спина.
%	\item Проводит классификацию метдов поиска ЭДМ в накопительном кольце с замороженным спином.
%	\item Проводит классификацию проблем, общих для всех методов поиска ЭДМ в накопительном кольце.
%	\item Описывает метод измерения ЭДМ в накопительном кольце с замороженным спином, разрешающий описанные проблемы.
%	\item Описывает магнитооптические структуры накопительных колец, которые можно использовать для детектирования ЭДМ предлагаемым методом.
%\end{enumerate}

Во \textbf{второй главе} содержится подробное рассмотрение проблем, обозначенных в первой главе, и методов их решения; описаны результаты моделирования. 

Рассматриваемые проблемы:
\begin{enumerate}
	\item возмущения спиновой динамики частицы, вызванные её бетатронными колебаниями, и их эффект на ЭДМ-статистику частотного метода измерения;
	\item декогеренция спинов частиц продольно-поляризованного пучка при работе в режиме замороженного спина;
	\item величина и свойства систематической ошибки эксперимента, связанной с МДМ-прецессией спинов частиц пучка, и вызванной неидеальностями оптической структуры ускорителя;
	\item процедура смены полярности ведущего поля накопительного кольца, при сохранении величины МДМ спин-прецессии, необходимая для исключения обозначенной выше ошибки из ЭДМ-статистики.
\end{enumerate}

Отдельно рассматривается вопрос интерпретации введённого в первой главе понятия \emph{эффективного Лоренц-фактора} ($\gamma_{eff}$). 

Больш\'{а}я часть методологии, исследованию которой посвящена настоящая работа, основана на этом понятии. Его можно определять таким образом: если две частицы имеют одно и то же значение $\gamma_{eff}$, то они эквивалентны с точки зрения спиновой динамики (а именно, направления и величины вектора угловой скорости спин-прецессии), независимо от частностей их орбитального движения. 

Именно фиксация значения $\gamma_{eff}$ позволяет нам исключить МДМ-прецессию, связанную с неидеальностями машины, из конечной ЭДМ-статистики частотного метода.

\textbf{Третья глава} посвящена статистическому моделированию эксперимента, 
и оценке его возможной статистической точности. Исследуется возможность повышения эффективности
поляриметрии путём использования частотно-модулированной схемы выборки. Модулированная схема 
состоит в том, чтобы измерять поляризацию пучка в момент максимальной скорости её изменения.

Мы приходим к выводу о нецелесообразности использования модулированной схемы выборки. Она даёт только
малый выигрыш (40\%) по сравнению с немодулированной схемой, \emph{даже если} не учитывать вариацию 
анализирующей способности детектора. Учитывая, что максимальная скорость изменения соответствует
окрестности продольной ориентации вектора поляризации пучка, в которой анализирующая способность 
детектора минимальна, полезность модулированной схемы ещё меньше.

Также важно отметить отсутствие прямой зависимости между частотой $\omega$ измеряемого сигнала, 
и стандартным отклонением оценки частоты $\sigma_{\hat\omega}$. То есть, нет принципиальной разницы
измеряется ли частота в 1 или 100~рад/сек. Это обстоятельство важно для методов детектирования ЭДМ, 
основанных на измерении частоты прецессии спина: благодаря ему, строго говоря, 
отсутствует необходимость подавлять МДМ-прецессию, связанную с неидеальностями 
оптической структуры ускорителя.


В \textbf{четвёртой главе} приведены наиболее значимые (для данной работы) технологии, разработанные в рамках исследований, проводимых на синхротроне COSY,\footnote{Принадлежащем институту ядерных исследований исследовательского центра ``Юлих'', Германия} описаны результаты процедуры оптимизации времени когерентности спина (spin coherence time, SCT) при помощи семейств секступолей, установленных на COSY. 

Отдельно стоит отметить наблюдение явления изменения SCT при длительном измерении поляризации деструктивными методами, связаного с переходом от внешней (оболочки) к внутренней (ядру) частям пучка. Наблюдение этого явления косвенно подтверждает теорию спин-декогеренции, изложенную в данной работе.


В \underline{\textbf{заключении}} приведены основные результаты работы, которые заключаются в следующем:
%% Согласно ГОСТ Р 7.0.11-2011:
%% 5.3.3 В заключении диссертации излагают итоги выполненного исследования, рекомендации, перспективы дальнейшей разработки темы.
%% 9.2.3 В заключении автореферата диссертации излагают итоги данного исследования, рекомендации и перспективы дальнейшей разработки темы.

\begin{enumerate}
	\item Разработан метод измерения электрического дипольного момента дейтрона, 
	основанный исключительно на измерении частоты прецессии спина.
	\item Предложен принцип построения магнитооптической структуры кольца-накопителя, 
	ориентированного на поиск электрического дипольного момента дейтрона.
	\item Получены результаты исследования спин-декогеренции пучка дейтронов в окрестности 
	состояния ``замороженного спина'', а также метод подавления спин-декогеренции, основанный на использовании нелинейных элементов.
	\item Исследованы эффекты различного рода несовершенств элементов накопительного кольца 
	на спин-орбитальную динамику пучка.
	\item Проведено численное моделирование матода калибровки нормализованной частоты прецессии спина 
	при попеременной смене полярности ведущего поля накопительного кольца.
	\item Исследованы систематические ошибки в различных предложениях по проведению эксперимента 
	по поиску электрического дипольного момента; проведён сравнительный анализ этих предлодений 
	с методом Frequency Domain.
	\item Проведена оценка статистических свойств Frequency Domain метода измерения 
	электрического дипольного момента в накопительном кольце.
\end{enumerate}
%\begin{enumerate}
%  \item Были изучены эффекты спиновой динамики, составляющие систематические ошибки эксперимента по поиску
%  электрического дипольного момента частицы методом замороженного спина в накопительном кольце, как то:
%  \begin{itemize}
%  	\item возмущения спиновой динамики вызванные бетатронным движением частицы;
%  	\item декогеренция спинов частиц пучка;
%  	\item МДМ прецессия спина, вызванная неидеальностями ускорителя.
%  \end{itemize}
%  \item Для каждого из эффектов, было описано средство борьбы, и проведено численное моделирование,
%  подтверждающее его эффективность.
%  \item Были сформулированы:
%  \begin{itemize}
%  	\item понятия методов пространственной и временной областей;
%  	\item понятие двумерно-замороженного спина;
%  	\item необходимые условия успешного измерения ЭДМ в накопительном кольце;
%  	\item метод Frequency Domain, удовлетворяющий всем сформулированным условиям.
%  \end{itemize}
%  \item Описаны структуры накопительных колец с непрерывно- и квази-замороженным спином.
%\end{enumerate}


\ifdefmacro{\microtypesetup}{\microtypesetup{protrusion=false}}{} % не рекомендуется применять пакет микротипографики к автоматически генерируемому списку литературы
\ifnumequal{\value{bibliosel}}{0}{% Встроенная реализация с загрузкой файла через движок bibtex8
  \renewcommand{\bibname}{\large \authorbibtitle}
  \nocite{*}
  \insertbiblioauthor           % Подключаем Bib-базы
  %\insertbiblioother   % !!! bibtex не умеет работать с несколькими библиографиями !!!
}{% Реализация пакетом biblatex через движок biber
  \ifnumgreater{\value{usefootcite}}{0}{
%  \nocite{*} % Невидимая цитата всех работ, позволит вывести все работы автора
  \insertbiblioauthorcited      % Вывод процитированных в автореферате работ автора
  }{
  \insertbiblioauthor           % Вывод всех работ автора
%  \insertbiblioauthorgrouped    % Вывод всех работ автора, сгруппированных по источникам
%  \insertbiblioauthorimportant  % Вывод наиболее значимых работ автора (определяется в файле characteristic во второй section)
  \insertbiblioother            % Вывод списка литературы, на которую ссылались в тексте автореферата
  }
}
\ifdefmacro{\microtypesetup}{\microtypesetup{protrusion=true}}{}
