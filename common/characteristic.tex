
{\actuality} Обзор, введение в тему, обозначение места данной работы в
мировых исследованиях и~т.\:п.,

Одной из главных проблем современной физики является барионная асимметрия вселенной, т.е. преобладание материи над антиматерией в наблюдаемой вселенной. В 1967 году, академик АН СССР Андрей Сахаров сформулировал условия бариогенеза, одним из которых является существование процессов, нарушающих C- и CP-симметрии.

Нарушение CP-симметрии в каонных системах известно с 1964 года; оно наблюдалось в B-распадах, и распадах очарованных мезонов. CP-нарушение может быть вписано в Стандартную Модель (СМ) элементарных частиц посредством матрицы Кабиббо-Кобаяши-Масакавы, но 

Интерес к поиску Электрических Дипольных Моментов (ЭДМ) частиц состоит в том, что, их существование нарушает пространственную и временную симметрии, а значит, в по теореме CPT, и CP-симметрию. Таким образом, обнаружение ненулевых ЭДМ элементарных частиц может обнаружить физику за гранью Стандартной Модели (СМ) элементарных частиц. Такие теории как SUSY (суперсимметрия) указывают на наличие ЭДМ гораздо большей величины (на уровне $10^{-29}$ до $10^{-24}~e\cdot$см), чем предсказывает СМ.

% {\progress} 
% Этот раздел должен быть отдельным структурным элементом по
% ГОСТ, но он, как правило, включается в описание актуальности
% темы. Нужен он отдельным структурынм элемементом или нет ---
% смотрите другие диссертации вашего совета, скорее всего не нужен.

{\aim} данной работы является численное моделирование метода поиска электрического дипольного момента дейтрона в накопительном кольце с ``замороженным'' спином.

Для~достижения поставленной цели необходимо было решить следующие {\tasks}:
\begin{enumerate}
  \item Исследовать явление декогеренции спина пучка в окрестности нулевой спиновой частоты, а также секступольный метод её подавления. 
  \item Исследовать влияние возмущений спиновой динамики на ЭДМ-статистику.
  \item Исследовать влияние неточности установки E+B спин-ротаторов на систематическую ошибку ЭДМ-статистики.
  \item Промоделировать процесс калибровки спин-тюна пучка при смене полярности ведущего поля.
\end{enumerate}


{\novelty}
\begin{enumerate}
  \item Впервые была промоделирована процедура калибровки спин-тюна пучка при смене направления его движения. 
  \item Было выполнено оригинальное исследование систематической ошибки эксперимента по поиску ЭДМ в накопительном кольце, связанной с бетатронными колебаниями.
\end{enumerate}

{\influence} \ldots

{\methods} \ldots

{\defpositions}
\begin{enumerate}
  \item Первое положение
  \item Второе положение
  \item Третье положение
  \item Четвертое положение
\end{enumerate}
В папке Documents можно ознакомиться в решением совета из Томского ГУ
в~файле \verb+Def_positions.pdf+, где обоснованно даются рекомендации
по~формулировкам защищаемых положений. 

{\reliability} полученных результатов обеспечивается \ldots \ Результаты находятся в соответствии с результатами, полученными другими авторами.


{\probation}
Основные результаты работы докладывались~на:
\begin{itemize}
\item IIX международной концеренции по ускорителям заряженных частиц IPAC'17, Копенгаген, Дания.
\item X международной конференции по ускорителям заряженных частиц IPAC'19, Мельбурн, Австралия.
\item конференциях коллаборации JEDI, Юлих, Германия, 2017--2019.
\item III междунарожной конференции ``Лазерные, плазменный исследования и технологии,'' (LaPlas) Москва, Россия. 
\item IV междунарожной конференции LaPlas, Москва, Россия.
\item V международной конференции LaPlas, Москва, Россия.
\item студенческих семинарах Института Ядерных Исследований, Исследовательский Центр ``Юлих,'' Германия.
\end{itemize}

{\contribution} Автор принимал активное участие в коллаборации JEDI.

%\publications\ Основные результаты по теме диссертации изложены в ХХ печатных изданиях~\cite{Sokolov,Gaidaenko,Lermontov,Management},
%Х из которых изданы в журналах, рекомендованных ВАК~\cite{Sokolov,Gaidaenko}, 
%ХХ --- в тезисах докладов~\cite{Lermontov,Management}.

\ifnumequal{\value{bibliosel}}{0}{% Встроенная реализация с загрузкой файла через движок bibtex8
    \publications\ Основные результаты по теме диссертации изложены в XX печатных изданиях, 
    X из которых изданы в журналах, рекомендованных ВАК, 
    X "--- в тезисах докладов.%
}{% Реализация пакетом biblatex через движок biber
%Сделана отдельная секция, чтобы не отображались в списке цитированных материалов
    \begin{refsection}[vak,papers,conf]% Подсчет и нумерация авторских работ. Засчитываются только те, которые были прописаны внутри \nocite{}.
        %Чтобы сменить порядок разделов в сгрупированном списке литературы необходимо перетасовать следующие три строчки, а также команды в разделе \newcommand*{\insertbiblioauthorgrouped} в файле biblio/biblatex.tex
        \printbibliography[heading=countauthorvak, env=countauthorvak, keyword=biblioauthorvak, section=1]%
        \printbibliography[heading=countauthorconf, env=countauthorconf, keyword=biblioauthorconf, section=1]%
        \printbibliography[heading=countauthornotvak, env=countauthornotvak, keyword=biblioauthornotvak, section=1]%
        \printbibliography[heading=countauthor, env=countauthor, keyword=biblioauthor, section=1]%
        \nocite{%Порядок перечисления в этом блоке определяет порядок вывода в списке публикаций автора
                vakbib1,vakbib2,%
                confbib1,confbib2,%
                bib1,bib2,%
        }%
        \publications\ Основные результаты по теме диссертации изложены в~\arabic{citeauthor}~печатных изданиях, 
        \arabic{citeauthorvak} из которых изданы в журналах, рекомендованных ВАК, 
        \arabic{citeauthorconf} "--- в~тезисах докладов.
    \end{refsection}
    \begin{refsection}[vak,papers,conf]%Блок, позволяющий отобрать из всех работ автора наиболее значимые, и только их вывести в автореферате, но считать в блоке выше общее число работ
        \printbibliography[heading=countauthorvak, env=countauthorvak, keyword=biblioauthorvak, section=2]%
        \printbibliography[heading=countauthornotvak, env=countauthornotvak, keyword=biblioauthornotvak, section=2]%
        \printbibliography[heading=countauthorconf, env=countauthorconf, keyword=biblioauthorconf, section=2]%
        \printbibliography[heading=countauthor, env=countauthor, keyword=biblioauthor, section=2]%
        \nocite{vakbib2}%vak
        \nocite{bib1}%notvak
        \nocite{confbib1}%conf
    \end{refsection}
}
При использовании пакета \verb!biblatex! для автоматического подсчёта
количества публикаций автора по теме диссертации, необходимо
их~здесь перечислить с использованием команды \verb!\nocite!.
