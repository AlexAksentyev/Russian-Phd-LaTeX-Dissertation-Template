%% Согласно ГОСТ Р 7.0.11-2011:
%% 5.3.3 В заключении диссертации излагают итоги выполненного исследования, рекомендации, перспективы дальнейшей разработки темы.
%% 9.2.3 В заключении автореферата диссертации излагают итоги данного исследования, рекомендации и перспективы дальнейшей разработки темы.

\begin{enumerate}
	\item Разработан метод измерения электрического дипольного момента дейтрона, 
	основанный исключительно на измерении частоты прецессии спина частицы 
	при движении в накопительном кольце.
	\item Предложен принцип построения магнитооптической структуры кольца-накопителя, 
	ориентированного на поиск электрического дипольного момента дейтрона.
	\item Получены результаты исследования спин-декогеренции пучка дейтронов в окрестности 
	состояния ``замороженного спина,'' а также метод подавления спин-декогеренции, основанный на использовании нелинейных элементов оптической структуры накопителя.
	\item Исследованы эффекты различного рода несовершенств элементов накопительного кольца 
	на спин-орбитальную динамику пучка.
	\item Проведено численное моделирование метода калибровки нормализованной частоты прецессии спина 
	при попеременной смене полярности ведущего поля накопительного кольца.
	\item Исследованы систематические ошибки в различных предложениях по проведению эксперимента 
	по поиску электрического дипольного момента; проведён сравнительный анализ этих предло;ений 
	с методом Frequency Domain.
	\item Проведена оценка статистических свойств Frequency Domain метода измерения 
	электрического дипольного момента в накопительном кольце.
\end{enumerate}
%\begin{enumerate}
%  \item Были изучены эффекты спиновой динамики, составляющие систематические ошибки эксперимента по поиску
%  электрического дипольного момента частицы методом замороженного спина в накопительном кольце, как то:
%  \begin{itemize}
%  	\item возмущения спиновой динамики вызванные бетатронным движением частицы;
%  	\item декогеренция спинов частиц пучка;
%  	\item МДМ прецессия спина, вызванная неидеальностями ускорителя.
%  \end{itemize}
%  \item Для каждого из эффектов, было описано средство борьбы, и проведено численное моделирование,
%  подтверждающее его эффективность.
%  \item Были сформулированы:
%  \begin{itemize}
%  	\item понятия методов пространственной и временной областей;
%  	\item понятие двумерно-замороженного спина;
%  	\item необходимые условия успешного измерения ЭДМ в накопительном кольце;
%  	\item метод Frequency Domain, удовлетворяющий всем сформулированным условиям.
%  \end{itemize}
%  \item Описаны структуры накопительных колец с непрерывно- и квази-замороженным спином.
%\end{enumerate}
