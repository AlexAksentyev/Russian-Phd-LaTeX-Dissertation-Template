%% Согласно ГОСТ Р 7.0.11-2011:
%% 5.3.3 В заключении диссертации излагают итоги выполненного исследования, рекомендации, перспективы дальнейшей разработки темы.
%% 9.2.3 В заключении автореферата диссертации излагают итоги данного исследования, рекомендации и перспективы дальнейшей разработки темы.
\begin{enumerate}
  \item Были изучены эффекты спиновой динамики, составляющие систематические ошибки эксперимента по поиску
  электрического дипольного момента частицы методом замороженного спина в накопительном кольце, как то:
  \begin{itemize}
  	\item возмущения спиновой динамики вызванные бетатронным движением частицы;
  	\item декогеренция спинов частиц пучка;
  	\item МДМ прецессия спина, вызванная неидеальностями ускорителя.
  \end{itemize}
  \item Для каждого из эффектов, было описано средство борьбы, и проведено численное моделирование,
  подтверждающее его эффективность.
  \item Были сформулированы:
  \begin{itemize}
  	\item понятия методов пространственной и временной областей;
  	\item понятие двумерно-замороженного спина;
  	\item необходимые условия успешного измерения ЭДМ в накопительном кольце;
  	\item метод Frequency Domain, удовлетворяющий всем сформулированным условиям.
  \end{itemize}
  \item Описаны структуры накопительных колец с непрерывно- и квази-замороженным спином.
\end{enumerate}
