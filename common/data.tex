\newcommand{\gml}{\guillemotleft}
\newcommand{\gmr}{\guillemotright}

%%% Основные сведения %%%
\newcommand{\thesisAuthorLastName}{\todo{Аксентьев}}
\newcommand{\thesisAuthorOtherNames}{\todo{Александр Евгеньевич}}
\newcommand{\thesisAuthorInitials}{\todo{А.\,Е.}}
\newcommand{\thesisAuthor}             % Диссертация, ФИО автора
{%
    \texorpdfstring{% \texorpdfstring takes two arguments and uses the first for (La)TeX and the second for pdf
        \thesisAuthorLastName~\thesisAuthorOtherNames% так будет отображаться на титульном листе или в тексте, где будет использоваться переменная
    }{%
        \thesisAuthorLastName, \thesisAuthorOtherNames% эта запись для свойств pdf-файла. В таком виде, если pdf будет обработан программами для сбора библиографических сведений, будет правильно представлена фамилия.
    }
}
\newcommand{\thesisAuthorShort}        % Диссертация, ФИО автора инициалами
{\thesisAuthorInitials~\thesisAuthorLastName}
%\newcommand{\thesisUdk}                % Диссертация, УДК
%{\todo{xxx.xxx}}
\newcommand{\thesisTitle}              % Диссертация, название
{\todo{Метод замороженного спина  для поиска электрического дипольного момента дейтрона в накопительном кольце}}
\newcommand{\thesisSpecialtyNumber}    % Диссертация, специальность, номер
{\todo{01.04.20}}
\newcommand{\thesisSpecialtyTitle}     % Диссертация, специальность, название
{\todo{Физика пучков заряженных частиц и ускорительная техника}}
\newcommand{\thesisDegree}             % Диссертация, ученая степень
{\todo{кандидата физико-математических наук}}
\newcommand{\thesisDegreeShort}        % Диссертация, ученая степень, краткая запись
{\todo{канд. физ.-мат. наук}}
\newcommand{\thesisCity}               % Диссертация, город написания диссертации
{\todo{Москва}}
\newcommand{\thesisYear}               % Диссертация, год написания диссертации
{\todo{2020}}
\newcommand{\thesisOrganization}       % Диссертация, организация
{\todo{Национальный исследовательский ядерный университет \gml МИФИ\gmr \\ (НИЯУ МИФИ)}}
\newcommand{\thesisOrganizationShort}  % Диссертация, краткое название организации для доклада
{\todo{НИЯУ ``МИФИ''}}

\newcommand{\thesisInOrganization}     % Диссертация, организация в предложном падеже: Работа выполнена в ...
{\todo{Национальном исследовательском ядерном университете \gml МИФИ\gmr~(Москва)}}

\newcommand{\supervisorAFio}            % Научный руководитель, ФИО
{\todo{Сеничев Юрий Валерьевич}}
\newcommand{\supervisorARegalia}        % Научный руководитель, регалии
{\todo{доктор~физ.-мат.~наук,~профессор}}
\newcommand{\supervisorAFioShort}       % Научный руководитель, ФИО
{\todo{Ю.\,В.~Сеничев}}
\newcommand{\supervisorARegaliaShort}   % Научный руководитель, регалии
{\todo{д-р.~физ.-мат.~наук,~проф.}}

\newcommand{\supervisorBFio}            % Научный руководитель, ФИО
{\todo{Полозов Сергей Маркович}}
\newcommand{\supervisorBRegalia}        % Научный руководитель, регалии
{\todo{доктор~физ.-мат.~наук,~доцент}}
\newcommand{\supervisorBFioShort}       % Научный руководитель, ФИО
{\todo{С.\,М.~Полозов}}
\newcommand{\supervisorBRegaliaShort}   % Научный руководитель, регалии
{\todo{д-р. физ-мат. наук,~доц.}}


\newcommand{\opponentOneFio}           % Оппонент 1, ФИО
{\todo{Андрианов Сергей Николаевич}}
\newcommand{\opponentOneRegalia}       % Оппонент 1, регалии
{\todo{доктор физ.-мат. наук, профессор}}
\newcommand{\opponentOneJobPlace}      % Оппонент 1, место работы
{\todo{Санкт-Петербургский государственный университет}}
\newcommand{\opponentOneJobPost}       % Оппонент 1, должность
{\todo{заведующий кафедрой компьютерного моделирования и многопроцессорных систем}}

\newcommand{\opponentTwoFio}           % Оппонент 2, ФИО
{\todo{Филатов Юрий Николаевич}}
\newcommand{\opponentTwoRegalia}       % Оппонент 2, регалии
{\todo{кандидат физ.-мат. наук}}
\newcommand{\opponentTwoJobPlace}      % Оппонент 2, место работы
{\todo{Московский физико-технический институт (национальный исследовательский университет)}}
\newcommand{\opponentTwoJobPost}       % Оппонент 2, должность
{\todo{заведующий лабораторией физики ускорителей}}

\newcommand{\leadingOrganizationTitle} % Ведущая организация, дополнительные строки
{\todo{НИЦ \gml Курчатовский институт\gmr~---~ИТЭФ}}

\newcommand{\defenseDate}              % Защита, дата
{\todo{\gml $\rule{1cm}{0.15mm}$\gmr~$\rule{2.2cm}{0.15mm}$~2021~г.~в~$\rule{1.5cm}{0.15mm}$~часов}}
\newcommand{\defenseCouncilNumber}     % Защита, номер диссертационного совета
{\todo{Д\,201.004.01}}
\newcommand{\defenseCouncilTitle}      % Защита, учреждение диссертационного совета
{\todo{НИЦ \gmlКурчатовский институт\gmr~---~Институт физики высоких энергий имени А.А. Логунова}}
\newcommand{\defenseCouncilAddress}    % Защита, адрес учреждение диссертационного совета
{\todo{142281, Московская область, г. Протвино, площадь науки, дом 1}}
\newcommand{\defenseCouncilPhone}      % Телефон для справок
{\todo{+7~(4967)~71-36-23}}

\newcommand{\defenseSecretaryFio}      % Секретарь диссертационного совета, ФИО
{\todo{Рябов Юрий Григорьевич}}
\newcommand{\defenseSecretaryRegalia}  % Секретарь диссертационного совета, регалии
{\todo{канд.~физ.-мат. наук}}            % Для сокращений есть ГОСТы, например: ГОСТ Р 7.0.12-2011 + http://base.garant.ru/179724/#block_30000

\newcommand{\synopsisLibrary}          % Автореферат, название библиотеки
{\todo{Название библиотеки}}
\newcommand{\synopsisWebsite}        % Автореферат, адрес сайта
{\url{http://www.ihep.ru/files/aksentyev_thesis.pdf}}
\newcommand{\synopsisDate}             % Автореферат, дата рассылки
{\todo{DD mmmmmmmm YYYY года}}

% To avoid conflict with beamer class use \providecommand
\providecommand{\keywords}%            % Ключевые слова для метаданных PDF диссертации и автореферата
{}
