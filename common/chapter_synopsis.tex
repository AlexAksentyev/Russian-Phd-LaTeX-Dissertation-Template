В \textbf{первой главе}: вводится понятие ``замороженного спина''; проводится сравнительный анализ методов поиска ЭДМ в накопительном кольце с ``замороженным спином''; классифицируются проблемы, общие для всех методов поиска ЭДМ в накопительном кольце; описывается метод измерения ЭДМ в накопительном кольце, позволяющий решить поставленные проблемы; представлена магнитооптическая структура накопительного кольца, в котором возможно детектировать ЭДМ дейтрона предлагаемым методом.
%\begin{enumerate}
%	\item Вводит понятие замороженного спина.
%	\item Проводит классификацию метдов поиска ЭДМ в накопительном кольце с замороженным спином.
%	\item Проводит классификацию проблем, общих для всех методов поиска ЭДМ в накопительном кольце.
%	\item Описывает метод измерения ЭДМ в накопительном кольце с замороженным спином, разрешающий описанные проблемы.
%	\item Описывает магнитооптические структуры накопительных колец, которые можно использовать для детектирования ЭДМ предлагаемым методом.
%\end{enumerate}

Во \textbf{второй главе} содержится подробное рассмотрение проблем, обозначенных в первой главе, и методов их решения; описаны результаты моделирования. 

Рассматриваемые проблемы:
\begin{enumerate}
	\item возмущения спиновой динамики частицы, вызванные её бетатронными колебаниями, и их эффект на ЭДМ-статистику частотного метода измерения;
	\item декогеренция спинов частиц продольно-поляризованного пучка при работе в режиме замороженного спина;
	\item величина и свойства систематической ошибки эксперимента, связанной с МДМ-прецессией спинов частиц пучка, и вызванной неидеальностями оптической структуры ускорителя;
	\item процедура смены полярности ведущего поля накопительного кольца, при сохранении величины МДМ спин-прецессии, необходимая для исключения обозначенной выше ошибки из ЭДМ-статистики.
\end{enumerate}

Отдельно рассматривается вопрос интерпретации введённого в первой главе понятия \emph{эффективного Лоренц-фактора} ($\gamma_{eff}$). 

Больш\'{а}я часть методологии, исследованию которой посвящена настоящая работа, основана на этом понятии. Его можно определять таким образом: если две частицы имеют одно и то же значение $\gamma_{eff}$, то они эквивалентны с точки зрения спиновой динамики (а именно, направления и величины вектора угловой скорости спин-прецессии), независимо от частностей их орбитального движения. 

Именно фиксация значения $\gamma_{eff}$ позволяет нам исключить МДМ-прецессию, связанную с неидеальностями машины, из конечной ЭДМ-статистики частотного метода.

\textbf{Третья глава} посвящена статистическому моделированию эксперимента, 
и оценке его возможной статистической точности. Исследуется возможность повышения эффективности
поляриметрии путём использования частотно-модулированной схемы выборки. Модулированная схема 
состоит в том, чтобы измерять поляризацию пучка в момент максимальной скорости её изменения.

Мы приходим к выводу о нецелесообразности использования модулированной схемы выборки. Она даёт только
малый выигрыш (40\%) по сравнению с немодулированной схемой, \emph{даже если} не учитывать вариацию 
анализирующей способности детектора. Учитывая, что максимальная скорость изменения соответствует
окрестности продольной ориентации вектора поляризации пучка, в которой анализирующая способность 
детектора минимальна, полезность модулированной схемы ещё меньше.

Также важно отметить отсутствие прямой зависимости между частотой $\omega$ измеряемого сигнала, 
и стандартным отклонением оценки частоты $\sigma_{\hat\omega}$. То есть, нет принципиальной разницы
измеряется ли частота в 1 или 100~рад/сек. Это обстоятельство важно для методов детектирования ЭДМ, 
основанных на измерении частоты прецессии спина: благодаря ему, строго говоря, 
отсутствует необходимость подавлять МДМ-прецессию, связанную с неидеальностями 
оптической структуры ускорителя.


В \textbf{четвёртой главе} приведены наиболее значимые (для данной работы) технологии, разработанные в рамках исследований, проводимых на синхротроне COSY,\footnote{Принадлежащем институту ядерных исследований исследовательского центра ``Юлих'', Германия} описаны результаты процедуры оптимизации времени когерентности спина (spin coherence time, SCT) при помощи семейств секступолей, установленных на COSY. 

Отдельно стоит отметить наблюдение явления изменения SCT при длительном измерении поляризации деструктивными методами, связаного с переходом от внешней (оболочки) к внутренней (ядру) частям пучка. Наблюдение этого явления косвенно подтверждает теорию спин-декогеренции, изложенную в данной работе.
