В BNL FS методе пучок продольно-поляризованных
дейтронов инжектируется в кольцо; посредством поляриметрии наблюдается
его спин-прецессия в вертикальной и горизонтальной плоскостях; ЭДМ
сигнал --- это изменение вертикальной компоненты поляризации со
временем, выражаемое как:~\cite[стр.~8]{BNL:Deuteron2008}
\begin{equation}
  \Delta P_V = P\frac{\w_{edm}}{\W}\sin\bkt{\W t + \Theta_0},
\end{equation}
где $\W = \sqrt{\w_{edm}^2 + \w_a^2}$; $\w_a,~\w_{edm}$ частоты прецесси спина,
связанные, соответственно, с магнитным и электрическим дипольными моментами.

Прилагая радиальное электрическое поле $E_r$ (величины, определяемой уравнением~\eqref{eq:FS_Er}), ожидается
уменьшить компоненту $\w_a$ по крайней мере на девять порядков; тогда, ввиду
малости величины гипотезируемой $\w_{edm}$, ${\Delta P_V\approx P\w_{edm} t}$, 
и максимальная величина $\Delta P_V$ возрастает в $10^9$ раз.

Ожидаемая чувствительность эксперимента равна $10^{-29}~e\cdot cm$ за $10^7$
секунд (6 месяцев) полного времени измерения. На этом уровне
чувствительности, необходимо детектировать изменения в величине 
асимметрии сечения взаимодействия $\varepsilon_{LR}$  
дейтронного пучка с углеродной мишенью на уровне 
$5\cdot10^{-6}$ для наименьших практически реализуемых значений
$\w_a$.~\cite[стр.~18]{BNL:Deuteron2008} Последнее обстоятельство
ставит серьёзную проблему для поляриметрии.~\cite[стр.~6]{Mane:SpinWheel} Один
из вариантов её решения лежит в применении внешнего радиального
магнитного поля и измерении общей частоты прецессии засчёт МДМ и ЭДМ
вместе. Это основа так называемого метода ``спинового колеса'' (spin wheel), 
о котором пойдёт речь в следующем разделе. 

Единственный известный систематический эффект спиновой динамики
первого порядка -- это присутствие ненулевой средней вертикальной
компоненты электрического поля $\avg{E_V}$. В этом случае, спин будет
прецессировать вокруг радиального направления с частотой~\cite[стр.~11]{BNL:Deuteron2008}
\[
\w_{syst} \approx \frac{\mu\avg{E_V}}{\beta c\gamma^2}.
\]
Здесь важно рассмотреть два обстоятельства:
\begin{itemize}
\item присутствие $\avg{E_V}\neq 0$ вызвано ошибкой юстировки оптических
  элементов ускорителя;
\item этот систематический эффект меняет знак при инжекции пучка в
  обратном направлении.
\end{itemize}
Последнее обстоятельство является причиной, по которой инжекция пучка
в кольцо в этом методе производится дважды: сначала по часовой, потом
против часовой стрелки (CW/CCW-инжекция). Хотя $\w_{syst}$ меняет знак при смене
направления движения пучка, а значит поддаётся контролю, эта методология, тем не менее,
плохо учитывает его \emph{величину}. В разделе~\ref{chpt3:imperfections} 
(численно в~\ref{chpt3:imperfections:magnitude}), мы показываем, что при реалистичной величине 
(стандартного отклонения) ошибки установки спин-ротаторов 100~мкм, 
частота МДМ прецессии вокруг радиальной оси находится на уровне 50--100~рад/сек.~\cite{Senichev:FDM} 
В связи с этим, невозможно использовать данную методологию в её оригинальном варианте.

Также, стоит отметить, что при попытке уменьшения $\w_{syst}$, увеличивается влияние
так называемой ошибки геометрической фазы, т.е. аккумуляции поворота вектора спина вокруг радиальной оси, связанного с некоммутативностью спин-поворотов вокруг продольной и вертикальной осей.~\cite[стр.~23]{BNL:Deuteron2008} 