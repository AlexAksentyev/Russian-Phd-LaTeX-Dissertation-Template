В BNL FS методе пучок продольно-поляризованных дейтронов инжектируется в кольцо. 
Посредством поляриметрии наблюдается его спин-прецессия в вертикальной и горизонтальной плоскостях. 
ЭДМ сигнал --- это изменение вертикальной компоненты поляризации со временем, 
выражаемое как:~\cite[стр.~8]{BNL:Deuteron2008}
\begin{equation}
  \Delta P_V = P\frac{\w_{edm}}{\W}\sin\bkt{\W t + \Theta_0},
\end{equation}
где $\W = \sqrt{\w_{edm}^2 + \w_a^2}$; $\w_a,~\w_{edm}$ частоты прецесси спина,
связанные, соответственно, с магнитным и электрическим дипольными моментами. 
Таким образом, выводы о величине ЭДМ делаются на основании набега фазы спин-прецессии 
за один измерительный цикл.

Вследствие последнего, в концепции BNL FS проблема разделения МДМ и ЭДМ компонент спин-прецессии
решается путём \emph{полного исключения} МДМ компоненты. 
Будем называть такое состояние  трёхмерно-замороженным спином, и обозначать как 3D FS. 
Отметим, что если бы было возможно полностью исключить МДМ компоненту спин-прецесии, 
направление оси прецесии спина определялось бы только ЭДМ компонентой.
С одной стороны, в этом и есть идея метода; с другой -- ввиду малости гипотетической величины ЭДМ --
направление оси прецессии спина теряет устойчивость, и становится подверженным малейшим возмущениям
электромагнитного поля. 

Проблема нестабильности оси прецессии спина лежит в основе так называемой 
ошибки геометрической фазы, т.е. аккумуляции поворота вектора спина вокруг радиальной оси, 
связанного с некоммутативностью спин-поворотов 
вокруг продольной и вертикальной осей.~\cite[стр.~23]{BNL:Deuteron2008} 
А поскольку величина ЭДМ в данном методе вычисляется из набега фазы спин-прецессии, очевидно, что
возмущения поля, через вариацию направления оси прецессии спина, являются источником
систематической ошибки.

Абстрагируясь от \emph{последствий}, обратимся к \emph{способу} решения проблемы 
МДМ спин-прецессии в данном методе.
Для подавления МДМ спин-прецессии вокруг вертикальной оси 
(ортогональной к оси ЭДМ прецессии), необходимо приложить
 радиальное электрическое поле $E_r$ (величины, определяемой уравнением~\eqref{eq:FS_Er}). 
Её подавление вокруг продольной и радиальной осей осуществляется путём
точной юстировки элементов оптической структуры ускорителя.

В методе предполагается уменьшение МДМ компоненты спин-прецессии $\w_a$ 
по крайней мере на девять порядков; тогда, ввиду
малости величины гипотетической $\w_{edm}$, ${\Delta P_V\approx P\w_{edm} t}$, 
и максимальная величина $\Delta P_V$ возрастает в $10^9$ раз.

Ожидаемая чувствительность эксперимента равна $10^{-29}~e\cdot cm$ за $10^7$
секунд (6 месяцев) полного времени измерения. На этом уровне
чувствительности, необходимо детектировать изменения в величине 
асимметрии сечения взаимодействия $\varepsilon_{LR}$  
дейтронного пучка с углеродной мишенью на уровне 
$5\cdot10^{-6}$ для наименьших практически реализуемых значений
$\w_a$.~\cite[стр.~18]{BNL:Deuteron2008} Последнее обстоятельство
ставит серьёзную проблему для поляриметрии.~\cite[стр.~6]{Mane:SpinWheel} Один
из вариантов её решения лежит в применении внешнего радиального
магнитного поля и измерении общей частоты прецессии за счёт МДМ и ЭДМ
вместе. Это основа так называемого метода ``спинового колеса'' (spin wheel), 
о котором пойдёт речь в следующем разделе. 

Единственный известный систематический эффект спиновой динамики
первого порядка -- это присутствие ненулевой средней вертикальной
компоненты электрического поля $\avg{E_V}$. В этом случае, спин будет
прецессировать вокруг радиального направления с частотой~\cite[стр.~11]{BNL:Deuteron2008}
\begin{equation}\label{eq:BNL:main-syst-error}
\w_{syst} \approx \frac{\mu\avg{E_V}}{\beta c\gamma^2},
\end{equation}
где магнитный дипольный момент $\mu = ge/2mc$.

Здесь важно рассмотреть два обстоятельства:
\begin{itemize}
\item присутствие $\avg{E_V}\neq 0$ вызвано ошибкой юстировки оптических
  элементов ускорителя;
\item этот систематический эффект меняет знак при инжекции пучка в
  обратном направлении.
\end{itemize}
Последнее обстоятельство является причиной, по которой инжекция пучка
в кольцо в этом методе производится дважды: сначала по часовой, потом
против часовой стрелки (CW/CCW-инжекция). Хотя $\w_{syst}$ меняет знак при смене
направления движения пучка, а значит поддаётся контролю, эта методология, тем не менее,
плохо учитывает его \emph{величину}. В разделе~\ref{chpt3:imperfections} 
(численно в~\ref{chpt3:imperfections:magnitude}), мы показываем, что при реалистичной величине 
(стандартного отклонения) ошибки установки спин-ротаторов 100~мкм, 
частота МДМ прецессии вокруг радиальной оси находится на уровне 50--100~рад/сек.~\cite{Senichev:FDM} 
В связи с этим, невозможно использовать данную методологию в её оригинальном варианте.