\newcommand{\wimp}{\w_{\avg{E_v}}}
\newcommand{\wsw}{\w_{SW}}
\newcommand{\wedm}{\w_{edm}}

\subsection{Метод BNL FS}
BNL FS метод, предложенный коллаборацией, занимающейся разработкой
метода измерения ЭДМ в накопительном кольце в Брукхейвенской
Национальной Лаборатории (США) в 2008 году,~\cite{BNL:Deuteron2008}
это метод для комбинированного кольца. Пучок продольно-поляризованных
дейтронов инжектируется в кольцо; с помощью поляриметрии наблюдается
его спин-прецессия в вертикальной и горизонтальной плоскостях; ЭДМ
сигнал --- это изменение вертикальной компоненты поляризации со
временем, выражаемое как:~\cite[стр.~8]{BNL:Deuteron2008}
\begin{equation}
  \Delta P_V = P\frac{\w_{edm}}{\W}\sin\bkt{\W t + \Theta_0},
\end{equation}
где $\W = \sqrt{\w_{edm}^2 + \w_a^2}$, $\w_a,~\w_{edm}$ угловые
скорости вызываемые, соответственно, магнитным и электрическим
дипольными моментами.

Прилагая радиальное магнитное поле $E_r$ (величины, определяемой уравнением~\eqref{eq:FS_Er}), ожидается
уменьшение компоненты $\w_a$ по крайней мере на фактор $10^9$; ввиду
малости величины гипотезируемой $\w_{edm}$, $\Delta P_V \approx P
\w_{edm} t$, и максимальная величина $\Delta P_V$ возрастает в $10^9$.

Ожидаемая чувствительность эксперимента $10^{-29}~e\cdot cm$ за $10^7$
секунд (6 месяцев) полного времени измерения. На этом уровне
чувствительности, асимметрия сечения $\varepsilon_{LR} \approx 5\cdot
10^{-6}$ для наименьших практически реализуемых значений
$\w_a$.~\cite[стр.~18]{BNL:Deuteron2008} Последнее обстоятельство
ставит серьёзную проблему для поляриметрии.~\cite{Mane:SpinWheel} Один
из вариантов её решения лежит в применении внешнего радиального
магнитного поля и измерении общей частоты прецессии засчёт МДМ и ЭДМ
вместе. Это основа так называемого метода Спинового Колеса (Spin Wheel), также 
часто называемого Кооп-Колесом (Koop Wheel), о котором в
следующем разделе. 

Единственный известный систематический эффект спиновой динамики
первого порядка это присутствие ненулевой средней вертикальной
компоненты электрического поля $\avg{E_V}$. В этом случае, спин будет
прецессировать вокруг радиального направления с частотой~\cite[стр.~11]{BNL:Deuteron2008}
\[
\w_{syst} \approx \frac{\mu\avg{E_V}}{\beta c\gamma^2}.
\]
Здесь важно рассмотреть два обстоятельства:
\begin{itemize}
\item присутствие $\avg{E_V}\neq 0$ вызвано ошибкой юстировки
  элементов ускорителя;
\item этот систематический эффект меняет знак при инжекции пучка в
  обратном направлении.
\end{itemize}
Последнее обстоятельство является причиной структуры инжекции пучка
использованной в этом методе (сначала по-часовой, потом
против-часовой; CW/CCW). Хотя $\w_{syst}$ меняет знак при смене
направления движения пучка а значит поддаётся контролю, эта методология тем не менее не учитывает
его \emph{величину}. В разделе~\ref{chpt3:imperfections} (численно в~\ref{chpt3:imperfections:magnitude}), 
мы показываем, что при реалистичной величине (стандартного отклонения) ошибки установки
спин-ротаторов 100 мкм, частота МДМ прецессии вокруг радиальной оси
находится на уровне 50--100 рад/сек.~\cite{Senichev:FDM} В связи с
этим, невозможно использовать данную методологию в её оригинальном варианте.

Также, стоит отметить, что при попытке уменьшения $\w_{syst}$, увеличивается влияние
так называемой ошибки геометрической фазы.~\cite[стр.~6]{BNL:Proton}

\subsection{Spin Wheel метод}
Озвученные выше проблемы с поляриметрией и высокой скоростью прецессии
решаются в Spin Wheel методе, предложенном проф. И. Коопом
(Новосибирский Государственный Университет).~\cite{Koop:SpinWheel} Основная идея метода в
следующем: сначала, обеспечивается условие замороженного спина; затем
включается радиальное магнитное поле величины $B_x$, достаточно сильное чтобы
вызвать вращение спина с частотой порядка 1 Гц. Поскольку поле
радиальное, вызванная им МДМ прецессия сонаправлена с ЭДМ, а значит
они складываются линейно: $\w \propto \W_{MDM} + \W_{EDM}$.

ЭДМ вклад вычисляется сравнением циклов с обратными знаками $B_x$:~\cite[стр.~1963]{Koop:SpinWheel}
\[
\W_{EDM} = \frac{\W_x(+B_x) + \W_x(-B_x)}{2}.
\]

Внешнее поле также вызовет разделение орбит
пучков.~\cite[стр.~1963]{Koop:SpinWheel} Это разделение может быть
измерено на уровне пико-метров SQUID магнетометрами; его предлагается
использовать для калибровки внешнего поля.

Поскольку из-за внешнего поля прецессия вокруг радиальной оси на 10
порядков выше чем в оригинальном предложении, значительно упрощается
задача для поляриметрии. Однако, существуют сомнения в возможности
измерить вызываемое внешним полем разделение орбит даже при помощи SQUIDов.

Также, проблема паразитного поля, вызванного ошибкой юстировки, не решена.

\subsection{Общая классификация методов FS-типа}
Методы поиска электрического дипольного момента элементарных частиц можно отнести к одной из 
двух больших категорий, которые мы будем называть 
\begin{enumerate*}[\itshape a\upshape)]
	\item методами пространственной области (Space Domain methods), и
	\item методами частотной области (Frequency Domain methods).
\end{enumerate*}

В парадигме пространственной области, наблюдают за \emph{изменением пространственной
ориентации} вектора поляризации пучка, \emph{вызванным ЭДМ}. 

Метод BNL FS --- это канонический пример методологии пространственной области: изначально
продольно-поляризованный пучок инжектируется в накопительное кольцо; наблюдают за 
вертикальной компонентой его вектора поляризации. При идеальных условиях, любое отклонение
вектора поляризации из горизонтальной плоскости связывают с действием ЭДМ.

Сразу же очевидны две технические проблемы такого подхода:
\begin{enumerate}
	\item он ставит трудную задачу для поляриметрии~\cite{Mane:SpinWheel};
	\item он налагает очень строгие ограничения на точность установки оптических элементов ускорителя.
\end{enumerate}

Первая проблема обусловлена необходимостью детектирования изменения асимметрии сечения 
взаимодействия $\epsilon_{LR}$ на уровне $5\cdot 10^{-6}$, чтобы достичь уровня чувствительности
ЭДМ $10^{-29}~e\cdot$ см.~\cite[стр.~18]{BNL:Deuteron2008}

Вторая --- требованием минимизировать величину угловой скорости МДМ прецессии в 
вертикальной плоскости:~\cite[стр.~11]{BNL:Deuteron2008}
\begin{equation}\label{eq:ImperfectionWheelRollRate}
	\w_{syst} \approx \frac{\mu\avg{E_v}}{\beta c\gamma^2},
\end{equation}
индуцированной неидеальностями ускорителя. 

В соответствии с оценками, сделанными проф. Сеничевым, чтобы выполнить это условие,
геодезическая точность установки элементов ускорителя должна достичь $10^{-14}$ м. 
Технологии сегодняшнего дня позволяют только около $10^{-4}$ м.

При практически-достижимом уровне неточности установки элементов, $\w_{syst}\gg\w_{edm}$,
и изменения ориентации вектора поляризации по большей части не имеют отношения к ЭДМ.

Другой критичной проблемой, возникающей в пространственной области, является ошибка
геометрической фазы.~\cite[стр.~6]{BNL:Proton}  Проблема заключается в том, что даже если
каким-то образом занулить неидеальности электромагнитного поля (связанные с неточностью
установки оптических элементов, или же случайными возмущениями поля) \emph{в среднем},
поскольку повороты спина не коммутируют, угол поворота поляризации, вызванный ими, не 
будет равен нулю.

Напротив, методология частотной области основана на измерении ЭДМ-\emph{добавки} к полной
(МДМ и ЭДМ вместе) \emph{угловой скорости} прецессии спина.

Вектор поляризации заставляют вращаться вокруг почти-постоянного, выделенного направления,
определённого вектором $\nbar$, с достаточно высокой угловой скоростью, при которой её
величину можно всегда легко измерить. Помимо упрощения условий поляриметрии, определённость
вектора угловой скорости является защитой от ошибки геометрической фазы.

Так называемое Спин-Колесо может быть введено в систему извне, как в методе Spin Wheel; или же
под ту же цель можно приспособить поля неидеальности машины (скорость вращения колеса 
определяется уравнением~\eqref{eq:ImperfectionWheelRollRate}). Последнее возможно потому,
что $\w_{syst}$ меняет знак при смене направления движения пучка.~\cite[стр.~11]{BNL:Deuteron2008}

\subsection{Общие проблемы методов поиска ЭДМ в накопительном кольце}
В качестве введения в предлагаемую методологию частотной области, коротко суммаризируем 
некоторые проблемы, общие для всех методов поиска ЭДМ в накопительном кольце; их можно
разделить на две большие категории:

\begin{itemize}
	\item Проблемы, решаемые Спин-Колесом:
	\begin{itemize}
		\item случайные возмущения электромагнитного поля;
		\item бетатронное движение.
	\end{itemize}
	\item Проблемы, имеющие специфические решения:
	\begin{itemize}
		\item декогеренция спина;
		\item неидеальности ускорителя.
	\end{itemize}
\end{itemize}

\subsubsection{Возмущения спиновой динамики}
Проблемы первой категории --- это такие, из-за которых возникает ошибка геометрической фазы.

И случайно-возникающие, и фокусирующие поля, действуя на бетатронно-осциллирующую частицу,
возмущают направление и величину вектора угловой скорости спин-прецессии. Эффектом является
спин-кик в направлении, определяемом возмущением.

Положим ЭДМ создаёт спин-кик вокруг радиальной ($\hat x$-) оси. Величина вектора угловой скорости
имеет общую форму
\[
\w = \sqrt{\w_x^2 + \w_y^2 + \w_z^2},
\]
где  $\w_y$ минимизируется путём удовлетворения условия замороженного спина; $\w_z$ (чья постоянная
составляющая вызвана неидеальностями ускорителя) может быть минимизирована установкой продольного 
соленоида на оптической оси.\footnote{Длины 1 м, магнитное поле приблизительно $10^{-6}$ Т.}

В пространственной области, стремятся также минимизировать добавку $\wimp$ к радиальной компоненте
угловой скорости $\w_x = \w_{edm} + \wimp$. Следовательно, и спин-кики должны быть минимизированы до
величины (значительно) меньшей чем $\w_{edm}$, чтобы понизить набег геометрической фазы до значений
меньших, чем аккумулированная ЭДМ фаза.

Польза от Спин-Колеса, сонаправленного с угловой скоростью угловой скорости ЭДМ состоит в том, что
МДМ добавки к общей угловой скорости складываются как квадраты, а потому, их эффект значительно уменьшен:
\begin{align*}
	\w &= \sqrt{(\wedm + \wsw)^2 + \w_y^2 + \w_z^2} \\
	&\approx (\wedm+\wsw)\cdot\bkt*{1 + \frac{\w_y^2 + \w_z^2}{\wsw^2}}^{\sfrac12} \\
	&\approx (\wedm + \wsw)\cdot \bkt{1 + \frac{\w_y^2 + \w_z^2}{2\wsw^2}} \\
	&\approx \wsw + \wedm + \underbrace{\frac12\frac{\w_y^2 + \w_z^2}{\wsw}}_{\epsilon}.
\end{align*}

Поскольку наша цель --- наблюдение смещения значения $\w$, связанное с ЭДМ, необходимо 
минимизировать случайную переменную $\epsilon$:
\[
\frac12\frac{\w_y^2+\w_z^2}{\wsw} < \wedm.
\]

Сделаем предварительные оценки. Положим $\wsw\approx 50$ рад/сек (причины выбора этого значения
разъяснены в разделе~\ref{chpt3:imperfections}), $\wedm\approx 10^{-9}$ рад/сек (соответствует величине
ЭДМ $10^{-29}~e\cdot$см). Тогда, сумма $\w_y^2 + \w_z^2$ должна быть меньше $10^{-7}$ рад/сек, или
же каждая из угловых скоростей меньше $3\cdot 10^{-4}$ рад/сек. Это на несколько порядков меньше,
чем ожидаемая стандартная ошибка оценки угловой скорости,~\cite{Aksentev:Stats} и потому не является проблемой.

Остаётся рассмотреть МДМ спин-кики вокруг $\hat x$-оси; они не аттеньюированы, вызывают наибольшие проблемы.
Их можно поделить на три вида: 
\begin{enumerate*}[\itshape a\upshape)]
	\item постоянные, не индуцированные ошибкой юстировки оптических элементов;
	\item полу-постоянные, индуцированные ошибками юстиорвки;
	\item случайные.
\end{enumerate*}

Полу-постоянные радиальные спин-кики (не важно, вызванные ли магнитными, или электрическими полями) меняют
знак, когда обращается направление циркуляции пучка. Влияние случайных спин-киков можно контролировать 
статистическим усреднением. Перманентные, нечувствительные ни к направлению ведущего поля, ни к направлению
движения пучка, не поддаются контролю. Но с другой стороны, их источники не должны присутствовать в ускорителе
при нормальных обстоятельствах. Мы рассматриваем влияние возмущений спиновой динамики пучка на измерение
ЭДМ в разделе~\ref{chpt3:smp}.

\subsubsection{Декогеренция спина}
Когеренцией спина называется мера или качество сохранения поляризации
в изначально полностью поляризованном пучке.~\cite[стр.~205]{Eremey:Thesis}
Под декогеренцией спина понимают деполяризацию, связанную с различием угловых скоростей 
частиц пучка. 

Разница угловых скоростей, в свою очередь, связана с разницей длин орбит частиц, и следовательно
их уровней равновесной энергии, от которых зависит спин-тюн. Одним из способов подавления декогеренции
спина является утилизация секступольных полей. Как это работает, мы рассматриваем в рвзделе~\ref{chpt3:decoherence}.

\subsubsection{Неидеальности ускорителя}
Как мы уже указывали, проблема неидеальностей ускорителя имеет две стороны:
\begin{enumerate*}[\itshape a\upshape)]
	\item они практически не могут быть убраны настоящини технологиями; но что ещё хуже
	\item если их убрать, мы попадаем в пространственную область, и открываем метод для геометрической фазы.
\end{enumerate*}

К счастью, спин-кики, возбуждаемые ими, меняют знак при смене направления движения пучка. К тому же,
их величина достаточна, чтобы использовать их в качестве Спин-Колеса. 

Остаётся одна проблема --- точность скорости вращения колеса при смене его направления. Этот момент
рассматривается в разделе~\ref{chpt3:GFF}.

\subsection{Метод Frequency Domain}\label{sec:FDM_concept}
Методология Frequency Domain (далее FDM)~\cite{Senichev:FDM} была разработана специально для решения проблемы неточности установки магнитов, и возникающего в связи с этим паразитного МДМ вращения спина. Как было обозначено выше, частота вращения спина в вертикальной плоскости, связанная с магнитным дипольным моментом, при реалистичной ошибке юстировки, находится на уровне 8--16 Гц, что делает невозможным наблюдение медленного нарастания вертикальной компоненты поляризации, связанное с наличием у частицы электрического дипольного момента, как предполагается оригинальным BNL FS методом. В FDM, ЭДМ-эффект вычисляется путём сравнения комбинированной (МДМ + ЭДМ) частоты прецессии, наблюдаемой при циркуляции пучка в прямом и обратном направлениях. Поскольку при смене полярности ведущего поля $\vec B \mapsto -\vec B$, $\vec\beta \mapsto -\vec\beta$, и $\vec E \mapsto \vec E$:
\begin{subequations}
	\begin{align}
		\w_x^{CW/CCW} &= \w_x^{MDM, CW/CCW} + \w_x^{EDM, CW/CCW}, \\
		\w_x^{MDM, CW} &= -\w_x^{MDM, CCW} \equiv \w_x^{MDM}, \label{eq:FDM_CW_CCW_MDM} \\
		\w_x^{EDM, CW} &= \w_x^{EDM, CCW} \equiv \w_x^{EDM},
		\intertext{поэтому, ЭДМ-эстиматор}
		\hat\w_x^{EDM} &:= \frac12\bkt{\w_x^{CW} + \w_x^{CCW}} \label{eq:FDM_estimator} \\
		&= \w_x^{EDM} + \underbrace{\frac12\bkt{\w_x^{MDM, CW} + \w_x^{MDM, CCW}}}_{\varepsilon \to 0}.\notag
	\end{align}
\end{subequations}

Для того, чтобы гарантировать малость $\varepsilon$ по-сравнению с требуемой точностью измерений, т.е., что уравнение~\eqref{eq:FDM_CW_CCW_MDM} выполняется достаточно точно, была разработана специальная процедура смены полярности ведущего поля, описанная в разделе~\ref{chpt3:GFF}.

\subsubsection{Метод оценки частоты и требуемые статистические свойства данных}
ДОПИСАТЬ

Также отметим, что существуют методы измерения ЭДМ элементарных частиц, не основанные на принципе замороженного спина; например~\cite{COSY:SpinTuneMapping}