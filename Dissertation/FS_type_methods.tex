\section{Методы, основанные на FS-методологии} \label{sect1_2}
\subsection{Метод BNL FS}
BNL FS метод, предложенный коллаборацией, занимающейся разработкой
метода измерения ЭДМ в накопительном кольце в Брукхейвенской
Национальной Лаборатории (США) в 2008 году,~\cite{BNL:Deuteron2008}
это метод для комбинированного кольца. Пучок продольно-поляризованных
дейтронов инжектируется в кольцо; с помощью поляриметрии наблюдается
его спин-прецессия в вертикальной и горизонтальной плоскостях; ЭДМ
сигнал --- это изменение вертикальной компоненты поляризации со
временем, выражаемое как:~\cite[стр.~8]{BNL:Deuteron2008}
\begin{equation}
  \Delta P_V = P\frac{\w_{edm}}{\W}\sin\bkt{\W t + \Theta_0},
\end{equation}
где $\W = \sqrt{\w_{edm}^2 + \w_a^2}$, $\w_a,~\w_{edm}$ угловые
скорости вызываемые, соответственно, магнитным и электрическим
дипольными моментами.

Прилагая радиальное магнитное поле $E_r$~\eqref{eq:FS_Er}, ожидается
уменьшение компоненты $\w_a$ по крайней мере на фактор $10^9$; ввиду
малости величины гипотезируемой $\w_{edm}$, $\Delta P_V \approx P
\w_{edm} t$, и максимальная величина $\Delta P_V$ возрастает в $10^9$.

Ожидаемая чувствительность эксперимента $10^{-29}~e\cdot cm$ за $10^7$
секунд (6 месяцев) полного времени измерения. На этом уровне
чувствительности, асимметрия сечения $\varepsilon_{LR} \approx 5\cdot
10^{-6}$ для наименьших практически реализуемых значений
$\w_a$.~\cite[стр.~18]{BNL:Deuteron2008} Последнее обстоятельство
ставит серьёзную проблему для поляриметрии.~\cite{Mane:SpinWheel} Один
из вариантов её решения лежит в применении внешнего радиального
магнитного поля и измерении общей частоты прецессии засчёт МДМ и ЭДМ
вместе. Это основа так называемого метода Спинового Колеса (Spin Wheel), о котором в
следующем разделе. 

Единственный известный систематический эффект спиновой динамики
первого порядка это присутствие ненулевой средней вертикальной
компоненты электрического поля $\avg{E_V}$. В этом случае, спин будет
прецессировать вокруг радиального направления с частотой~\cite[стр.~11]{BNL:Deuteron2008}
\[
\w_{syst} \approx \frac{\mu\avg{E_V}}{\beta c\gamma^2}.
\]
Здесь важно рассмотреть два обстоятельства:
\begin{itemize}
\item присутствие $\avg{E_V}\neq 0$ вызвано ошибкой юстировки
  элементов ускорителя;
\item этот систематический эффект меняет знак при инжекции пучка в
  обратном направлении.
\end{itemize}
Последнее обстоятельство является причиной структуры инжекции пучка
использованной в этом методе (сначала по-часовой, потом
против-часовой; CW/CCW). Хотя $\w_{syst}$ меняет знак при смене
направления движения пучка, эта методология тем не менее не учитывает
его \emph{величину}. В разделах~\ref{sec:systematic_error-fake_signal} и~\ref{sec:simulation-fake_signal}, 
мы
показываем, что при реалистичной величине ошибки установки
спин-ротаторов 100 мкм, частота МДМ прецессии вокруг радиальной оси
находится на уровне 50--100 рад/сек.~\cite{Senichev:FDM} В связи с
этим, невозможно использовать данную методологию в её оригинальном варианте.

Также, стоит отметить, что при попытке уменьшения $\w_{syst}$, увеличивается влияние
так называемой ошибки геометрической фазы.~\cite[стр.~6]{BNL:Proton}

\subsection{Spin Wheel концепция}
Озвученные выше проблемы с поляриметрией и высокой скоростью прецессии
решаются в Spin Wheel (SW) модификации, предложенной проф. И. Коопом
(Новосибирский Государственный Университет). Основная идея метода в
следующем: сначала, обеспечивается условие замороженного спина; затем
включается радиальное магнитное поле величины $B_x$, достаточно сильное чтобы
вызвать вращение спина с частотой порядка 1 Гц. Поскольку поле
радиальное, вызванная им МДМ прецессия сонаправлена с ЭДМ, а значит
они складываются линейно: $\w \propto \W_{MDM} + \W_{EDM}$.

ЭДМ вклад вычисляется сравнением циклов с обратными знаками $B_x$:~\cite[стр.~1963]{Koop:IPAC13}
\[
\W_{EDM} = \frac{\W_x(+B_x) + \W_x(-B_x)}{2}.
\]

Внешнее поле также вызовет разделение орбит
пучков.~\cite[стр.~1963]{Koop:IPAC13} Это разделение может быть
измерено на уровне пико-метров SQUID магнетометрами; его предлагается
использовать для калибровки внешнего поля.

Поскольку из-за внешнего поля прецессия вокруг радиальной оси на 10
порядков выше чем в оригинальном предложении, значительно упрощается
задача для поляриметрии. Однако, существуют сомнения в возможности
измерить вызываемое внешним полем разделение орбит даже при помощи SQUIDов.

Также, проблема паразитного поля, вызванного ошибкой юстировки, не решена.

\subsection{Метод Frequency Domain}\label{sec:FDM_concept}
Методология Frequency Domain (далее FDM)~\cite{Senichev:FDM} была разработана специально для решения проблемы неточности установки магнитов, и возникающего в связи с этим паразитного МДМ вращения спина. Как было обозначено выше, частота вращения спина в вертикальной плоскости, связанная с магнитным дипольным моментом, при реалистичной ошибке юстировки, находится на уровне 8--16 Гц, что делает невозможным наблюдение медленного нарастания вертикальной компоненты поляризации, связанное с наличием у частицы электрического дипольного момента, как предполагается оригинальным BNL FS методом. В FDM, ЭДМ-эффект вычисляется путём сравнения комбинированной (МДМ + ЭДМ) частоты прецессии, наблюдаемой при циркуляции пучка в прямом и обратном направлениях. Поскольку при смене полярности ведущего поля $\vec B \mapsto -\vec B$, $\vec\beta \mapsto -\vec\beta$, и $\vec E \mapsto \vec E$:
\begin{subequations}
  \begin{align}
    \w_x^{CW/CCW} &= \w_x^{MDM, CW/CCW} + \w_x^{EDM, CW/CCW}, \\
    \w_x^{MDM, CW} &= -\w_x^{MDM, CCW} \equiv \w_x^{MDM}, \label{eq:FDM_CW_CCW_MDM} \\
    \w_x^{EDM, CW} &= \w_x^{EDM, CCW} \equiv \w_x^{EDM},
    \intertext{поэтому, ЭДМ эстиматор}
    \hat\w_x^{EDM} &:= \frac12\bkt{\w_x^{CW} + \w_x^{CCW}} \label{eq:FDM_estimator} \\
                  &= \w_x^{EDM} + \underbrace{\frac12\bkt{\w_x^{MDM, CW} + \w_x^{MDM, CCW}}}_{\varepsilon \to 0}.
  \end{align}
\end{subequations}

Для того, чтобы гарантировать малость $\varepsilon$ по сравнению с требуемой точностью измерений, т.е., что уравнение~\eqref{eq:FDM_CW_CCW_MDM} выполняется достаточно точно, была разработана специальная процедура смены полярности ведущего поля, описанная в разделе~\ref{sec:field_flipping-theory}.

\subsubsection{Метод оценки частоты и требуемые статистические свойства данных}

\subsubsection{Смена полярности ведущего поля}\label{sec:field_flipping-theory}
Как было описано в разделе~\ref{sec:FDM_concept}, для того, чтобы исключить МДМ-эффект из конечной статистики эксперимента, построенного на основе Frequency Domain методологии в комбинированном накопительном кольце, необходимо произвести смену полярности ведущего магнитного поля. Электростатическое поле $E_r = \frac{GB_yc\beta\gamma^2}{1-G\beta^2\gamma^2}$ (см. раздел~\ref{sec:FS_in_a_ring}) при этом фиксировано.

Частоты прецессии спинов частиц пучка определяются по формуле~\cite[стр.~4]{COSY:SpinTuneMapping}
\[
(\W_x, \W_y, \W_z) = 2\pi\cdot f_{rev} \cdot \nu_s \cdot \bar n,
\]
где $f_{rev}$ есть циклотронная частота частицы, а $\nu_s$ и $\bar n$ --- её спин-тюн и ось стабильного спина, соответственно. В разделе~\ref{sec:decoherence_simulations} мы приведём свидетельства того, что при использовании секступольных полей выравниваются не только спин-тюны частиц, но и направления их осей стабильного спина, в связи с чем в дальнейшем рассмотрении мы будем предполагать что спин-векторы всех частиц в пучке вращаются вокруг $\bar n^{CO}$, определённой на референсной орбите. Таким образом, при смене полярности ведущего поля достаточно восстановить эффективный Лоренц-фактор пучка, для того, чтобы восстановить величину угловой скорости паразитной МДМ прецессии.

Калибровка $\gamma_{eff}$ выполняется напрямую, через восстановление угловой скорости прецессии спина в горизонтальной плоскости:
В начальном состоянии, $\W_x\gg \W_y, \W_z$, и $\bar n^{CO}\approx \hat x$. Используя спин-суппрессор (Вин-фильтр), мы подавляем прецессию вокруг вектора $\hat x$; одновременно с этим, мы отходим от ``замороженного'' значения энергии (это делается для того, чтобы избежать неустойчивого состояния ``заморозки'' спина во всех плоскостях). При изменении энергии пучка, меняется также и величина ведущего поля, за тем, чтобы сохранить референсную орбиту. Горизонтальная прецессия становится доминантной, и $\bar n^{CO} \approx \hat y$. После смены полярности ведущего поля, мы опять подстраиваем его величину таким образом, чтобы восстановить условие замороженности спина в горизонтальной плоскости. Тогда, при выключении спин-суппрессора, и возвращении энергии пучка на изначальный уровень, мы получаем $\bar n^{CO} \approx -\hat x$, $\gamma_{eff}^{CCW} = \gamma_{eff}^{CW}$, то есть, МДМ прецессия происходит с той же угловой скоростью, но в обратном направлении. 

Также отметим, что существуют методы измерения ЭДМ элементарных частиц, не основанные на принципе замороженного спина; например~\cite{COSY:SpinTuneMapping}

\subsection{Общая классификация методов FS-типа}
Все методы типа замороженного спина можно разделить на две категории: 
\begin{enumerate*}[\itshape i\upshape)]
	\item Методы пространственной области (Space Domain): в которых суждения о величине ЭДМ базируются на изменении пространственной ориентации вектора поляризации пучка (измеряемой углом между начальным и конечным ориентациями вектора поляризации), и 
	\item Методы в частотной области (Frequency Domain): такие, в которых величина ЭДМ оценивается на основе 
	скорости изменения обозначенного выше угла.
\end{enumerate*} 

Эти две категории взаимно-исключающие: для того, чтобы измерить
угол поворота вектора поляризации, необходимо обеспечить малость угловой скорости поворота спин-векторов
частиц; для того, чтобы измерить частоту их прецессии, необходимо наблюдение некоторого числа
колебаний вертикальной компоненты поляризации, а значит обеспечить достаточно высокую угловую скорость
прецессии спин-векторов. BNL FS относится к Space Domain категории методов, в то время как SW --- это
метод из Frequency Domain.

Методы из пространственной области принципиально обладают некоторыми недостатками, как то:
\begin{itemize}
	\item Ввиду малости измеряемой асимметрии сечения, они характеризуются трудной поляриметрией~\cite[стр.~6]{Mane:SpinWheel}.
	\item Поскольку для них требуется минимизация любых факторов поворота спин-вектора частицы, кроме ЭДМ,
	чья угловая скорость, если не нулевая, достаточно мала, они характеризуются отсутствием выделенного 
	направления оси стабильного спина. Такое состояние можно называть трёхмерным состоянием замороженного спина
	(3D FS), в отличие от двухмерного состояния замороженного спина, которое создаётся при приложении к
	пучку Спинового Колеса. Состояние трёхмерно-замороженного спина нестабильно, и требует серьёзных усилий
	по экранированию: в нём, плоскость прецессии спина может быть отклонена любым локальным возмущением магнитного поля.
\end{itemize}

Методы частотной области лишены этих недостатков, поскольку в них стремятся максимизировать величину
угловой скорости поворота вектора полиризации.
