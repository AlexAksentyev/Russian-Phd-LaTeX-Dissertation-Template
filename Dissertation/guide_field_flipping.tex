
\newcommand{\Traj}{\mathcal T}
\DeclareDocumentCommand{\Stab}{s}{\mathcal{S}\IfBooleanT{#1}{\vert_{\W_y=0}}}
\newcommand{\Fail}{\mathcal F}
\DeclareDocumentCommand{\g}{s}{\gamma\IfBooleanT{#1}{_{eff}}}
\newcommand{\CO}{\mathrm{CO}}


\subsection{Алгоритм калибровки}
Пусть $\Traj$ обозначает множество всех возможных траекторий частицы в ускорителе. $\Traj = \Stab \bigcup \Fail$,
где $\Stab$ это все стабильные траектории, а $\Fail$ --- это такие траектории,
при попадании на одну из которых частица теряется из пучка.

Калибровка производится в два этапа:
\begin{enumerate}
\item На первом этапе величина поля выставляется таким образом, чтобы частицы инжектированного пучка
попадали на траектории $t\in \Stab$. В первом приближении, это будет та же величина,
что и для обратно-йиркулирующего пучка, но с противоположным знаком.
\item Затем величина поля уточняется, путём удовлетворения условия замороженности спина в горизонтальной
плоскости. При выполнении этого условия, из $\Stab$ выбирается подмножество $\Stab*$ тракеторий, для которых
$\W_y = 0$.
\end{enumerate}

Предположим, что $\W_y = \W_y(\g*)$ --- инъективная функция, а значит
$\W_y(\g*^1) = \W_y(\g*^2) \rightarrow \g*^1 = \g*^2$. Пространство траекторий делится на
классы эквивалентности по величине эффективного Лоренц-фактора: траектории с одинаковым $\g*$ эквивалентны
с точки зрения спин-динамики (то есть, обладают одним и тем же значением спин-тюна $\nu_s$ и направлением
оси стабильного спина $\nbar$), и принадлежат одному классу. Поскольку $\W_y$ инъективная, значит существует
уникальное $\g*$, один класс эквивалентности, при котором $\W_y=0$: $[\W_y=0]\equiv [\g*^0] = \Stab*$.

Если бы в структуре кольца не использовались секступоли, $\Stab*$ было бы синглетоном (множеством с
единственным элементом). В разделе~\ref{chpt3:decoherence}, мы уже показали, что
при использовании секступолей, $\forall t_1,t_2\in\Stab*$:
$\nu_s(t_1) = \nu_s(t_2)$, $\nbar(t_1) = \nbar(t_2)$, и значит $\Stab*$ содержит
несколько траекторий.~\footnote{Строго говоря, как видно из симуляций рисунка~\ref{decoh:fig:ST_vs_y0_GSY}
раздела~\ref{chpt3:decoherence}, даже при использовании секступольных полей, сохраняется некоторая,
пренебрежимо малая, зависимость спин-тюна от эффективного Лоренц-фактора. В связи с этим, равенства
здесь приближённые, и множество $\Stab*$ следует считать нечётким множеством: мы будем полагать траектории,
на которых $|\W_y| < \delta$ для некоторого малого $\delta$, как принадлежащими классу $[\W_y=0]$.}

Тогда, чтобы подвтердить валидность калибровочной процедуры, нам нужно показать следующее:
\begin{enumerate}
\item $\Stab*^{CCW} = \Stab*^{CW}$ --- то есть, что и я прямом, и в обратном случае циркуляции пучка,
  $\W_y = 0$ для одних и тех же траекторий (эквивалентно, $\W_y=0$ при одном и том же $\g*$ и в CW, и в CCW
  случаях);
\item $\forall t_1,t_2\in\Stab*^{CCW}$: $\nu_s(t_1) = \nu_s(t_2)$, $\nbar(t_1) = \nbar(t_2)$ ---
  то есть, те же самые секступольные поля подавляют декогеренцию и прямого, и обратного пучков.
\end{enumerate}

Для выполнения этой задачи мы:
\begin{enumerate}
\item строим зависимости $\nu_s(z),~z\in\{x,y,\delta\}$ для CW и CCW пучков;
\item вычисляем их невязку $\epsilon(z) = \nu_s^{CW} - \nu_s^{CCW}$.
\end{enumerate}

Если невязка мала в широком диапазоне $z$, значит:
\begin{inparaenum}[1)]
\item секступольная когеренция работает без изменений значения градиентов для обоих пучков, и
\item спин-тюн (соответственно $\g*$) одинаков для обоих пучков, и значит их Спин-Колёса
вращаются с одинаковой скоростью.
\end{inparaenum}

Углы наклона $\nbar^{CW}$, $\nbar^{CCW}$ по отношению к горизонтальной плоскости определяются точностью установки
$\W_y=0$.

\subsection{Симуляция}
