\chapter{Стркутуры накопительных колец с ``замороженным,'' и ``квази-замороженным'' спином} \label{chpt:FS_and_QFS_lattices}

Существуют два подхода к проблеме построения накопительного кольца для измерения ЭДМ дейтрона: 
\begin{enumerate*}[\itshape a\upshape)]
	\item структура с ``замороженным'' спином (FS), и 
	\item структура с ``квази-замороженным'' спином (QFS).
\end{enumerate*}

В следующих разделах мы рассмотрим возможные варианты колец обоих типов.

\section{Структура с ``замороженным'' спином} \label{sec:FS_BNL_lattice}
В структуре FS-типа, горизонтальная проекция вектора спина частицы пучка \emph{непрерывно} сонаправлена с вектором её импульса. Для реализации условия непрерывности, в такой структуре используются цилиндрические спин-ротаторы, создающие одновременно и электростатическое, и магнитное поля. На Рисунке~\ref{fig:BNL_lattice} представлен вариант кольца FS-типа.~\cite{Senichev:Lattices} Данное кольцо имеет длину 145.85 м, и рассчитано на инжекцию пучка дейтронов на энергии 270 МэВ. В структуре предусмотрено использование ВЧ-резонатора для подавления линейного эффекта декогеренции спина путём усреднения энергии вокруг значения равновесной энергии частицы. Продольное напряжение резонатора $V = 100$ кВ, частота поля $f_{RF} = 5\cdot f_{rev}$, где $f_{rev} = 1$ МГц --- частота оборота пучка.

\begin{figure}[h!]
	\centering
	\includegraphics[width=\linewidth]{images/chapter2/BNL_lattice}
	\caption{Вариант кольца, построенного по принципу ``замороженного'' спина. (Рисунок взят из~\cite{Senichev:Lattices})\label{fig:BNL_lattice}}
\end{figure}

Основная цель FS-концепции кольца --- максимизация ЭДМ сигнала. Однако, следует обратить внимание на то, что строгое выполнение условия замороженности спина возможно только для референсной частицы. Это связано с тем, что, как следует из уравнения~\eqref{eq:TBMT_MDM}, для заданных E-, B-полей, существует уникальное значение Лоренц-фактора $\gamma$, при котором $\W_y^{MDM} = 0$. Таким образом, даже в FS-структуре, спин-векторы большинства частиц ``заморожены'' лишь приблизительно.

\section{Структура с ``квази-замороженным'' спином} \label{sec:QFS_concept}
В QFS-концепции кольца отходят от условия непрерывности сонаправленности векторов поляризации и импульса пучка. Вместо этого, требуется лишь равенство нулю \emph{совокупного за оборот} угла поворота вектора поляризации относительно импульса в электростатических ($\Phi_s^E$) и магнитных ($\Phi_s^B$) элементах:~\cite{Senichev:Lattices}
\begin{equation*}
	\sum_i \Phi_{s,i}^E = -\sum_j \Phi_{s,j}^B.
\end{equation*}

Как следует из определения спин-тюна (см. раздел~\ref{sec:TBMT_introduction}), угол поворота спин-вектора частицы относительно её импульса в электромагнитном поле $\Phi_s = \nu_s \cdot \Phi$, где $\Phi$ угол поворота импульса, а $\nu_s$ спин-тюн.

Угловая скорость поворота вектора импульса частицы в магнитном поле $\vec B$ есть 
\[
\w_B = \frac qm \frac B \gamma,
\]
в электростатическом $\vec E$:
\[
\w_E = \frac qE \frac{\vec E\times \vec\beta}{c\beta^2\gamma},
\]

из чего следуют выражения для спин-тюна частицы в электростатическом и магнитном полях:
\begin{equation}
	\begin{cases}
		\nu_s^B &= \gamma G, \\
		\nu_s^E &= \beta^2\gamma\bkt{\frac{1}{\gamma^2-1} - G}.
	\end{cases}
\end{equation}

Преимущество кольца QFS-типа над кольцом FS-типа в относительной простоте исполнения: нет необходимости использовать цилиндрические фильтры Вина; в двух вариантах QFS-кольца, рассматриваемых ниже, используются либо прямые фильтры Вина, либо цилиндрические электростатические дефлекторы. С другой стороны, из-за появления вертикальной компоненты оси прецессии спина $\bar n_y$, максимальная амплитуда ЭДМ-сигнала уменьшается по сравнению с полностью замороженным случаем. Однако, коэффициент пропорциональности амплитуд
\[
J_0(\Phi_s) \approx 1 - \frac{\Phi_s^2}{4},
\]
в случае рассматриваемых структур, не ниже 0.98.~\cite[стр.~18]{Senichev:Lattices}

\subsection{Структура с кодовым названием E+B}\label{sec:QFS_EB_lattice}

\begin{figure}[h!]
	\centering
	\includegraphics[width=\linewidth]{images/chapter2/E+B_lattice}
	\caption{Вариант кольца, построенного по принципу квази-замороженного спина, дизайн с прямыми фильтрами Вина. (Рисунок взят из~\cite{Senichev:Lattices})\label{fig:QFS_E+B_lattice}}
\end{figure}

\subsection{Структура с кодовым названием 6.3}\label{sec:QFS_6_3_lattice}

\begin{figure}[h!]
	\centering
	\includegraphics[width=\linewidth]{images/chapter2/6_3_lattice}
	\caption{Вариант кольца, построенного по принципу квази-замороженного спина, дизайн с разделением E- и B-полей. (Рисунок взят из~\cite{Senichev:Lattices})\label{fig:QFS_6_3_lattice}}
\end{figure}