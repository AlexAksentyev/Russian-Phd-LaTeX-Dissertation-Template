Структуру $(\nbar, \w_{SW}, \vec P)$, визуально представленную на Рис.~\ref{fig:SpinWheel},
будем называть спин-колесом. Драйвером спин-колеса будем называть электро-магнитное поле $(\vec E, \vec B)$,
определяющее направление оси $\nbar$ колеса, и его скорость вращения $\w_{SW}$.

Спин-колесо может быть устойчивым (если дисперсия направлений $\nbar$ пренебрежимо мала), 
или неустойчивым (в обратном случае). Устойчивость спин-колеса напрямую зависит от 
величины угловой скорости вращения колеса $\w_{SW}$. 
 
В методе spin wheel в качестве драйвера спин-колеса используется поле 
специально введённого в оптическую структуру кольца магнитного диполя; альтернативно, 
под ту же цель можно приспособить поля неидеальности машины (скорость вращения колеса при этом
 определяется уравнением~\eqref{eq:ImperfectionWheelRollRate}). 
Последний подход имеет смысл, в рамках методологии измерения ЭДМ частицы, потому, 
что $\w_{syst}$ меняет знак при смене направления движения пучка.~\cite[стр.~11]{BNL:Deuteron2008}
 
 \begin{figure}[h]\centering
\begin{tikzpicture}[>=stealth, line width=.8pt]
%\draw[help lines] (-2,-2) grid (4,4);
\node (c) at (0,0) {.};
\node[blue,label=above:{$\vec\w_{SW}$, $\vec B$}] (B) at ($(c)+(153:4.6cm)$) {};
\node[red,label=above:$\bar n$] (nbar) at ($(c)+(153:2.4cm)$) {};
\node[label=above:$\vec P$] (sp) at ($(c)+(63:3cm)$) {};
\draw[rotate=60, dashed,
decoration={markings, 
	mark=at position 1 with {\arrow{<}}},
postaction={decorate}]  circle [x radius=3cm, y radius=2cm];
\draw[->] (c.center) -- (sp.center);
\draw[->] (c.center) -- (B.center);
\draw[->, red] (c.center) -- (nbar.center);
\end{tikzpicture}
 	\caption{Спин-колесо: в частном случае, ось стабильного спина $\nbar$, вокруг которой вращается вектор поляризации $\vec P$, определяется направлением магнитного поля-драйвера колеса $\vec B$.\label{fig:SpinWheel}}
 \end{figure}