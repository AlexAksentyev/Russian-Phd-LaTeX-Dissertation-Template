\subsection{Фальш-сигнал}\label{sec:systematic_error-fake_signal}
Систематические ошибки, вызванные физическими неидеальностями
ускорителя, включая неточность юстировки оптических элементов,
вызывают фальш-сигнал ЭДМ.~\cite[стр.~230]{Eremey:Thesis} Особенно в
этом отношении проблематичны наклоны элементов вокруг оптической оси, поскльку они
индуцируют паразитные горизонтальные компоненты магнитного поля $B_x$
и $B_z$, которые обе вращают спин в вертикальной плоскости; той, в которой измеряется ЭДМ.

Ю. Сеничевым были сделаны~\cite{Senichev:FDM} аналитические оценки МДМ частоты прецессии спина
вокруг радиальной оси. Из уравнения Т-БМТ, и выражения силы Лоренца,
скорость МДМ прецессии вокруг радиальной оси есть
\begin{equation}
\SD{\W_x^{MDM}} = \frac{q}{m\gamma}\frac{G+1}{\gamma}\frac{\SD{B_x}}{\sqrt{n}},
\end{equation}
где $n$ есть число наклонённых спин-ротаторов, и $\SD{B_x} = B_y
\SD{\delta h}/L$, при стандартном отклонении ошибки юстировки
$\SD{\delta h}$. При величине ошибки $\SD{\delta h} = 100$ мкм, и
длине дефлектора $L=1$ м, $\SD{\W_x^{MDM}} \approx 100$ рад/сек.~\cite{Senichev:FDM}

Мы изучили спиновую динамику в структурах с замороженным и
квази-замороженным спином в присутствии наклонов оптических элементов
с помощью кода COSY INFINITY. Наши симуляции согласуются с оценками,
представленными выше.

\paragraph{Имплементация паразитного поля.}\label{sec:error_field_implementation}
Имплементируя неидеальности полей, мы следовали рекомендациям
изложенным в~\cite[стр.~235]{Eremey:Thesis}. Малое возмущение
магнитного поля, в первом приближении, действует как маленький пропорциональный поворот
спин-вектора. Поэтому мы имплементировали наклон E+B элемента как
домножение соответствующей матрицы поворота на его спиновую матрицу
перехода, ``спин-кик.''

В соответствии с уравнением~\eqref{eq:TBMT_MDM}, изменение МДМ частоты
прецессии, ассоциированное с введённым паразитным полем $(B_x, 0, B_z)$ есть
\begin{align*}
	\Delta\W_{MDM} &= \frac qm (B_x, 0, B_z),
	\intertext{поэтому угол спин-кика равен}
	\Theta_{kick} &= t_0\Delta\W_{MDM},
\end{align*}
где $t_0 = L/v_0$ пролётное время референсной частицы через элемент.
