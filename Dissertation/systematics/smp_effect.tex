
\subsection{Мотивация изучения эффекта}
Инвариантная спиновая ось частицы, учавствующей в бетатронном движении, колеблется вокруг своего референсного значения.~\cite[стр.~11]{Shatunov} По этой причине, \hl{амплитуда решения уравнения}\footnote{Здесь бы не помешала ссылка.} Т-БМТ для вертикальной компоненты спин-вектора:
\begin{align}
s_y &= \sqrt{\bkt{\frac{\w_y\w_z}{\w}}^2 + \bkt{\frac{\w_x}{\w}}^2}\cdot\sin\bkt{\w\cdot t + \phi}\notag\\
&= \sqrt{\bkt{\bar n_y\bar n_z}^2 + \bar n_x^2} \cdot \sin\bkt{2\pi\cdot\nu_s\cdot n_{turn} + \phi},\label{eq:sy_varying_amplitude}
\end{align}
превращается в изменяющуюся во времени функцию. Если вариация оси стабильного спина (а также спин-тюна частицы) имеет достаточно большую амплитуду, использование гармонической функции с постоянными параметрами в качестве модели для фитирования сигнала повлечёт за собой систематическую ошибку спецификации модели. Ошибки данного типа отражаются на валидности оценок параметров модели, то есть на оценке частоты, и потому требуют анализа.

Вариация спин-тюна $\nu_s$ особенно проблематична в этом отношении, т.к. она напрямую влияет на фазу сигнала; однако, эта проблема может быть решена введением в ускорительную структуру секступольных полей, как описано в разделе~\ref{sec:sextupole_spin_dec_solution}. В связи с этим, в настоящем разделе мы сфокусируемся на рассмотрении вариации $\bar n$.

\subsection{Симуляция}
Симуляция проводилась следующим образом: частица, смещённая с референсной орбиты в вертикальном направлении
на 0.3 мм, многократно инжектируется в неидеальную структуру с замороженным спином~\cite{Senichev:Lattices},
использующую секступоли для подавления декогеренции, вызванной бетатронными колебаниями
в вертикальной плоскости (см. раздел~\ref{sec:sextupole_spin_dec_solution}).
Неидеальности структуры симулируются наклонами E+B элементов.
Введённые таким образом неидеальности не ведут к возмущению референсной орбиты (то есть,
референсная орбита --- равно как и орбита бетатрон-осциллирующей частицы --- одинакова для всех инжекций.)

На каждой инжекции, углы наклонов E+B элементов генерируются случайным образом из
нормального распределения $\alpha\sim N(\mu_i, 5\cdot 10^{-6})$ рад, $i\in\{1,\dots,11\}$, где
$\mu_i$ изменяется в диапазоне $[-5\cdot10^{-5}, +5\cdot10^{-5}]$ рад. Ненулевые ожидания $\mu_i$
симулируют введение в систему Кооп-Колеса.~\cite{Koop:SW} Величины $\mu_i$ и $\sigma_{\alpha}$
выбраны с целью детализации эффекта. При больших значениях, труднее различимы эффекты влияния вариации
$\nu_s$ и $\bar n$.

Трэкинг спина выполнялся с помощью кода COSY Infinity~\cite{COSYINF:Website}, на протяжении $1.2\cdot10^6$
оборотов; каждые 800 оборотов $\nu_s$ и $\bar n$ вычисляются (процедурой
TSS~\cite[стр.~41]{COSYINF:BeamPhysMan}) в точке фазового пространства, занимаемой частицей на данный момент,
что даёт нам серию $(\nu_s(n), \bar n(n))$, $n$ --- номер оборота частицы в ускорителе.
Соответствующие компоненты спин-вектора $(s_x^{trk}(n), s_y^{trk}(n), s_z^{trk}(n))$,
вычисленные трэкером (процедура TR~\cite[p.~41]{COSYINF:BeamPhysMan}), составляют вторую серию данных,
используемых в анализе.

\section{Анализ}
Используя данные первой серии, мы сгенерировали ожидаемую $s_y^{gen}(t)$ ``генераторную'' серию,
в соответствии с уравнением~\eqref{eq:sy_varying_amplitude}, а также ``идеальную'' серию $s_y^{idl}$, в которой
мы положили постоянные значения $\nu_s = \avg{\nu_s(t)}$ и $\bar n =\avg{\bar n(t)}$. 

Наша гипотеза состоит в том, что бетатронное движение частицы
должно ввести несоответствие между синусоидальной моделью
\begin{equation}
  f(t) = a\cdot\sin(\w\cdot t + \delta),\label{eq:fit_model}
\end{equation}
и данными трекера, путём вариации оси прецессии спина $\bar n$, а значит амплитуды
фитируемого сигнала. ``Идеальная'' серия служит базой сравнительного анализа,
так как она идеально соответствует модели; ``генераторная'' серия учитывает вариацию $\bar n$,
всё ещё оставаясь в пределах модели. ``Трекерная'' серия --- наиболее близкое приближение
к реальным измерительным данным.

Для сравнения серий между собой, мы
\begin{enumerate*}[\itshape a\upshape)]
\item вычислили и проанализировали невязки $\epsilon_1(t) = s_y^{gen}(t) -
  s_y^{idl}(t)$ и $\epsilon_2(t) = s_y^{trk}(t) - s_y^{idl}(t)$;
\item профитировали модель~\eqref{eq:fit_model} к трём сериям данных, и
  сравнили качество фита;
\item вычислили стандартные отклонения компонент $\bar n$ при каждом
  значении скорости Спин-Колеса.
\end{enumerate*}

\begin{figure}[h]
  \centering
    \includegraphics[height=.35\paperheight]{images/smp_sim/residual_amplitude_vs_SW(both)}
  \caption{Амплитуды сравнительных невязок серий, как функции относительной скорости Спин-Колеса.
    Верхняя панель: невязка $\epsilon_1$; нижняя панель: невязка $\epsilon_2$\label{fig:residuals}}
\end{figure}

\begin{figure}[h]
  \centering
	\subbottom[Компонент $\bar n$\label{fig:sd:nbar}]{%
    \includegraphics[height=.35\paperheight]{images/smp_sim/NBAR_variation_sd_vs_SW}}
	\subbottom[Сравнительных невязок.
	Верхняя панель: невязка $\epsilon_1$; нижняя панель: невязка $\epsilon_2$\label{fig:sd:res}]{%
    \includegraphics[height=.35\paperheight]{images/smp_sim/residual_SD_vs_SW(both)}}
  \caption{Стандартные отклонения против относительной скорости Спин-Колеса\label{fig:sd}}
\end{figure}

\begin{table}[h]
  \caption{Оценки параметров модели (надо переписать)\label{tbl:param_estimates}}
  \begin{tabular}{r|rllr}
    \toprule
    Series & Par. & Value & St.Error & AIC \\
    \midrule
    \multirow{3}{*}{$s_y^{idl}$} & $\hat f$ & 74.452466549766214 & $6\cdot10^{-15}$ & \multirow{3}{*}{-86246} \\
    & $\hat a$ & 0.99841729771960 & $1\cdot10^{-14}$ & \\
    & $\hat\delta$ & 3.14159265358978 & $2\cdot 10^{-14}$ &\\
    \hline
    \multirow{3}{*}{$s_y^{gen}$} & $\hat f$ & 74.452466548 & $1\cdot 10^{-9}$ & \multirow{3}{*}{-49917} \\
    & $\hat a$ & 0.998417300 & $2\cdot 10^{-9}$ & \\
    & $\hat\delta$ & 3.1415926564 & $4\cdot 10^{-9}$ &\\
    \hline
    \multirow{3}{*}{$s_y^{trk}$} & $\hat f$ & 4.3276781 & $6\cdot 10^{-7}$ & \multirow{3}{*}{-30665} \\
    & $\hat a$ & 0.998418 & $1\cdot10^{-6}$ & \\
    & $\hat\delta$ & 3.141589 & $3\cdot 10^{-6}$ &\\
    \bottomrule
  \end{tabular}
\end{table}

На Рисунке~\ref{fig:residuals} мы наблюдаем, что ``генераторная'' серия почти идентична
``идеальной'' серии, при $\epsilon_1 \le 6\cdot10^{-6}$ в течении длительности цикла,
в то время как ``трэкерная'' серия отклоняется от неё на уровне
$\epsilon_2 \le 3.5\cdot 10^{-5}$. This discrepancy, which can get as high as four order of magnitude at
higher SW strengths (not shown here), between $\epsilon_1$ and $\epsilon_2$ has no explanation as of yet.

In Figure~\ref{fig:sd:res} we see that both residual standard deviations exhibit the same pattern as
the standard deiviation of $\nu_s$ (Figure~\ref{fig:sd:nbar}, bottom panel), but not the $\bar n$ components.
This is an indication that frequency variation is a much more significant factor in the mismatch between
model~\eqref{eq:fit_model} and tracker data, than is the presumed amplitude variation due to
the change of orientation of $\bar n$.

Table~\ref{tbl:param_estimates} characterizes the fit model's goodness-of-fit with respect to the time series.


\section{Выводы}
Вопрос влияния бетатронного движения на ЭДМ статистику в FD-методологии следует рассматривать
ввиду трёх обстоятельств:
\begin{enumerate}
\item Осцилляции амплитуды сигнала очень малы. Они происходят на уровне не более $10^{-4}$ (при
  $\sigma_{\alpha}=5\cdot 10^{-4}$), тогда как ожидаемая неточность измерений поляризации находится
  на уровне процентов. Это значит суперпозиция систематической ошибки и случайной ошибки измерения
  не будет проявлять статистически-значимую систематичность.
\item Коэффициент корреляции между оценками амплитуды и частоты не значителен. Колебания амплитуды
  влияюи на оценку $\hat a$ в первую очередь; их эффект на оценку $\hat\w$ опосредован, и описывается
  коэффициентом корреляции. Поскольку он меньше 10\%, даже если колебания окажутся достаточными, чтобы повлиять
  на оценку амплитуды, их эффект на оценку частоты будет уменьшен по крайней мере в 10 раз.
\item Этот систематический эффект контролируется. И этот фактор является основным достоинством методологий
  частотной области. Вводя в систему внешнее Спин-Колесо, колебания $\bar n$ могут быть непрерывно минимизированы
  до необходимого уровня, без каких-либо модификаций паттерна эксперимента.
\end{enumerate}


\subsection{Анализ}
