\subsection{Декогеренция спинов частиц пучка в идеальном и неидеальном кольцах}
Когеренцией спина называется мера или качество сохранения поляризации
в изначально поляризованном пучке.~\cite[стр.~205]{Eremey:Thesis}

Когда поляризованный пучок инжектируется в накопительное кольцо, спин
векторы частиц пучка начинают прецессировать вокруг вертикального
(ведущего) поля. Частота прецессии зависит от равновесного уровня
энергии частицы, который различен для частиц пучка.

Это обстоятельство не является проблемой в том случае, когда начальная
поляризация пучка вертикальна; однако метод измерения ЭДМ в
накопительном кольце, основанный
на принципе замороженного спина требует, чтобы вектор поляризации
пучка был сонаправлен с его вектором импульса, т.е. лежал в
горизонтальной плоскости. Таким образом, декогеренция спина есть
внутренняя проблема метода замороженного спина.
\subsubsection{Требования ко времени когеренции пучка}
Время когеренции спина (spin coherence time; SCT) для метода
замороженного спина, выполненного в накопительном кольце с идеально
установленными элементами определяется минимальным детектируемым углом
отклонения вектора поляризации пучка из горизонтальной плоскости
только засчёт ЭДМ. Для уровня чувствительности $10^{-29}~e\cdot cm$
это примерно $5\cdot10^{-6}$.~\cite{BNL:Deuteron2008}

В соответствии с уравнением Т-БМТ,
\[
\W_{EDM,x} = \eta\frac{qE_x}{2mc},
\]
где $\eta$ есть коэффициент пропорциональности между ЭДМ и спином,
равный $10^{-15}$ для дейтрона,для данного уровня чувствительности.~\cite[стр.~206]{Eremey:Thesis}

Для дейтронного BNL FS кольца, $E_x = 12$
МВ/м,~\cite[стр.~19]{BNL:Deuteron2008} так что $\W_{EDM,x}\approx
10^{-9}$ рад/сек. Таким образом получаем, что для того, чтобы достичь
детектируемый уровень отклонения вектора поляризации на 1 мкрад требуется SCT порядка 1000 секунд.~\cite[стр.~207]{Eremey:Thesis}
\subsubsection{Происхождение декогеренции}
Декогеренция спина в пучке вызвана разницей угловых скоростей
прецессии спинов частиц, которая, в свою очередь, вызвана разницей
длин орбит и импульсов частиц. Это можно видеть исходя из следующих
соображений.

Когда частица со спином входит в область магнитного поля, её спин-вектор
начинает поворачиваться вокруг вектора магнитного поля с угловой
скоростью определяемой уравнением Т-БМТ~\eqref{eq:TBMT_MDM}:
\begin{equation*}
	\vec\W_{MDM} = \frac qm G\vec B.
\end{equation*}
На выходе из области, вектор спина повёрнут на угол
\begin{equation*}
	\theta = \Delta t\cdot\W_{MDM} = \frac Lv \cdot \frac qm GB\cdot \frac{\gamma_0}{\gamma_0} = \frac{L\gamma_0 GB}{B\rho} = \frac L\rho\gamma_0\cdot G,
\end{equation*}
где $L$ есть длина пути внутри области с магнитным полем, и $B\rho =
p/q$ магнитная жёсткость.

В простой модели рассмотренной выше, влияние орбитальной динамики на
спиновую динамику вырадено через $\gamma_0 L/\rho$ (эффективный
Лоренц-фактор). В случае референсной частицы, $\gamma_0L/\rho =
\gamma_0\alpha$, $\alpha$ угол поворота вектора импульса,в то время
как для частицы участвующей в бетатронном движении, эффективный
Лоренц-фактор больше. В следующих разделах мы выразим связь между
спиновой и орбитальной динамиками частицы в накопительном кольце в
более общих терминах.

\paragraph{Сдвиг равновесного значения импульса частицы.}
Продольная динамика заряженной частицы на референсной орбите в
накопительном кольце описывается системой уравнений
\begin{equation*}
	\begin{cases}
		\ddt{\varphi} &= -\w_{RF}\eta\delta, \\
		\ddt{\delta} &= \frac{q V_{RF}\w_{RF}}{2\pi h\beta^2E}\sin\varphi.
	\end{cases}
\end{equation*}
В уравнениях выше: $\varphi$ отклонение фазы частицы от референсной
$\varphi_0 = 0$; $\delta = \frac{\Delta p}{p_0}$ относительное
отклонение импульса от $p_0$ референсной частицы; $V_{RF}$, $\w_{RF}$
амплитуда и частоты колебаний ВЧ поля; $\eta = \alpha_0 - \gamma^{-2}$
слип-фактор, где $\alpha_0$ --- коэффициент сжатия орбиты, определяемый
через $\sfrac{\Delta L}{L} = \alpha_0\delta$, $L$ длина орбиты; $h$
гармоническое число; $E$ полная энергия ускоряемой частицы. $\w_{RF} =
2\pi h f_{rev}$, где $f_{rev}=T_{rev}^{-1}$ --- частота оборота пучка.

Решения этой системы формируют семейство эллипсов в плоскости
$(\varphi, \delta)$, центрированных на $(0,0)$. Однако, если
рассмотреть частицу, участвующую в бетатронных колебаниях, и
использовать разложение Тейлора более высокого порядка для
коэффициента сжатия орбиты $\alpha = \alpha_0 + \alpha_1\delta$,
первое уравнение системы превратится в:~\cite[стр.~2579]{Senichev:IPAC13}
\[
\ddt{\varphi} = -\w_{RF} \bkt*{\bkt{\frac{\Delta L}{L}}_\beta + \bkt{\alpha_0 + \gamma^{-2}}\delta + \bkt{\alpha_1 - \alpha_0\gamma^{-2} + \gamma^{-4}}\delta^2},
\]
где $\bkt{\frac{\Delta L}{L}}_\beta =
\frac{\pi}{2L}\bkt*{\varepsilon_xQ_x + \varepsilon_yQ_y},$ есть
удлинение орбиты, связанное с бетатронным движением; $\varepsilon_x$ и
$\varepsilon_y$ --- горизонтальный и вертикальный эмиттансы пучка, и
$Q_x$ и $Q_y$ горизонтальный и вертикальный тюны.~\cite[стр.~2580]{Senichev:IPAC13}

Решения модифицированной системы более не центрированы на одной и той
же точке. Удлинение орбиты и отклонение импульса вызывают сдвиг
равновесного значения импульса частицы~\cite[стр.~2581]{Senichev:IPAC13}
\begin{equation}\label{eq:EquLevMom_shift}
\Delta\delta_{eq} = \frac{\gamma_0^2}{\gamma_0^2\alpha_0 - 1}\bkt*{\frac{\delta_m^2}{2}\bkt{\alpha_1 - \alpha_0\gamma^{-2} + \gamma_0^{-4}} + \bkt{\frac{\Delta L}{L}}_\beta},
\end{equation}
где $\delta_m$ --- амплитуда синхротронных колебаний.

\paragraph{Понятие эффективного Лоренц-фактора.}
Равновесное значение энергии, связанное со сдвигом импульса~\eqref{eq:EquLevMom_shift}, и называемое \emph{эффективным Лоренц-фактором}, есть~\cite{Senichev:FDM}
\begin{equation}\label{eq:EffectiveGamma}
\gamma_{eff} = \gamma_0 + \beta_0^2\gamma_0\cdot\Delta\delta_{eq},
\end{equation}
где $\gamma_0$, $\beta_0$ --- Лоренц-фактор референсной частицы и
нормализованное значение
скорости. Уравнения~\eqref{eq:EquLevMom_shift}
и~\eqref{eq:EffectiveGamma} определяют связь между спиновой и
орбитальной динамиками частицы.

Из уравнения для спин-тюна частицы в магнитном поле $\nu_s = \gamma_{eff}\cdot G$ следует, что спин-тюны двух частиц с одинаковыми эффективными Лоренц-факторами равны, независимо от их траекторий в ускорителе. Этот принцип используется при использовании секступольных полей для подавления спиновой декогеренции, а также при смене полярности ведущего магнитного поля кольца.

\subsubsection{Подавление декогеренции с помощью секступольных полей}
Чтобы минимизировать декогеренцию спина, связанную с бетатронным
движением и отклонением импульса, могут быть использованы
секступольные (или октупольные) поля~\cite[стр.~212]{Eremey:Thesis}

Секступоль силы
\[
S_{sext} = \frac{1}{B\rho} \pddx{B_y}[x][2],
\]
где $B\rho$ магнитная жёсткость, влияет на коэффициент сжатия орбиты
первого порядка как~\cite[стр.~2581]{Senichev:IPAC13}
\begin{align}
	\Delta \alpha_{1,sext} &= -\frac{S_{sext}D_0^3}{L}, \label{eq:Sext_compaction_effect}
	\intertext{и одновременно на длину орбиты как}
	\bkt{\frac{\Delta L}{L}}_{sext} &= \mp \frac{S_{sext}D_0\beta_{x,y}\varepsilon_{x,y}}{L}, \label{eq:Sext_OL_effect}
\end{align}
где $D(s,\delta) = D_0(s) + D_1(s)\delta$ обозначает функцию дисперсии.

В следующих разделах мы будем называть декогеренцию, связанную си
горизонтальными/вертикальными бетатронными, и синхротронными
колебаниями соответственно X-/Y-, и D-декогеренцией. 

Из уравнений~\labelcref{eq:Sext_compaction_effect,eq:Sext_OL_effect} можно
видеть, что для подавления декогеренции необходимы три семейства
секступолей, помещённых в максимумы функций: $\beta_x$, $\beta_y$ для подавления
X-,Y-декогеренции, и $D_0$ для D-декогеренции.