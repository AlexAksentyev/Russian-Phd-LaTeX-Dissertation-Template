
\newcommand{\Tlmsq}{Тл/м$^2$}

\subsection{Требования к времени когеренции спина пучка}
Время когеренции спина (spin coherence time; SCT) для метода
замороженного спина, выполненного в накопительном кольце с идеально
установленными элементами определяется минимальным детектируемым углом
отклонения вектора поляризации пучка от горизонтальной плоскости
только за счёт ЭДМ. Для уровня чувствительности $10^{-29}~e\cdot cm$
это примерно $5\cdot10^{-6}$.~\cite{BNL:Deuteron2008}

В соответствии с уравнением Т-БМТ,
\[
\W_{EDM,x} = \eta\frac{qE_x}{2mc},
\]
где $\eta$ есть коэффициент пропорциональности между ЭДМ и спином,
равный $10^{-15}$ для дейтрона, для данного уровня чувствительности.~\cite[стр.~206]{Eremey:Thesis}

Для дейтронного BNL FS кольца, $E_x = 12$
МВ/м,~\cite[стр.~19]{BNL:Deuteron2008} так что $\W_{EDM,x}\approx
10^{-9}$ рад/сек. Таким образом получаем, что для того, чтобы достичь
детектируемый уровень отклонения вектора поляризации на 1 мкрад требуется SCT порядка 1000 секунд.~\cite[стр.~207]{Eremey:Thesis}
\subsection{Происхождение спин-декогеренции}\label{sec:decoh:origin}
Декогеренция спина в пучке вызвана разницей угловых скоростей
прецессии спино-векторов частиц, которая, в свою очередь, вызвана разницей
их длин орбит и импульсов. Влияние длины орбиты на спин-тюн частицы описывается 
понятием эффективного Лоренц-фактора, введённым в разделе~\ref{chpt1:FS-methods:effective-Lorentz-factor}.

Из уравнений~\eqref{eq:spin_tune_vs_gamma} для спин-тюна частицы в электромагнитном поле следует, 
что спин-тюны двух частиц с одинаковыми эффективными Лоренц-факторами равны, независимо от их траекторий в ускорителе. Этот принцип используется при подавлении спиновой декогеренции секступольными полями, а также при смене полярности ведущего магнитного поля кольца.

\subsection{Теория секступольного подавления спин-декогеренции}\label{sec:sextupole_spin_dec_solution}
Чтобы минимизировать декогеренцию спина, связанную с бетатронным
движением и отклонением импульса, могут быть использованы
секступольные (или октупольные) поля~\cite[стр.~212]{Eremey:Thesis}

Секступоль силы
\[
S_{sext} = \frac{1}{B\rho} \pddx{B_y}[x][2],
\]
где $B\rho$ магнитная жёсткость, влияет на коэффициент сжатия орбиты
первого порядка как~\cite[стр.~2581]{Senichev:IPAC13}
\begin{align}
	\Delta \alpha_{1,sext} &= -\frac{S_{sext}D_0^3}{L}, \label{eq:Sext_compaction_effect}
	\intertext{и одновременно на длину орбиты как}
	\bkt{\frac{\Delta L}{L}}_{sext} &= \mp \frac{S_{sext}D_0\beta_{x,y}\varepsilon_{x,y}}{L}, \label{eq:Sext_OL_effect}
\end{align}
где $D(s,\delta) = D_0(s) + D_1(s)\delta$ обозначает функцию дисперсии.

Принцип действия секступольного подавления декогеренции можно сформулировать следующим образом. Частица в ускорителе совершает бетатронные колебания вокруг замкнутой орбиты. Из-за дисперсии, замкнутая орбита отличается для разных частиц пучка. Секступоль работает как призма, фокусируя (либо дефокусируя) замкнутые орбиты различных частиц.

В следующих разделах мы будем называть декогеренцию, связанную с
горизонтальными/вертикальными бетатронными, и синхротронными
колебаниями соответственно X-/Y-, и D-декогеренцией. Семейства секступолей, 
подавляющие X-, Y-, и D-декогеренцию, будем обозначать соответственно GSX, GSY, GSD.

Из уравнений~\eqref{eq:Sext_compaction_effect}, и~\eqref{eq:Sext_OL_effect} можно
видеть, что для подавления декогеренции необходимы три семейства
секступолей, помещённых в максимумы функций: $\beta_x$, $\beta_y$ для подавления
X-,Y-декогеренции, и $D_0$ для D-декогеренции.

\subsection{Численное моделирование секступольного подавления спин-декогеренции в идеальном ускорителе}\label{sec:decoh:suppression_in_ideal_lattice}

Мы провели симуляцию для проверки возможности подавления декогеренции секступольными полями. В симуляции 
была использована идеальная структура с замороженным спином, описанная в разделе~\ref{chpt2:lattice:FS_BNL}.
Поскольку элементы в структуре установлены идеально, спин-векторы частиц не поворачиваются вокруг радиальной
оси; прецессия происходит только в горизонтальной слоскости, вокруг вектора $\hat y$. 

Оптимизация производится на энергии пучка 270.00 МэВ, орбитальная и спиновая трансфер-матрицы структуры
вычисляются до пятого порядка разложения ряда Тэйлора.

В структуре используются три семейства секступолей, для подавления, соответственно, X-, Y-, и D-декогеренций.
Оптимальное значение градиента для каждого семейства отыскивается по отдельности; значения полей двух
других семейств при этом зануляются. Решение оптимизировать секступоли отдельно было принято потому, что
одновременная оптимизация всех трёх градиентов вела к численным проблемам в процедуре TSS.\footnote{Также,
мы изучали принципиальную возможность оптимизации всех трёх семейств секступолей, посредством прямого
вычисления необходимых коэффициентов разложения ряда Тэйлора на трёхмерной сетке значений градиентов. 
Вопрос требует более детального рассмотрения, но на данном этапе мы сомневаемся в принципиальной возможности
оптимизации всех трёх семейств секступолей. Возможно по этой причине в~\cite[стр.~219]{Eremey:Thesis} в BNL структуре
используется всего два семейства.}

В процессе оптимизации сначала вычисляются трансфер-матрицы~\footnote{Имеется ввиду орбитальная и
	 спин-трансфер матрицы.} структуры для заданного значения градиента, 
затем процедурой TSS вычисляется разложение Тэйлора спин-тюна 
\begin{equation*}
\nu_s(x,a,y,b,\ell,\delta) = a_0 + a_{1,x}x + a_{2,x}x^2 + a_{1,y}y + \dots,
\end{equation*}
где $a = p_x/p_0$, 
$b=p_y/p_0$, 
$\ell = -(t-t_0)v_0\frac{\gamma}{1+\gamma}$, 
$\delta = (K-K_0)/K_0$; 
$v$ скорость частицы, $K$ её кинетическая энергия, 
а индекс 0 обозначает референсную частицу.~\cite[стр.~9]{COSYINF:Manual:BeamPhys}

В зависимости от оптимизируемого семейства, из этого разложения выбирается коэффициент 
при квадрате соответствующей координаты фазового пространства ($a_{2,x}$, $a_{2,y}$, или $a_{2,\delta}$). 
Модуль коэффициента служит целевой функцией: т.е., 
при оптимальном значении градиента секступолей, спин-тюн не зависит (параболически) от соответствующего отклонения частицы от референсной.

При оптимизации используется алгоритм Simplex.~\cite[стр.~37]{COSYINF:Manual:Programmer}

На Рисунке~\ref{fig:decoh:perfect} изображены зависимости спин-тюна от смещения частицы от референсной по трём
координатам фазового пространства до и после включения оптимизированных секступолей. 
Можно наблюдать, что во всех трёх случаях удалось подавить параболическую зависимость 
спин-тюна от координаты. При этом сохраняется линейная зависимость, 
которая не чувствительна к секступольным полям. 

\begin{figure}[H]
	\centering
	\subbottom[В горизонтальном направлении]{%
		\includegraphics[height=.33\paperheight]{images/decoh_sim/spin_tune_decoh_x_offset}}
\end{figure}
\begin{figure}[H]\centering
	\contsubbottom[В вертикальном направлении]{%
		\includegraphics[height=.33\paperheight]{images/decoh_sim/spin_tune_decoh_y_offset}}
	\contsubbottom[По энергии]{%
		\includegraphics[height=.33\paperheight]{images/decoh_sim/spin_tune_decoh_d_offset_1}}
	\legend{Цветом выделены зависимости при нулевом (чёрный) и оптимизированном (красный) значениях градиента секступоля}
	\caption{Зависимость спин-тюна частицы от её смещения от референсной частицы.\label{fig:decoh:perfect}}
\end{figure}

Линейная зависимость наблюдается при моделировании ускорителя в коде COSY INFINITY, в коде MODE, 
а также при помощи программы MAD. 
Исходя из этого, можно предположить, что эффект не является численным артефактом COSY INFINITY,
а имеет физическое основание. Этот вопрос требует дальнейшего рассмотрения, 
но на данный момент считается, что он подавляется соответствующей подстройкой 
параметров ВЧ-резонатора.~\cite[стр.~210,~219]{Eremey:Thesis}

\subsection{Переход спин-декогеренции из горизонтальной плоскости в вертикальную при увеличении радиальной компоненты частоты МДМ спин-прецессии}\label{sec:Decoherence-plane-transfer}
Целью данной симуляции было показать, что секступольный метод подавления спин-декогеренции
не зависит от плоскости, в которой декогерируют спин-векторы частиц.

В неидеальную структуру с замороженным спином инжектировался ансамбль из 30 частиц, равномерно смещённых 
от референсной в вертикальном направлении в диапазоне $y \in [-1, +1]$ мм. Поскольку анализ производится только
на основании данных трекинга,~\footnote{То есть, отсутствуют численные проблемы процедуры TSS  
	при ``строго-замороженном спине.''} пучок инжектирован на кинетической энергии строго замороженного спина
270.0092 МэВ.

Неидеальности структуры моделируются наклонами E+B элементов вокруг оптической оси на углы, 
взятые из нормального распределения ${\Theta_{tilt} \sim N(0, 1\cdot 10^{-4})}$ радиан. 
Поскольку при введении таких неидеальностий сохраняется величина силы Лоренца, они не искажают 
орбитальную динамику частицы, и отражаются только на спиновой динамике. Величина стандартного отклонения
отражает реалистичную неточность установки элементов ускорителя. 

На Рисунке~\ref{fig:decoh:SX_SD} представлено стандартное отклонение распределения 
радиальных компонент спин-векторов ансамбля
до и после включения секступолей. Поскольку частицы движутся в неидеальной структуре, их спин-векторы вращаются
в вертикальной плоскости с большой скоростью, и потому $\sigma_{s_x}$ --- быстро-осциллирующая функция, не
показывающая долгосрочной тенденции к росту (наклон прямой $(2\pm2)\cdot 10^{-8}$ 1/сек). Таким образом, 
не наблюдается декогеренции спина в горизонтальной плоскости. При использовании секступолей 
амплитуда колебаний $\sigma_{s_x}$ уменьшается на порядок.

\begin{figure}[H]
	\centering
	\subbottom[Без секступолей]{%
		\includegraphics[height=.33\paperheight]{images/decoh_sim/SX_decoh_20sec_unopt}}
	\subbottom[С секступолями]{%
		\includegraphics[height=.33\paperheight]{images/decoh_sim/SX_decoh_20sec_opt}}
	\caption{Стандартное отклонение радиальной компоненты спин-вектора частицы от спин-вектора референсной частицы\label{fig:decoh:SX_SD}}
\end{figure}

На Рисунке~\ref{fig:decoh:SY_SD} представлена та же статистика  для вертикальных компонент спин-векторов. Наблюдается долгосрочный тренд, 
(наклон $(4.5 \pm 0.6)\cdot 10^{-7}$ 1/сек) до включения корректирующих секступолей. Секступольная коррекция не уменьшает амплитуду колебаний, 
но подавляет тренд (наклон после включения секступолей $(5\pm 6)\cdot 10^{-8}$ 1/сек).

\begin{figure}[H]
	\centering
	\subbottom[Без секступолей]{%
		\includegraphics[height=.3\paperheight]{images/decoh_sim/SY_decoh_20sec_unopt}}
		\subbottom[С секступолями]{%
		\includegraphics[height=.3\paperheight]{images/decoh_sim/SY_decoh_20sec_opt}}
	\caption{Стандартное отклонение вертикальной компоненты спин-вектора частицы от спин-вектора референсной частицы\label{fig:decoh:SY_SD}}
\end{figure}
\subsection{Численное моделирование эксперимента по подавлению декогеренции  в неидеальном ускорителе}\label{sec:decoh:sim-imperfect}
При проведении нижеследующих тестов симулировалась инжекция
плоского, гауссовского пучка в структуру с замороженным
спином. E+B спин-ротаторы в структуре были установлены со случайно распределённым углом наклона вокруг оптической оси, взятым из распределения $N(0, 5\cdot10^{-4})$ радиан.

Инжектируемые пучки состояли из 30 частиц, распределённых в вертикальной
плоскости ($y$--$z$) как $y\sim N(y_0, 0.1)$ мм; все остальные координаты равны нулю.
Сдвиг $y_0$ варьировался в диапазоне $[-1, +1]$ мм. Начальное
направление спин-векторов всех частиц --- продольное: $\vec S(t=0) = (0,0,1)$.

Также в структуре варьировалось значение градиента $G_Y$ секступоля,
модулирующего Y-декогеренцию. $G_Y$ менялся в
диапазоне ${[G_Y^0 - 5, G_Y^0 + 5]}$~\Tlmsq, где
$G_Y^0=-0.577$~\Tlmsq --- оптимальное значение градиента для заданных 
неидеальностей структуры. Величина $G_Y^0$ была найдена путём минимизации коэффициента разложения $a_2$
ряда Тэйлора $\nu_s(y) \approx a_0 + a_1\cdot y + a_2\cdot y^2 + O(y^3)$.

На каждое значение градиента приходится 10 инжекций.

Для того, чтобы обеспечить устойчивость процедуры TSS COSY INFINITY~\cite{COSYINF:Manual:BeamPhys}, пучок инжектировался на энергии 270 МэВ (строгий FS находится на энергии 270.0092 МэВ), а 
орбитальная и спин-трансфер матрицы  строились до третьего порядка разложения ряда Тэйлора. 

Далее ансамбль начальных значений, представляющий пучок, трекается
через структуру на протяжении $1.2\cdot10^6$ оборотов, что
примерно эквивалентно 1.2 секундам. Каждые 800 оборотов производится
запись необходимых для анализа данных.

Собираемые данные: 
\begin{enumerate*}[(\bfseries i\normalfont)]
	\item результаты вычислений процедуры TSS: спин-тюн $\nu_s$ и компоненты вектора оси инвариантного спина $\bar n$, и
	\item компоненты спина $(S_X, S_Y, S_Z)$, и фазового пространства $(X,A,Y,B,T,D)$.
\end{enumerate*}
Мы также записывали разложения ряда Тэйлора функций $\nu_s$, $\nbar$, орбитальной, и спиновой трансфер матриц
структур для каждого значения $G_Y$.

Из данных по компонентам спина вычисляется вектор поляризации банча:
\begin{equation}\label{eq:polarization_formula}
\vec P = \frac{\sum_i\vec s_i}{|\sum_i\vec s_i|}.
\end{equation}

Вертикальная компонента вектора фитируется функцией 
\[
{f(t; a,f,\phi) = a\cdot \sin(2\pi\cdot f\cdot t + \phi)},
\]
оцениваются все три параметра 
${(\hat a, \hat f, \hat\phi)}$. 

\subsubsection{Эффект секступольных полей на спин-тюн и на ось стабильного спина}\label{par:Sextupole-effect-on-spin-tune-and-nbar}
На Рисунке~\ref{decoh:fig:ST_vs_y0_GSY} представлена зависимость спин-тюна от вертикального смещения частицы от референсной орбиты: $\nu_s(y) \approx a_0 + a_1\cdot y + a_2\cdot y^2 + O(y^3)$. На Рисунке~\ref{decoh:fig:full:ST_vs_y0_GSY} можно наблюдать разгибание ветвей параболы при $G_Y \rightarrow G_Y^0$.
\begin{figure}[H]
	\centering
	\subbottom[Полный диапазон~\label{decoh:fig:full:ST_vs_y0_GSY}]{%
		\includegraphics[height=.35\paperheight]{images/decoh_sim/spin_tune_vs_offset}}
\end{figure}
\begin{figure}[H]\centering
	\contsubbottom[Деталировка кривой при оптимальном значении $G_Y$\label{decoh:fig:zoom:ST_vs_y0_GSY}]{%
		\includegraphics[height=.35\paperheight]{images/decoh_sim/spin_tune_vs_offset_zoom}
	}
	\legend{Цветом обозначены данные для различных значений градиента $G_Y$ Y-секступоля.}
	\caption{Спин-тюн $\nu_s$ в зависимости от смещения частицы от референсной орбиты.\label{decoh:fig:ST_vs_y0_GSY}}
\end{figure}

Аналогичная зависимость для вертикальной компоненты оси стабильного спина частицы представлена на Рисунке~\ref{decoh:fig:ny_vs_y0_GSY}. На Рисунке~\ref{decoh:fig:full:ny_vs_y0_GSY} мы обнаруживаем, что компонента оси стабильного спина ведёт себя аналогично спин-тюну при $G_Y \rightarrow G_Y^0$. Как и в случае идеальной структуры, на Рисунке~\ref{decoh:fig:zoom:ny_vs_y0_GSY} наблюдается присутствие в разложении $\nbar_y(y)$ линейного члена, не чувствительного к секступольным полям.

\begin{figure}[H]
	\centering
	\subbottom[Полный диапазон.\label{decoh:fig:full:ny_vs_y0_GSY}]{%
		\includegraphics[height=.35\paperheight]{images/decoh_sim/ny_vs_offset}}
	\subbottom[Деталировка кривой при оптимальном значении $G_Y$.\label{decoh:fig:zoom:ny_vs_y0_GSY}]{%
		\includegraphics[height=.35\paperheight]{images/decoh_sim/ny_vs_offset_zoom}}
	\legend{Цветом обозначены данные для различных значений градиента $G_Y$ Y-секступоля.}
	\caption{Вертикальная компонента $\bar n_y$ оси прецессии спина в зависимости от смещения частицы от референсной орбиты.\label{decoh:fig:ny_vs_y0_GSY}}
\end{figure}

На рисунках выше, значения спин-тюна и компонент оси стабильного спина были вычислены как функции только одной переменной; все остальные координаты фазового пространства приняты равными нулю. Анализируя трекинговые данные мы обнаружили, что компоненты оси стабильного спина (как впрочем и спин-тюн) частицы практически не варьируются, как можно было бы ожидать исходя из рисунков, а находятся на практически постоянном уровне. Мы предположили, что зависимости $\nu_s$ и $\nbar$ от вертикальной координаты и от её производной ($y'\equiv B$) компенсируют друг друга во время движения частицы по реальной тракетории. На следующих рисунках мы изобразили значения функций $\nu_s$, $\nbar$ на истинных траекториях частиц в ускорителе.

На Рисунке~\ref{decoh:fig:yb_traj} изображены траектории частиц в плоскости $(Y,B)$ фазового пространства, полученные в результате трекинга через неидеальный ускоритель.
\begin{figure}[H]
	\centering
	\includegraphics[height=.35\paperheight]{images/decoh_sim/YB-PHASE_SPACE_IMPERFECT_UNOPT}
	\caption{Траектории частиц в плоскости $(Y,B)$ фазового пространства.\label{decoh:fig:yb_traj}} 
\end{figure}

На Рисунках~\ref{decoh:fig:ST_on_traj}, \ref{decoh:fig:NX_on_traj}, \ref{decoh:fig:NY_on_traj}, и \ref{decoh:fig:NZ_on_traj} изображены, соответственно: спин-тюн, радиальная, вертикальная, и продольная компоненты оси прецессии спина, вычисленные на траекториях частиц из Рисунка~\ref{decoh:fig:yb_traj}, в двух случаях:
\begin{enumerate*}[(\bfseries i\normalfont)]
	\item с выключенными, и 
	\item с включенными секступолями $G_Y$.
\end{enumerate*}  

\begin{figure}[H]
	\centering
	\subbottom[С выключенными секступолями.]{%
		\includegraphics[height=.35\paperheight]{images/decoh_sim/ST_VS_YB_IMPERFECT_UNOPT}}
	\hfill
	\subbottom[С включенными секступолями.]{%
		\includegraphics[height=.35\paperheight]{images/decoh_sim/ST_VS_YB_IMPERFECT_OPTIM}}
	\hfill
	\caption{Спин-тюны частиц на их траекториях в неидеальной FS-труктуре.\label{decoh:fig:ST_on_traj}}
\end{figure}

\begin{figure}[H]
	\centering
	\subbottom[С выключенными секступолями.]{%
		\includegraphics[height=.35\paperheight]{images/decoh_sim/NX_VS_YB_IMPERFECT_UNOPT}}
	\hfill
	\subbottom[С включенными секступолями.]{%
		\includegraphics[height=.35\paperheight]{images/decoh_sim/NX_VS_YB_IMPERFECT_OPTIM}}
	\hfill
	\caption{Радиальные компоненты осей прецессии спинов частиц на их траекториях в неидеальной FS-труктуре.\label{decoh:fig:NX_on_traj}}
\end{figure}

\begin{figure}[H]
	\centering
	\subbottom[С выключенными секступолями.]{%
		\includegraphics[height=.35\paperheight]{images/decoh_sim/NY_VS_YB_IMPERFECT_UNOPT}}
	\hfill
	\subbottom[С включенными секступолями.]{%
		\includegraphics[height=.35\paperheight]{images/decoh_sim/NY_VS_YB_IMPERFECT_OPTIM}}
	\hfill
	\caption{Вертикальные компоненты осей прецессии спинов частиц на их траекториях в неидеальной FS-труктуре.\label{decoh:fig:NY_on_traj}}
\end{figure}

\begin{figure}[H]
	\centering
	\subbottom[С выключенными секступолями.]{%
		\includegraphics[height=.35\paperheight]{images/decoh_sim/NZ_VS_YB_IMPERFECT_UNOPT}}
	\hfill
	\subbottom[С включенными секступолями.]{%
		\includegraphics[height=.35\paperheight]{images/decoh_sim/NZ_VS_YB_IMPERFECT_OPTIM}}
	\hfill
	\caption{Продольные компоненты осей прецессии спинов частиц на их траекториях в неидеальной FS-труктуре.\label{decoh:fig:NZ_on_traj}}
\end{figure}

Исходя из рисунков можно отметить следующее:
\begin{enumerate}
	\item в безсекступольном случае, и $\nu_s$ и направление $\nbar$ частицы по большей части (с точностью до влияния линейного члена разложения) фиксированы величиной её поперечного эмиттанса;
	\item при применении секступольных полей, уровни $\nu_s$ и $\nbar$ различных частиц сравниваются, 
	и становится виден эффект бетатронных колебаний, связанный с присутствием 
	в разложении Тэйлора функций линейной компоненты.
\end{enumerate}
Таким образом, Рисунки~\ref{decoh:fig:NX_on_traj} и~\ref{decoh:fig:NY_on_traj} свидетельствуют о том, 
что применение секступольных полей не только выравнивает \textbf{модули} частот прецессии частиц банча, 
но и их \textbf{направления}. Продольная компонента оси стабильного спина не чувствительна 
к секступольным полям, как видно из Рисунка~\ref{decoh:fig:NZ_on_traj}.

На Рисунке~\ref{decoh:fig:nbar_vs_ST} представлены зависимости средних значений (уровней) радиальной и вертикальной компонент оси стабильного спина частицы от среднего значения её спин-тюна. На основании этого рисунка, в разделе~\ref{sec:spin_stune_traj_equ:B_form} мы сделали вывод о полной эквивалентности спиновых динамик частиц с одинаковыми эквивалентными Лоренц-факторами.~\footnote{По крайней мере при работе ускорителя в режиме замороженного спина.}
\begin{figure}[H]
	\centering
	\includegraphics[height=.35\paperheight]{images/decoh_sim/mean_n_bar_vs_spin_tune}
	\caption{Средние уровни поперечных компонент осей стабильного спина частиц, в зависимости от уровня их спин-тюна.\label{decoh:fig:nbar_vs_ST}}
\end{figure}

\subsection{Анализ механизма подавления декогеренции секступольными полями}\label{sec:sext_decoh_suppression_effect_analysis}
Исходя из уравнений~\eqref{eq:spin_tune_vs_gamma} и~\eqref{eq:EquLevMom_shift}, зависмость спин-тюна от равновесной энергии частицы можно выразить следующим образом:
\[
\nu_s = G\gamma_0 + G\frac{\gamma_0^2-1}{\gamma_0}\cdot C_0\cdot \Big[f_1(\epsilon_x, \epsilon_y, Q_x, Q_y) +  f_2(\alpha_1, \avg{\Delta K/K}^2)\Big],
\]
где $C_0$ константа, а $f_1$ и $f_2$ определяются уравнением~\eqref{eq:EquLevMom_shift}.

Поскольку частица, совершающая бетатронные колебания, автоматически совершает и 
синхротронные колебания, эффект секступольных полей на её спин-тюн --- это суперпозиция эффектов. 
Однако же частица, находящаяся на референсной орбите, но имеющая начальное отклонение по импульсу, 
имеет такую же длину орбиты, и совершает только синхротронные колебания. 
Соответственно, секступольные поля влияют на спин-тюн такой частицы только 
путём модификации коэффициента сжатия орбиты, т.е. функции $f_2$. 

В связи с этим, мы провели симуляцию, в которой последовательно инжектировали два пучка по 30 частиц: 
в первом (обозначим его как D-банч) частицы были распределены как 
${\delta=\sfrac{\Delta K}{K}\sim N(0, 0.5\cdot 10^{-6})}$, во втором (Y-банче) --- ${y\sim N(0, 0.5)}$ мм. 
Остальные координаты фазового пространства имели начальное значение равное нулю. 

Пучки инжектировались в идеальную структуру. Это сделано для того, чтобы исключить эффекты, 
связанные с возмущением орбит нереференсных частиц. Для первого пучка были включены 
только GSD секспутоли, для второго --- GSY. Градиенты секступолей варьировались 
{$\pm 5$~Тл/м$^2$} от оптимального значения соответствующего семейства.

Трекинг проводился на протяжении $1.2\cdot 10^6$ оборотов, данные выводились каждые 800 оборотов.

На Рисунке~\ref{fig:long_PS_sext_settings} представлены фазовые траектории частиц пучков 
в продольном фазовом пространстве. Мы видим, что фазовые эллипсы частиц D-банча 
центрированы практически~\footnote{При увеличении можно наблюдать различие центров фазовых эллипсов, 
	но это различие не чувствительно к значению градиента секступоля, 
	и скорее всего следствие конечности статистики.} на одной и той же точке, а их эмиттансы 
не меняются при изменении силы поля сексуполя. 

\begin{figure}[H]
	\centering
	\subbottom[Фазовые эллипсы D-банча.]{%
		\includegraphics[height=.3\paperheight]{images/decoh_sim/propdef/long_phase_space_for_sext_settings_D}
	}
	\subbottom[Фазовые эллипсы Y-банча.]{%
		\includegraphics[height=.3\paperheight]{images/decoh_sim/propdef/long_phase_space_for_sext_settings_Y}
	}
	\legend{Цветом различаются траектории частиц с разным начальным вертикальным смещением 
		от замкнутой орбиты.}
	\caption[Продольное фазовое пространство пучка. 
	Звёздочками отмечены центры эллипсов]{Продольное фазовое пространство пучка. 
		Звёздочками отмечены центры эллипсов.\label{fig:long_PS_sext_settings}}
\end{figure}

В то же время, фазовые портреты частиц Y-банча меняются при изменении значения градиента секступоля. 
При этом, мы видим, что максимальная скученность их центров (следовательно равновесных уровней энергии) 
не соответствует оптимальному значению градиента секступоля (фазовый портрет для последнего изображён 
на центральной панели). Именно это наблюдение послужило для нас стимулом инжекции в структуру D-банча. 
Мы объясняем это наблюдение суперпозицией эффектов сжатия равновесных орбит частиц, 
и модификации коэффициента сжатия орбиты.

Для более тщательного рассмотрения эффектов секступоля на функции $f_1$ и $f_2$, мы построили зависимости средних уровней спин-тюнов частиц от их равновесных уровней энергии (Рисунок~\ref{fig:ST_vs_dkok_for_sext_strenghts}). Мы видим, что скученность точек графика для D-банча не меняется при варьировании значения градиента секступоля, а меняется только функциональная зависимость спин-тюна от равновесной энергии, как и предполагается функциональной формой $f_2$ (см. раздел~\ref{chpt1:FS-methods:effective-Lorentz-factor}). Таким образом, сигнатурой оптимизации коэффициента сжатия орбиты служит изменение функциональной зависимости $\avg{\nu_s} = f(\avg{\Delta K/K})$.

\begin{figure}[H]
	\centering
	\subbottom[Для D-банча.]{%
		\includegraphics[height=.3\paperheight]{images/decoh_sim/propdef/stune_vs_dkok_SS_D}
	}
\end{figure}
\begin{figure}[H]\centering
	\contsubbottom[Для Y-банча.]{%
		\includegraphics[height=.3\paperheight]{images/decoh_sim/propdef/stune_vs_dkok_SS_Y}
	}
	\caption{Зависимость среднего уровня спин-тюна частицы от её равновесного уровня энергии для различных значений градиента секступоля.\label{fig:ST_vs_dkok_for_sext_strenghts}}
\end{figure}

На рисунке для Y-банча наблюдаются оба эффекта секступольного подавления декогеренции: изменяется и скученность точек (т.е. эмиттанс пучка), и функциональная зависимость.

\paragraph{Вывод:} Симуляция подтверждает утверждения~\eqref{eq:Sext_compaction_effect} и~\eqref{eq:Sext_OL_effect}.