
{\actuality} 
Данная работа посвящена исследованию особенностей измерения электрического дипольного момента (ЭДМ) 
элементарной частицы в накопительном синхротроне.

Одной из альтернатив Стандартной Модели (СМ) элементарных частиц являются теории суперсимметрии 
(так называемые  SUSY-теории). Электрический дипольный момент (ЭДМ) элементарных частиц может служить
отличным инструментом для подтверждения одной из этих моделей; к примеру: в случае нейтрона, 
электрический дипольный момент, совместный со стандартной моделью, 
находится в диапазоне от $10^{-33}$ до $10^{-30}$~\ecm,~\cite{Harris:Neutron2007} 
в то время как SUSY-теории предсказывают наличие ЭДМ гораздо большей величины -- 
на уровне ${10^{-29} - 10^{-24}}$~\ecm. 

Поиск ЭДМ частиц был начат более 50-ти лет назад. Первый эксперимент по измерению ЭДМ нейтрона был проведён д-р. Н.Ф. Рэмзи (Dr. N.F. Ramsey) в конце 1950-х годов. По результатам эксперимента верхняя граница ЭДМ нейтрона была ограничена величиной ${5\cdot 10^{-20}}$~\ecm.~\cite{Ramsey:Neutron1957} 
С тех пор было проведено множество более точных экспериментов 
и на данный момент верхняя граница на ЭДМ нейтрона находится на уровне 
${2.9\cdot10^{-26}}$~\ecm.~\cite{Baker:nEDM:Main, Baker:nEDM:Reply}

Большинство экспериментов проводятся на зарядово-нейтральных частицах, таких как нейтрон или атомы. 
ЭДМ заряженных частиц, таких как протон или дейтрон, можно измерить в накопительном кольце 
на основе прецессии поляризации пучка в электрическом поле в системе центра масс пучка. 
Преимущество использования накопительного кольца как инструмента измерения ЭДМ 
обусловлено такими свойствами ускорителя, как:
\begin{enumerate*}[(1)]
	\item чистота среды;
	\item высокая интенсивность и поляризованность, а также
	\item длительность времени жизни циркулирующего пучка.~\cite[стр.~9]{BNL:Deuteron2008}
\end{enumerate*}

Идея использования накопительного кольца для детектирования ЭДМ заряженных частиц 
появилась в процессе разработки $g-2$ эксперимента~\cite{BNL:g-2:2001} 
в Брукхейвенской Национальной Лаборатории (BNL, США). По результатам экспериментов в BNL 
верхняя граница электрического дипольного момента мюона была определена 
на уровне $10^{-19}$~\ecm.~\cite{BNL:muon_ANA:2009} В 1990-х годах дискуссия преимущественно велась 
вокруг мюонного эксперимента~\cite{Farley:SREDM:Muon}, однако также рассматривался и дейтрон, 
имеющий похожее отношение аномального магнитного момента к массе.

В 2004 году коллаборацией srEDM (Storage Ring EDM Collaboration~\cite{BNL:SREDM}) 
в Брукхейвенской Национальной Лаборатории (BNL, США) был предложен эксперимент 970 
по детектированию ЭДМ дейтрона на уровне $10^{-27}~e\cdot$см в накопительном кольце. 
Тогда же была предложена идея ``замороженного спина,''~\cite{Farley:SREDM:Muon} 
в котором направления векторов спина и импульса (референсной) частицы совпадают в каждый момент времени. Это условие должно обеспечивать максимальный рост сигнала электрического дипольного момента при его наличии. Теретически, условие ``замороженного спина'' означает нулевой спиновый резонанс, при котором ориентация спин-вектора остается  пространственно-неизменной при отсутствии электрического дипольного момента. 
Тогда любой рост вертикальной компоненты поляризациии пучка детектирует наличие электрического дипольного момента. Таким образом, измеряя  амплитуду вертикальной компоненты спина, можно определить величину электрического дипольного момента. Реализация этой концепции потребует создание специального накопительного кольца и определенных параметров пучка.

Однако, в последствии выяснилось, что ``замороженный спин'' -- лишь одно из условий успешного детектирования электрического дипольного момента. В частности, для измерения ЭДМ с требуемой точностью необходимо накопление большой статистики, которое возможно при сохранении поляризации, то есть максимальной однонаправленности спина всех частиц в пучке, в течение достаточно длительного времени, порядка 1~000~секунд. 

Другим важным условием является требование исключения примешивания к сигналу ЭДМ 
сигнала магнитного дипольного момента (МДМ), возникающего из-за различного рода несовершенств 
элементов кольца и соизмеримого по величине с ЭДМ. 
Классический метод ``замороженного спина'' (когда спин-прецессия, связанная с МДМ, исключена полностью) проблематичен в этом отношении по двум причинам: 
во-первых, при приближении к состоянию ``замороженности'' спина малейшие возмущения со стороны магнитного и электрического полей приводят к нарушению ориентации оси стабильного спина, что сразу же вносит неопределенность в измерение вертикальной компоненты спина; 
во-вторых, для уменьшения скорости МДМ спин-прецессии вокруг радиальной оси до уровня, позволяющего измерить ЭДМ с точностью $10^{-29}$~\ecm, необходимо устанавливать оптические элементы ускорителя с точностью, значительно превышающей технологические возможности современных геодезических методов.

Начиная с 2005 года на циклотроне AGOR KVI-центра передовых радиационных технологий 
(KVI-Center for Advanced Radiation Technology) в университете Гронингена была проведена серия тестов 
по технико-экономическому обоснованию эксперимента.

В 2008 году начались исследования на накопительном кольце COSY в Исследовательском центре ``Юлих'' (Forschungszentrum J\"ulich GmbH, Германия). 
В период с 2015 по 2019 автор принимал непосредственное участие в этих работах. 
Исследования велись по трем направлениям:
\begin{enumerate}[(1)]
	\item Экспериментальное изучение декогеренции спина частиц в пучке в процессе циркуляции в кольце. 
	Поскольку кольцо COSY не отвечает требованиям реализации условия ``замороженного спина,'' 
	спин-декогеренция изучалась по времени исчезновения средней по пучку ассиметрии 
	сечения взаимодействия в реакции рассеяния дейтронного пучка на углеродной мишени~\cite{COSY:SCT:IPAC15}.
	Вектор поляризации пучка при этом быстро прецессировал в плоскости замкнутой орбиты, 
	что, однако, не влияет на сделанные выводы. 
	
	\item Экспериментальное детектирование сигнала электрического дипольного момента с помощью возбуждения параметрического резонанса прецессии спина. Сила резонанса при этом пропорциональна величине детектируемого ЭДМ. Резонансный метод не требует условия ``замороженного спина,'' но его чувствительность на четыре-пять порядков ниже; в лучшем случае, его достижимый предел измерения ЭДМ находится на уровне $10^{-24}$~\ecm. 
	
	\item Разработка метода измерения ЭДМ и его полномасштабное моделирование с целью его совершенствования, а также  разработки новых подходов к измерению электрического дипольного момента заряженной частицы с использованием накопительного кольца. Автор настоящей работы был вовлечён в это направление исследований.
\end{enumerate}

Впоследствии, эти тесты развились в программу по изучению спин-орбитальной динамики пучка 
для разработки технологий, требуемых для эксперимента по поиску ЭДМ. 
В этом же году было сделано второе предложение~\cite{BNL:Deuteron2008} эксперимента 
по поиску ЭДМ дейтрона на основе концепции ``замороженного спина''; 
в этот раз --- на уровне $10^{-29}$~\ecm~ при условии накопления результатов измерения в течение года.

В то же время было решено, что эксперимент по детектированию ЭДМ протона, 
поскольку его можно измерить в полностью электростатическом кольце, 
обладает некоторыми техническими достоинствами. Среди таковых предполагается возможность 
одновременной инжекции противоположно-циркулирующих пучков, что позволяет уменьшить 
систематические ошибки измерения ЭДМ протона, вызванные несовершенством элементов накопительного кольца.
Тем не менее, на COSY была продолжена работа над экспериментом с дейтроном, ввиду того,
что результаты, полученные для дейтрона, распространяются и на протон.

В 2011 году была сформирована коллаборация JEDI (J\"ulich Elecric Dipoe moment Investigations).~\cite{JEDI:Website} 
Целью коллаборации является не только разработка ключевых технологий для srEDM, но также и проведение предварительного эксперимента прямого наблюдения ЭДМ дейтрона. 

В 2018 году JEDI-коллаборация выполнила первое измерение дейтронного ЭДМ на COSY на основе резонансного метода~\cite{COSY:Partially-Frozen-Spin, COSY:SpinTuneMapping} 
с использованием специально разработанного для этой цели RF Wien filter~\cite{JSlim:RFWF:Design, JSlim:RFWF:Commisioning}. 
В кольце с незамороженным спином 
ЭДМ генерирует осцилляции вертикакльной компоненты поляризации пучка с малой амплитудой; например, 
при импульсе дейтронов 970 МэВ/с, как на COSY, амплитуда колебаний ожидается на уровне 
$3\cdot10^{-10}$ при величине ЭДМ $d = 10^{-24}$~\ecm. В связи с малостью амплитуды колебаний, установленный в данном эксперименте предел измерения ЭДМ оценивается на уровне $d=10^{-24}$~\ecm.