
\begin{figure}[h]
	\centering
	\includegraphics[scale=.5]{images/chapter4/800px-COSY_Ring}
	\caption{Синхротрон COSY.\label{fig:COSY_Ring}}
\end{figure}

COSY -- это изохронный синхротрон типа racetrack с возможностью охлаждения (рисунок~\ref{fig:COSY_Ring}), находящийся в Исследовательском центре Юлих, Германия.~\cite{COSY-Ring}

Для инжекции ионов $H^-$ и $D^-$ в COSY  используется циклотрон JULIC, предоставляющий 8 мкА неполяризованных (или 1 мкА поляризованных) ионов $H^-$ с импульсом 45 МэВ/с. При инжекции отрицательные ионы проходят через углеродный стриппер для изменения их заряда на положительный. Экстракция пучка из циклотрона производится с помощью септум-дефлектора.~\cite{JULIC-Injector}
%
Возможности направляющей магнитной системы синхротрона ограничивают импульс пучка в диапазоне до 3700 МэВ/с. Экстракция из кольца осуществляется при помощи медленной экстракции (экстракция с помощью кикера).

На COSY доступны два типа охлаждения пучка: электронное (диапазон энергий электронов в ``старом'' и ``новом'' кулере -- 20-100 кэВ и 20 кэВ - 2 МэВ, соответственно) и стохастическое. 
%
Два электронных кулера, установленые в кольце на прямой секции, обеспечивают электронное охлаждение во всём диапазоне энергий кольца. Система стохастического охлаждения обеспечивает охлаждение пучка при импульсах 1.5-3.7 ГэВ/с.