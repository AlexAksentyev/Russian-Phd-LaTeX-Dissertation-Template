% здесь я пытаюсь подтянуть свои статьи по ТРИКу

Отметим, что целью экспериментов по поиску ЭДМ является проверка CP-инвариантности. При этом, ЭДМ элементарных частиц нарушают одновременно и P-, и T-симметрию, а следовательно требуют дополнительных модельных предположений, для того, чтобы связать их существование с CP-нарушением.~\cite[стр.~1926]{Aksentev:TRIC}

Альтернативой является эксперимент TRIC (Time Reversal Invariance at Cosy),~\cite{Aksentev:TRIC} в котором используется T-нечётное, P-чётное взаимодействие, и следовательно нарушается только T-симметрия. В связи с этим, никаких дополнительных предположений не требуется.

TRIC входит в физическую программу PAX (Polarised Antiproton eXperiments)~\cite{Aksentev:PAX}, для которой требуются высокоинтенсивные поляризованные пучки. Существует два подхода к получению поляризованных пучков: спин-флиппинг и спин-филтеринг. Спин-флиппинг позволяет получать более интенсивные пучки, однако на данный момент не существует стабильно-работающих методов спин-флиппинга.

Рассмотренные в настоящей работе особенности спиновой динамики вблизи состояния ``замороженного'' спина (в частности --- подавление спин-декогеренции секступольными полями) представляют большой интерес с точки зрения сохранения поляризации в окрестности нулевого спинового резонанса и могут быть использованы при планировании экспериментов на ускорительном комплексе NICA (Дубна).