
\begin{figure}[h]
	\centering
	\includegraphics[scale=.5]{images/chapter4/800px-COSY_Ring}
	\caption{Синхротрон COSY.\label{fig:COSY_Ring}}
\end{figure}

Ускорительный комплекс COSY~\cite{COSY-Ring} --- это синхротрон длиной 183 метра, позволяющий проводить эксперименты с поляризованными и охлаждёнными пучками протонов и дейтронов в диапазоне энергии от 45 МэВ до 3.7 ГэВ. Схема ускорителя приведена на Рисунке~\ref{fig:COSY_Ring}.

Для инжекции ионов $H^-$ и $D^-$ в COSY  используется циклотрон JULIC (J\"ulich Light Ion Cyclotron), предоставляющий 8 мкА неполяризованных (или 1 мкА поляризованных) ионов $H^-$ с импульсом 45 МэВ/с. При инжекции отрицательные ионы проходят через углеродный стриппер для изменения их заряда на положительный. Экстракция пучка из циклотрона производится с помощью септум-дефлектора.~\cite{JULIC-Injector}

Возможности направляющей магнитной системы синхротрона ограничивают импульс пучка в диапазоне до 3700 МэВ/с. Экстракция из кольца осуществляется с помощью кикера.

На COSY доступны два типа охлаждения пучка: электронное (диапазон энергий электронов в ``старом'' и ``новом'' кулере: 20--100 кэВ и 20--2,000 кэВ, соответственно) и стохастическое. 
%
Два электронных кулера, установленые в кольце на прямой секции, обеспечивают электронное охлаждение во всём диапазоне энергий кольца. Система стохастического охлаждения обеспечивает охлаждение пучка при импульсах 1.5-3.7 ГэВ/с.

Поляризация пучка непрерывно отслеживается на поляриметре  EDDA. Также недавно был установлен поляриметр на основе детекторов WASA, а в конце 2019 года будет установлен новый поляриметр на основе LYSO-сцинцилляторов.
Поляризация протонов достигает 75\% вплоть до наивысших значений импульса; векторная и тензорная поляризации дейтронного пучка достигают 60\%.