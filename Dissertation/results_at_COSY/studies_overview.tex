

\section{Высокоточное измерение спин-тюна}
Пучок дейтронов с вертикально-ориентированным вектором поляризации инжектировался в ускоритель. После подготовительной фазы, во время которой он охлаждался и банчировался, поляризация пучка разворачивалась в горизонтальную плоскость, при помощи ВЧ-соленоида, создающего спиновый резонанс.~\cite[p.~7]{COSY:SpinTuneMapping}

Далее, пучок непрерывно экстрагировался на углеродную мишень, и измерялась асимметрия частоты событий на верхней и нижней секциях детектора, пропорциональная горизонтальной поляризации пучка. Благодаря использованию специально разработанной для этого системы сбора данных,~\cite{COSY:DAQ} было возможно точно определить количество оборотов, сделанных пучком к времени наблюдения события на детекторе.

Проблема измерения заключается в невозможности вычислить спин-тюн путём простого фитирования данных поляриметрии, с $\nu_s$ как оцениваемым параметром, поскольку частота прецессии спина происходит с частотой приблизительно 120 кГц, в то время как частота детектирования событий не превосходит 5 кГц, в связи с чем наблюдалось только одно событие за 24 оборота поляризации вокруг вертикальной оси. Для решения проблемы разреженности данных, был применён алгоритм отображения измерений на период одной осцилляции.~\cite{Eversmann:SpinTuneMeasurement}

В результате была получена беспрецедентная точность определения спин-тюна на уровне $10^{-10}$ в измерительном цикле цикле длительностью 100 секунд, что теоретически позволяет определить величину ЭДМ на уровне $10^{-24}$\ecm.

\section{Юстировка квадруполей при помощи пучка}
\newcommand{\Nbpm}{N_{\mathrm{BPM}}}
Для юстировки положения квадруполя (Beam Based Alignment~\cite{Wagner:BBA2018}), варьируют силу поля квадруполя, и наблюдают за реакцией пучка. Если пучок проходит не через центр квадруполя, он отклоняется. Величина отклонения описывается выражением
\[
\Delta x = \frac{\Delta k\cdot x(s_0)\ell}{B\rho}\cdot \frac{1}{1 - k\frac{\ell\beta(s_0)}{2B\rho\tan\pi\nu}}\cdot \frac{\sqrt{\beta(s)\beta(s_0)}}{2\sin\pi\nu}\cos\bkt{\phi(s) - \phi(s_0) - \pi\nu},
\]
где $\Delta x$ изменение орбиты; $s$ координата, в которой измеряется отклонение пучка; $s_0$ точка расположения квадруполя; $\Delta k$ изменение силы квадруполя; $\ell$ длина квадруполя; $\nu$ бетатронный тюн; $\phi$ бетатронная фаза; $x(s_0)$ положение пучка относительно магнитного центра квадруполя.

Поскольку изменение орбиты $\Delta x(s)$ --- линейная функция отклонения пучка от магнитного центра квадруполя, возможно определить оптимальное положение квадруполя, минимизируя функцию
\[
f = \frac{1}{\Nbpm}\sum_{i=1}^{\Nbpm} (x_i(+\Delta k) -x_i(-\Delta k)^2 \propto x^2(s_0).
\]

Впервые, проверка технологии BBA была проведена в ноябре-декабре 2017 года. Методология требует варьирования силы одного квадруполя за раз, иначе наблюдаемый эффект отклонения пучка будет суперпозицией нескольких отклонений. Поскольку квадруполи на COSY питаются группами по четыре, для варьирования силы поля единичного квадруполя были использованы дополнительные обмотки полюсов некоторых квадруполей. В этом случае, поле квадруполя становится суперпозицией двух квадрупольных полей, но это не отражается на работе методики. 

Для варьирования точки входа пучка в квадруполь использовались кикеры, отклоняющие пучок от референсной орбиты.

Повторная отработка методологии была проведена в феврале 2019 года.

По результатам работы, положения квадруполей были определены с точностью 0.2 мм.~\cite[стр.~182]{YellowReport}