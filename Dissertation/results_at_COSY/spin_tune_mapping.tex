\newcommand{\cbar}{\vec c}
\newcommand{\wbar}{\vec w}

Хотя это напрямую не связано с детектированием ЭДМ частицы методом замороженного спина,
измерение частоты прецессии поляризации пучка вокруг направления ведущего поля ускорителя
является важным шагом к пониманию и контролю спин-прецессии в накопительном кольце.

\subsection{Возможные наблюдаемые}
В идеальном магнитном накопительном кольце, спины частиц пучка прецессируют вокруг ведущего магнитного
поля. Введём правостороннюю систему координат, связанную с центром масс пучка, так что орт $\hat y$ будет
сонаправлен с ведущим полем, $\hat z$ сонаправлен с импульсом референсной частицы. ЭДМ влияет
на спиновую динамику пучка двумя путями. Во-первых, он отклоняет ось стабильного спина референсной частицы
от строго вертикального направления:~\cite{COSY:SpinTuneMapping}
\[
\cbar = \hat x\sin\xi_{EDM} + \hat y\cos\xi_{EDM}, ~~ \tan\xi_{EDM} = \frac{\eta\beta}{G}.
\]

Во-вторых, он изменяет значение спин-тюна, от его канонического $\nu_s = \gamma G$ на
\[
\nu_s = \frac{\gamma G}{\cos\xi_{EDM}}.
\]

На COSY используется резонансный метод измерения спин-тюна, основанный на применении
ВЧ Вин-фильтра (ВЧ ВФ).~\cite{COSY:Partially-Frozen-Spin} ВЧ ВФ с вертикально-направленным магнитным,
радиально-направленным электрическим полем, создаёт ВЧ-модулируемую МДМ прецессию поляризации пучка
вокруг вертикальной оси. При этом, если частота ВЧ поля
фильтра совпадает с частотой прецессии спина без него, модуляция спин-тюна резонирует с ЭДМ взаимодействием
с электрическим полем $\vec E \propto \vec\beta\times\vec B$, и генерирует прецессию спина вокруг радиальной оси.

Сила резонанса
\[
\epsilon = \frac12 \chi_{WF}|\cbar\times\wbar|,
\]
где $\cbar$ обозначает ось стабильного спина кольца в месте расположения ВЧ ВФ \emph{до} его включения,
а $\chi_{WF}$ и $\wbar$ --- угол поворота спина и направление магнитного поля в ВЧ ВФ.

В идеальном ВЧ ВФ
\[
\epsilon = \frac12 \chi_{WF}\sin\xi_{EDM},
\]
равна нулю, если ЭДМ $d$, а значит и $\xi_{EDM}$, равны нулю.~\cite[стр.~5]{COSY:SpinTuneMapping}

Гораздо более выгодным было бы измерение направления самой оси стабильно спина в ускорителе, а не силы резонанса,
поскольку последняя аттенюирована фактором $\chi_{WF}\ll 1$. Проблема заключается в том, что, как и ЭДМ,
неидеальности ускорителя дают вклад в наклон оси от вертикального направления. В связи с этим, задача анализа
неидеальности накопительного кольца стала приоритетной в исследованиях на COSY.

\subsection{Наблюдение наклона оси стабильного спина}
Поскольку спин-тюн является очень точной наблюдаемой, которая к тому же чувствительна к неидеальностям
в ускорителе, она может быть использована для определения направления оси стабильного спина $\cbar$.
Для этого используются два соленоида, которые выступают в качестве искусственных неидеальностей. Обозначив
угол поворота спина в соленоиде $\chi_{AI}$, направление его магнитного поля $\vec k$,


\subsection{Процедура измерения и анализа}
Одной из задач эксперимента, проведённого в 2014 году на COSY было изучение оси стабильного спина, путём
введения искусственных неидеальностей


\subsection{Результаты}
