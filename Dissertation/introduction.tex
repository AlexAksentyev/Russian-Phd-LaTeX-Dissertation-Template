\chapter*{Введение}                         % Заголовок
\addcontentsline{toc}{chapter}{Введение}    % Добавляем его в оглавление

\newcommand{\actuality}{}
\newcommand{\progress}{}
\newcommand{\aim}{{\textbf\aimTXT}}
\newcommand{\tasks}{\textbf{\tasksTXT}}
\newcommand{\novelty}{\textbf{\noveltyTXT}}
\newcommand{\influence}{\textbf{\influenceTXT}}
\newcommand{\methods}{\textbf{\methodsTXT}}
\newcommand{\defpositions}{\textbf{\defpositionsTXT}}
\newcommand{\reliability}{\textbf{\reliabilityTXT}}
\newcommand{\probation}{\textbf{\probationTXT}}
\newcommand{\contribution}{\textbf{\contributionTXT}}
\newcommand{\publications}{\textbf{\publicationsTXT}}


% {\progress} 
% Этот раздел должен быть отдельным структурным элементом по
% ГОСТ, но он, как правило, включается в описание актуальности
% темы. Нужен он отдельным структурынм элемементом или нет ---
% смотрите другие диссертации вашего совета, скорее всего не нужен.

{\aim} данной работы является развитие метода поиска электрического дипольного момента дейтрона с использованием накопительного кольца на основе измерения частоты прецессии спина (frequency domain method) с экспериментально подтвержденной точностью.

Для~достижения поставленной цели необходимо было решить следующие {\tasks}:
\begin{enumerate}
%  \item Исследовать явление декогеренции спина пучка в окрестности нулевой спиновой частоты, а также секступольный метод её подавления. 
%  \item Исследовать влияние возмущений спиновой динамики на ЭДМ-статистику.
%  \item Исследовать влияние неточности установки E+B спин-ротаторов на систематическую ошибку ЭДМ-статистики.
%  \item Промоделировать процесс калибровки спин-тюна пучка при смене полярности ведущего поля.
	\item Разработать метод измерения электрического дипольного момента дейтрона на основе измерения частоты прецессии спина.
	\item Проанализировать требования к магнитооптической структуре кольца-накопителя, ориентированного на поиск электрического дипольного момента дейтрона.
	\item Исследовать явление спин-декогеренции пучка дейтронов в окрестности состояния ``замороженного'' спина. 
	\item Разработать метод подавления декогеренции спина с помощью нелинейных магнитных элементов.
	\item Исследовать влияние различного рода несовершенств элементов кольца на спин-орбитальную динамику. 
	\item Выполнить математическое моделирование процесса калибровки нормализованной частоты прецессии спина (спин-тюн) при помеременной смене полярности ведущего поля.
	\item Проанализировать систематические ошибки в различных предложениях по проведению эксперимента по поиску электрического дипольного момента, и сравнить их с разработанным методом. 
	\item Изучить накопление необходимой статистики измерения электрического дипольного момента.
\end{enumerate}

{\novelty}
\begin{enumerate}
	\item Впервые предложен метод измерения электрического дипольного момента дейтрона,
	 основанный исключительно на измерении частоты прецессии спина в накопительном кольце 
	 (Frequency Domain method) с ограничением по точности, оцениваемым на уровне $10^{-29}$~\ecm.
	\item Изучена спин-орбитальная динамика дейтронного пучка в окрестности состояния ``замороженного спина''
	в накопительном кольце, предназначенном для поиска электрического дипольного момента. 
	\item Предложен метод калибровки средней по пучку нормированной частоты прецессии спина, позволяющий уменьшить вклад систематических ошибок.
	\item Введено определение эффективного значения фактора Лоренца, необходимое для 
	определения зависимости частоты прецессии спина частицы от её координат в фазовом пространстве. 
	\item Сделаны статистические оценки предельной чувствительности измерения ЭДМ предложенным методом в накопительном кольце. 
%	\item Исследована систематическая ошибка эксперимента по поиску ЭДМ в накопительном кольце, связанная с бетатронными колебаниями.
	\item Проведена общая классификация методов поиска ЭДМ в накопительном кольце, систематизированы их общие проблемы.
\end{enumerate}

{\influence}. Результаты исследования вошли в Yellow Report под названием 
``Feasibility Study for an EDM Storage Ring,'' подготавливаемый для CERN коллаборацией CPEDM 
(Charged Particle EDM Collaboration~\cite{CPEDM:Website}), в которую входит коллаборация JEDI.

% здесь я пытаюсь подтянуть свои статьи по ТРИКу

Целью экспериментов по поиску ЭДМ является проверка CP-инвариантности. При этом ЭДМ элементарных частиц нарушают одновременно и P-, и T-симметрию, а следовательно требуют дополнительных модельных предположений, для того, чтобы связать их существование с CP-нарушением.~\cite[стр.~1926]{Aksentev:TRIC}

Альтернативой является эксперимент TRIC (Time Reversal Invariance at Cosy~\cite{Aksentev:TRIC}), 
в котором используется T-нечётное, P-чётное взаимодействие, а значит нарушается \emph{только} T-симметрия, 
в связи с чем никаких дополнительных предположений не требуется.
%
TRIC входит в физическую программу PAX (Polarised Antiproton eXperiments~\cite{Aksentev:PAX}), для которой требуются высокоинтенсивные поляризованные пучки. Существует два подхода к получению поляризованных пучков: спин-флиппинг и спин-филтеринг. Спин-флиппинг позволяет получать более интенсивные пучки, однако на данный момент не существует стабильно-работающих методов спин-флиппинга.

%Рассмотренные в настоящей работе особенности спиновой динамики вблизи состояния ``замороженного'' спина (в частности --- подавление спин-декогеренции секступольными полями) представляют большой интерес с точки зрения сохранения поляризации в окрестности нулевого спинового резонанса и могут быть использованы при планировании экспериментов на ускорительном комплексе NICA (Дубна).

Разработанный метод представляет интерес с точки зрения планирования экспериментов по поиску ЭДМ
 на различных ускорителях, в том числе на ускорительном комплексе NICA ОИЯИ (Дубна).

{\methods} Основными методами исследования являются математическое и компьютерное моделирование,
численный эксперимент.

{\defpositions}
\begin{enumerate}
	\item Метод измерения электрического дипольного момента дейтрона, основанный исключительно на измерении частоты прецессии спина при движении пучка в накопительном синхротроне.
	\item Принцип построения магнитооптической структуры накопительного кольца, ориентированного на поиск электрического дипольного момента дейтрона.
	\item Результаты исследования спин-декогеренции пучка дейтронов в окрестности состояния ``замороженного'' спина и метод её подавления с помощью нелинейных магнитных элементов.
	\item Результаты исследования влияния различного рода несовершенств элементов накопительного кольца 
	на спин-орбитальную динамику пучка. 
	\item Метод калибровки нормализованной частоты прецессии спина (спин-тюн) при попеременной смене полярности ведущего поля и его численная модель.
	\item Результаты исследования систематических ошибок в различных предложениях по проведению эксперимента по поиску электрического дипольного момента и их сравнения с разработанным методом. 
	\item Результаты исследования статистических свойств разработанного метода измерения электрического дипольного момента	в накопительном кольце.
%	\item Подтверждена теория механизма секступольного подавления декогеренции. % section 2.2.7
%  	\item Подтверждено утверждение о равенстве спин-тюнов частиц с одинаковыми эффективными Лоренц-факторами; найдена интерпретация эффективного Лоренц-фактора как меры продольного эмиттанса частицы. % section 2.5.2
%  	\item Показано, что калибровка ведущего магнитного поля ускорителя посредством наблюдения частоты прецессии поляризации пучка в горизонтальной плоскости --- потенциально работающая методика.
%  	\item Доказано, что возмущения спиновой динамики пучка, вызванные бетатронными колебаниями --- пренебрежимо малый систематический эффект, поддающийся контролю в методологии частотной области. % pretty well-founded
%  % statistics
%  	\item Доказано, что эффективная длительность цикла измерения поляризации находится в диапазоне от двух до трёх постоянных времени жизни поляризации. % this is a fairly well-founded claim
%  	\item Показана принципиальная возможность получения верхнего предела оценки ЭДМ на уровне $10^{-29}~e\cdot$см за полное время измерений длительностью один год. % if beam 1e11, sampling rate 375 Hz etc
%  	\item Доказано, что угловая скорость паразитного МДМ вращения линейно зависит от среднего угла наклона спин-ротаторов, и не зависит от конкретной реализации распределения наклонов. % multiple random distributions
%  	\item Доказано, что точность установки оптических элементов ускорителя не позволяет измерять ЭДМ частицы методами пространственной области. % b/c spin precession frequency is on the order of 50 rad/sec
\end{enumerate}

{\reliability} полученных результатов подтверждается согласованием аналитических вычислений с результатами численных экспериментов. Результаты компьютерного моделирования находятся в соответствии с результатами, полученными другими авторами, и результатами, полученными в экспериментах на ускорителе COSY (Исследовательский центр ``Юлих,'' Германия).


{\probation}
Основные результаты работы докладывались~на:
\begin{itemize}
\item IIX международной концеренции по ускорителям заряженных частиц IPAC'17, Копенгаген, Дания.
\item X международной конференции по ускорителям заряженных частиц IPAC'19, Мельбурн, Австралия.
\item конференциях коллаборации JEDI, Юлих, Германия, 2017--2019.
\item III международной конференции ``Лазерные, плазменные исследования и технологии,'' (LaPlas) Москва, Россия. 
\item IV междунарожной конференции LaPlas, Москва, Россия.
\item V международной конференции LaPlas, Москва, Россия.
\item семинарах Института Ядерных Исследований, Исследовательский Центр ``Юлих,'' Германия.
\end{itemize}

{\contribution} Все результаты, выносимые на защиту, получены автором лично. Результаты аналитического и численного исследования спин-орбитальной динамики пучка для моделирования метода измерения электрического дипольного момента дейтрона с помощью измерения прецессии спина в накопительном кольце получены автором лично либо при участии научного руководителя. Вклад соавторов в результаты, полученные совместно, оговаривается в тексте диссертации для каждого случая.
%Автор принимал активное участие в коллаборации JEDI, а также подготовке Yellow Report для CERN.

%\publications\ Основные результаты по теме диссертации изложены в ХХ печатных изданиях~\cite{Sokolov,Gaidaenko,Lermontov,Management},
%Х из которых изданы в журналах, рекомендованных ВАК~\cite{Sokolov,Gaidaenko}, 
%ХХ --- в тезисах докладов~\cite{Lermontov,Management}.

\ifnumequal{\value{bibliosel}}{0}{% Встроенная реализация с загрузкой файла через движок bibtex8
    \publications\ Основные результаты по теме диссертации изложены в 12 печатных изданиях, 
    \hl{X из которых изданы в журналах}, рекомендованных ВАК, 
    7 "--- в тезисах докладов.%
}{% Реализация пакетом biblatex через движок biber
%Сделана отдельная секция, чтобы не отображались в списке цитированных материалов
    \begin{refsection}[vak,papers,conf]% Подсчет и нумерация авторских работ. Засчитываются только те, которые были прописаны внутри \nocite{}.
        %Чтобы сменить порядок разделов в сгрупированном списке литературы необходимо перетасовать следующие три строчки, а также команды в разделе \newcommand*{\insertbiblioauthorgrouped} в файле biblio/biblatex.tex
        \printbibliography[heading=countauthorvak, env=countauthorvak, keyword=biblioauthorvak, section=1]%
        \printbibliography[heading=countauthornotvak, env=countauthornotvak, keyword=biblioauthornotvak, section=1]%
        \printbibliography[heading=countauthorconf, env=countauthorconf, keyword=biblioauthorconf, section=1]%
        \printbibliography[heading=countauthor, env=countauthor, keyword=biblioauthor, section=1]%
        \nocite{%Порядок перечисления в этом блоке определяет порядок вывода в списке публикаций автора
	        	TRIC,% VAK, Scopus, WoS
	        	PAX,% Scopus, WoS
	        	Stats,% Scopus, WoS
	        	Modeling, % Scopus, WoS
	        	FDM, % Scopus, WoS
                Aksentev:IPAC17,% Scopus
                Aksentev:LaPlas17,Aksentev:LaPlas18,Aksentev:LaPlas19,% РИНЦ
                Aksentev:IPAC19:GFF,Aksentev:IPAC19:SMP,Aksentev:IPAC19:DECOH% Scopus
        }%
        \publications\ Основные результаты по теме диссертации изложены в~\arabic{citeauthor}~печатных работах:
        \arabic{citeauthornotvak} изданы в журналах, индексируемых в международных базах цитирования Scopus и Web of Science,
        %из которых \arabic{citeauthorvak} в журнале, рекомендованном ВАК, 
        \arabic{citeauthorconf} "--- в~трудах докладов на международных конференциях. Из последних: 4 работы входят в базу Scopus, 3 в РИНЦ.
    \end{refsection}
    \begin{refsection}[vak,papers,conf]%Блок, позволяющий отобрать из всех работ автора наиболее значимые, и только их вывести в автореферате, но считать в блоке выше общее число работ
        \printbibliography[heading=countauthorvak, env=countauthorvak, keyword=biblioauthorvak, section=2]%
        \printbibliography[heading=countauthornotvak, env=countauthornotvak, keyword=biblioauthornotvak, section=2]%
        \printbibliography[heading=countauthorconf, env=countauthorconf, keyword=biblioauthorconf, section=2]%
        \printbibliography[heading=countauthor, env=countauthor, keyword=biblioauthor, section=2]%
        \nocite{TRIC}%vak
        \nocite{Stats, PAX, Modeling, FDM}%notvak
        \nocite{Aksentev:LaPlas17,Aksentev:LaPlas18,Aksentev:LaPlas19,Aksentev:IPAC17,%
	        	Aksentev:IPAC19:GFF,Aksentev:IPAC19:SMP,Aksentev:IPAC19:DECOH}%conf
    \end{refsection}
}
%При использовании пакета \verb!biblatex! для автоматического подсчёта
%количества публикаций автора по теме диссертации, необходимо
%их~здесь перечислить с использованием команды \verb!\nocite!.
 % Характеристика работы по структуре во введении и в автореферате не отличается (ГОСТ Р 7.0.11, пункты 5.3.1 и 9.2.1), потому её загружаем из одного и того же внешнего файла, предварительно задав форму выделения некоторым параметрам

\textbf{Объем и структура работы.} Диссертация состоит из~введения, трёх глав,
заключения и~одного приложения.
%% на случай ошибок оставляю исходный кусок на месте, закомментированным
%Полный объём диссертации составляет  \ref*{TotPages}~страницу
%с~\totalfigures{}~рисунками и~\totaltables{}~таблицами. Список литературы
%содержит \total{citenum}~наименований.
%
Полный объём диссертации составляет
\formbytotal{TotPages}{страниц}{у}{ы}{}, включая
\formbytotal{totalcount@figure}{рисун}{ок}{ка}{ков} и
\formbytotal{totalcount@table}{таблиц}{у}{ы}{}.   Список литературы содержит
\formbytotal{citenum}{наименован}{ие}{ия}{ий}.

В \textbf{первой главе}: вводится понятие ``замороженного спина''; проводится сравнительный анализ методов поиска ЭДМ в накопительном кольце с ``замороженным спином''; классифицируются проблемы, общие для всех методов поиска ЭДМ в накопительном кольце; описывается метод измерения ЭДМ в накопительном кольце, позволяющий решить поставленные проблемы; представлена магнитооптическая структура накопительного кольца, в котором возможно детектировать ЭДМ дейтрона предлагаемым методом.
%\begin{enumerate}
%	\item Вводит понятие замороженного спина.
%	\item Проводит классификацию метдов поиска ЭДМ в накопительном кольце с замороженным спином.
%	\item Проводит классификацию проблем, общих для всех методов поиска ЭДМ в накопительном кольце.
%	\item Описывает метод измерения ЭДМ в накопительном кольце с замороженным спином, разрешающий описанные проблемы.
%	\item Описывает магнитооптические структуры накопительных колец, которые можно использовать для детектирования ЭДМ предлагаемым методом.
%\end{enumerate}

Во \textbf{второй главе} содержится подробное рассмотрение проблем, обозначенных в первой главе, и методов их решения; описаны результаты моделирования. 

Рассматриваемые проблемы:
\begin{enumerate}
	\item возмущения спиновой динамики частицы, вызванные её бетатронными колебаниями, и их эффект на ЭДМ-статистику частотного метода измерения;
	\item декогеренция спинов частиц продольно-поляризованного пучка при работе в режиме замороженного спина;
	\item величина и свойства систематической ошибки эксперимента, связанной с МДМ-прецессией спинов частиц пучка, и вызванной неидеальностями оптической структуры ускорителя;
	\item процедура смены полярности ведущего поля накопительного кольца, при сохранении величины МДМ спин-прецессии, необходимая для исключения обозначенной выше ошибки из ЭДМ-статистики.
\end{enumerate}

Отдельно рассматривается вопрос интерпретации введённого в первой главе понятия \emph{эффективного Лоренц-фактора} ($\gamma_{eff}$). 

Большая часть методологии, исследованию которой посвящена настоящая работа, основана на этом понятии. Его можно определять таким образом: если две частицы имеют одно и то же значение $\gamma_{eff}$, то они эквивалентны с точки зрения спиновой динамики (а именно, направления и величины вектора угловой скорости спин-прецессии), независимо от частностей их орбитального движения. 

Именно фиксация значения $\gamma_{eff}$ позволяет нам исключить МДМ-прецессию, связанную с неидеальностями машины, из конечной ЭДМ-статистики частотного метода.

В \underline{\textbf{третьей главе}} приведены наиболее значимые (для данной работы) технологии, разработанные в рамках исследований, проводимых на синхротроне COSY,\footnote{Принадлежащем институту ядерных исследований Исследовательского центра ``Юлих'', Германия} описаны результаты процедуры оптимизации времени когерентности спина (spin coherence time, SCT) при помощи семейств секступолей, установленных на COSY. 

Отдельно стоит отметить наблюдение явления изменения SCT при длительном измерении поляризации деструктивными методами, связаного с переходом от внешней (оболочки) к внутренней (ядру) частям пучка. Наблюдение этого явления косвенно подтверждает теорию спин-декогеренции, изложенную в данной работе.

В \textbf{заключении} приведены результаты работы.

\textbf{Приложение}~\ref{Apx:Stats} посвящено статистическому моделированию эксперимента, и оценке его возможной статистической точности.