
Spin wheel метод ставит перед собой задачу решения проблемы неидеальности магнитных полей ускорителя, 
а именно -- присутствие ненулевой средней радиальной компоненты магнитного поля $\avg{B_x}$, 
индуцирующей ненулевую радиальную копоненту МДМ-частоты прецессиии спина $\W_x^{MDM}$. 
Для этого, вместо измерения угла поворота вектора поляризации относительно плоскости замкнутой орбиты, 
ЭДМ вычисляется посредством сравнения величин частоты прецессии спина в вертикальной плоскости
при разных условиях.

\begin{figure}[h]\centering
	\begin{tikzpicture}[>=stealth]
	%	\draw[help lines] (-2,-3) grid (6,4);
	\draw[line width=1.2pt] (0,0) circle[x radius=4cm, y radius=2cm];
	\node[red] (c) at ($(0,0)+(-60:4 and 2)$) {.};
	\node (x) at ($(c)+(-30:3cm)$) {$x$};
	\node (z) at ($(c)+(0,2.4cm)$) {$z$};
	\node (s) at ($(c)+(15:3cm)$) {$s$};
	\draw[red, line width=.8pt, ->] (c.center) -- (z.south);
	\draw[red, line width=.8pt, ->] (c.center) -- (x.north west);
	\draw[red, line width=.8pt, ->] (c.center) -- (s.west);
	\end{tikzpicture}
	\caption{Направления координатных осей в методе spin wheel.}
\end{figure}

В методе предлагается использовать коллайдер, с пучками ионов разных $Z/A$-отношений,~\footnote{$Z$, $A$ -- зарядовое, массовое числа соответственно.} 
движущихся с разными скоростями по замкнутым орбитам $\CO{1}$, $\CO{2}$ одного и того же радиуса. ЭДМ частиц
одного из пучков (поляризованного) нужно измерить; второй пучок (неполяризованный) используется в 
качестве комагнетометра, чувствительного к радиальному магнитному полю.
При $\avg{B_x}=0$ замкнутые орбиты пучков находятся в одной и той же горизонтальной плоскости, т.е. ${\CO{1} = \CO{2} = \CO{0}}$ (номинальная).
 
Пусть теперь $\avg{B_x}\neq0$. Предполагая, что вертикальная фокусировка осуществляется только 
постоянным, однородным по кольцу градиентом электрического поля, для пучков с разными скоростями 
$\beta_1$ и $\beta_2$ условие вертикальной устойчивости (У.В.У.) выполняется при различных 
вертикальных смещениях $\avg{z}$ от номинальной замкнутой орбиты:
\begin{align}
	\avg{E_z} &= \avg{E_z(0)}+ \avg{G_z}z,\\
	\left|{\partial B_x}/{\partial z}\right| &\ll\avg{G_z} = \const, \\
	\avg{\W_V}{}_x &= -\frac{Ze}{\gamma Am_pc}\cdot\left(\avg{B_x} - \frac{\avg{E_z}}{\beta}\right)\equiv0, \tag{У.В.У.}
	\intertext{где $\W_V$ обозначает угловую скорость импульса пучка. 
		Из этих выражений получим:}
	\avg{z_1} - \avg{z_2} &= \frac{\avg{B_x}}{\avg{G_z}}\left(\beta_1-\beta_2\right).
\end{align}

Таким образом, если возможно измерять $\Delta = \avg{z_1} - \avg{z_2}$ с достаточной точностью, 
ЭДМ оценивается как
\begin{equation}\label{eq:SW:EDM-estimator}
	\hat\W_{EDM} = \frac{\hat\W_x(+\Delta) + \hat\W_x(-\Delta)}{2},
\end{equation}
где $\hat\W_x$ -- оценка совокупной (имеется ввиду $\W_x^{MDM} + \W_x^{EDM}$) частоты прецессии спина.
Переход от $+\Delta$ к $-\Delta$ предполагается осуществлять посредством специально введённого
в структуру поперечного магнитного диполя, вращающего вектор поляризации пучка вокруг радиальной оси
с частотой 0.1--1~Гц~\cite[стр.~4]{Koop:SpinWheel2015} 

Рассмотрим данный метод в контексте поставленных во введении к этому разделу вопросов.

Во-первых, необходимо отметить переход от фазы к частоте спин-прецесии, 
как основы инференции о величине ЭДМ. Разумеется, чтобы иметь возможность измерять частоту,
поляризации пучка позволено свободно прецессировать (за счёт МДМ) вокруг некоторой оси -- 
наиболее оптимально, чтобы эта ось совпадала с осью ЭДМ спин-прецессии. Поскольку спин-векторы частиц
прецессируют с достаточно высокой угловой скоростью, чувствительность направления оси прецессии спина
к случайным возмущениям поля значительно снижена. Ось стабильна. В связи с этим, метод более защищён 
от ошибки геометрической фазы.

С точки зрения проблемы измерения поляризации пучка, преимущества метода не так очевидны. 
С одной стороны, благодаря тому, что поляризация свободно осциллирует в вертикальной плоскости, 
проблема стремления анализирующей способности детектора к нулю при измерении малого угла 
не возникает.
С другой стороны, необходимость использовать ко-циркулирующие пучки 
накладывает некоторые ограничения на их энергии, а следовательно -- и на величину
сечения взаимодействия пучка и мишени. Представленные в 
таблицах~2--7~\cite{Koop:SpinWheel2015} кольца расчитаны на малые энергии. 
Малость радиусов колец представлена как их достоинство; однако, из неё следует малость величины
сечения взаимодействия, а значит малость статистики и трудность измерения поляризации. 

Касательно решения проблем МДМ спин-прецессии и спин-декогеренции. 
Проблема ортогональных к ЭДМ компонент МДМ спин-прецессии $\W_y, \W_z$ решается увеличением 
сонаправленной компоненты $\W_x$.~\cite[стр.~5, ур.~(14)]{Koop:SpinWheel2015} 
Также предполагается, что это увеличение $\W_x$ решает проблему декогеренции спина. 
Этот вывод аргументируется, в том числе, измерениями, сделанными в результате экспериментов 
на синхротроне COSY~\cite{COSY:DAQ}. Мы хотим обратить внимание на то, что в этих экспериментах
поляризация пучка измерялась только в одной плоскости. 

Действительно, вследствие того, что 
вектор поляризации начинает вращаться в вертикальной плоскости с достаточной частотой, та же оптика,
которая вызывает рост дисперсии направлений спин-векторов частиц пучка на первой четверти
 периода осцилляции, уменьшает её на второй. Таким образом, дисперсия в горизонтальной плоскости
 действительно перестаёт нарастать. Однако, спин-декогеренция вызвана различием частот
 прецессии спин-векторов частиц пучка, и это различие, в свою очередь, основано на различии 
 равновесных энергий частиц пучка. Подробнее, связь между длиной орбиты частицы, её равновесной энергией, и частотой спин-прецессии описана в подразделе~\ref{chpt1:FS-methods:effective-Lorentz-factor};
 численная симуляция, показывающая, что при появлении спин-прецессии вокруг радиальной оси, 
 спин-декогеренция переходит из горизонтальной в вертикальную плоскость, но не исчезает окончательно
 описана в подразделе~\ref{sec:Decoherence-plane-transfer}.
 
Нам остаётся рассмотреть вопрос разделения ЭДМ и МДМ сигналов с сонаправленными осями прецессии.
Как видно из уравнения~\eqref{eq:SW:EDM-estimator}, в методе Spin wheel это предлагается делать 
на основе измерений вертикального разделения $\Delta = \avg{z_1} - \avg{z_2}$ замкнутых орбит пучков,
которое в свою очередь линейно связано со средним значением радиальной компоненты магнитного поля
$\avg{B_x}$, индуцирующего МДМ-вращение спин-векторов частиц. 

Отметим, что в первую очередь, $\avg{B_x} \equiv \avg{B_x}^{mi}$ (machine imperfection) возникает в связи 
с ошибкой юстировки элементов оптической структуры ускорителя, и потому 
однозначно связано с ведущим полем. 
В связи с последним, существует возможность изменять знак $\avg{B_x}$ 
(а вместе с ним и знак $\W_x^{MDM}$), посредством \emph{обращения полярности} ведущего поля кольца.

В Spin wheel-методе, для контроля величины и знака $\Delta$ используется радиально-направленный
магнитный диполь. Варьируя величину поля диполя $\avg{B_x}^{md}$, добиваются, чтобы совокупная величина 
поля ${\avg{B_x} = \avg{B_x}^{mi} + \avg{B_x}^{md}}$ генерировала разделение $\Delta$ на уровне 
$\pm 10^{-12}$~м. Необходимость определения $\Delta$ с такой точностью мотивирована
величиной магнитного поля (на уровне $10^{-15}$~Тл), имитирующего ЭДМ ${d\approx 10^{-29}}$~\ecm.

Для аргументации возможности измерения разделения замкнутых орбит пучков на таком уровне точности
авторы метода ссылаются на~\cite{Kawall:BPM}. 


