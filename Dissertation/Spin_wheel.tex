
Spin wheel метод ставит себе задачей решение проблемы неидеальности магнитных полей ускорителя, 
а именно -- присутствие ненулевой средней радиальной компоненты магнитного поля $\avg{B_x}$, 
индуцирующей ненулевую радиальную копоненту МДМ-частоты прецессиии спина $\W_x^{MDM}$. 
Для этого, вместо измерения угла поворота вектора поляризации относительно плоскости замкнутой орбиты, 
ЭДМ вычисляется посредством сравнения частоты прецессии спина в вертикальной плоскости
при разных условиях.

\begin{figure}[h]\centering
	\begin{tikzpicture}[>=stealth]
	%	\draw[help lines] (-2,-3) grid (6,4);
	\draw[line width=1.2pt] (0,0) circle[x radius=4cm, y radius=2cm];
	\node[red] (c) at ($(0,0)+(-60:4 and 2)$) {.};
	\node (x) at ($(c)+(-30:3cm)$) {$x$};
	\node (z) at ($(c)+(0,2.4cm)$) {$z$};
	\node (s) at ($(c)+(15:3cm)$) {$s$};
	\draw[red, line width=.8pt, ->] (c.center) -- (z.south);
	\draw[red, line width=.8pt, ->] (c.center) -- (x.north west);
	\draw[red, line width=.8pt, ->] (c.center) -- (s.west);
	\end{tikzpicture}
	\caption{Направления координатных осей в методе spin wheel.}
\end{figure}

В методе предлагается использовать коллайдер, с пучками ионов разных $Z/A$-отношений,~\footnote{$Z$, $A$ -- зарядовое, массовое числа соответственно.} 
движущихся с разными скоростями по замкнутым орбитам одного и того же радиуса. ЭДМ частиц
одного из пучков (поляризованного) нужно измерить, второй пучок (неполяризованный) используется в 
качестве комагнетометра, чувствительного к радиальному магнитному полю.
При $\avg{B_x}=0$ замкнутые орбиты пучков находятся в одной и той же горизонтальной плоскости.
 
Пусть теперь $\avg{B_x}\neq0$. Предполагая, что вертикальная фокусировка осуществляется только 
однородным градиентом электрического поля, для пучков с разными скоростями $\beta_1$ и $\beta_2$
условие вертикальной устойчивости (У.В.У.) выполняется при различных вертикальных смещениях $\avg{z}$ 
от номинальной замкнутой орбиты:
\begin{align}
	\avg{E_z} &= \avg{E_z(0)}+ \avg{G_z}z,\\
	\left|{\partial B_x}/{\partial z}\right| &\ll\avg{G_z}, \\
	\avg{\W_V}{}_x &= C\cdot\left(\avg{B_x} + \frac{\avg{E_z}}{\beta}\right), \tag{У.В.У.}\\
	\avg{z_1} - \avg{z_2} &= \frac{\avg{B_x}}{\avg{G_z}}\left(\beta_1-\beta_2\right).
\end{align}

Таким образом, если возможно измерять $\Delta = \avg{z_1} - \avg{z_2}$ с достаточной точностью, 
ЭДМ оценивается как
\begin{equation}\label{eq:SW:EDM-estimator}
	\hat\W_{EDM} = \frac{\hat\W_x(+\Delta) + \hat\W_x(-\Delta)}{2},
\end{equation}
где $\hat\W_x$ -- оценка совокупной (имеется ввиду $\W_x^{MDM} + \W_x^{EDM}$) частоты прецессии спина.

В данном случае, переход от $+\Delta$ к $-\Delta$ предполагается осуществлять посредством специально введённого
в структуру поперечного магнитного диполя, вращающего вектор поляризации пучка вокруг радиальной оси
с частотой 0.1--1~Гц~\cite[стр.~4]{Koop:SpinWheel2015} Необходимо отметить, что при этом частота 
МДМ-прецессии спина вокруг радиальной оси явно предполагается удовлетворяющей условиям BNL FS метода, 
а именно -- находится на уровне $10^{-9}$~Гц. Однако, необходимым условием успешности измерения ЭДМ
методом BNL FS является точная установка оптических элементов ускорителя. Требуемая
точность установки оптических элементов ускорителя является проблематичной.

Таким образом, можно заключить, что:
\begin{itemize}
	\item[-] проблема систематической ошибки МДМ-прецессии, связанной с неидеальностью самого ускорителя 
	не является решённой;
	\item[+] индуцированная диполем контролируемая МДМ-прецессия служит средством подавления (только)
	случайных возмущений магнитного поля;
	\item[+] она также позволяет уйти от проблемы геометрической фазы.
\end{itemize}

%%%%%%%%%%%%%%%%%%%%
%В данном методе используется коллайдер с противоположно-циркулирующими ионными пучками: ЭДМ частиц первого (поляризованного) требуется измерить, а второй (неполяризованный) используется в качестве комагнетометра, чувствительного к радиальной компоненте магнитного поля.
%
%В связи с присутствием в структуре радиальной компоненты магнитного поля, замкнутые орбиты пучков разделяются в вертикальном ($z$) направлении. Если контролировать разделение орбит $\Delta = \avg{z_1} - \avg{z_2}$ с достаточной точностью (а именно, на уровне $10^{-12}$~м~\cite{Kawall:BPM}, при ограничении ошибки измерения ЭДМ на уровне $10^{-29}$~\ecm), то возможно определить ЭДМ по формуле
%\[
%\hat\W_{EDM} = \frac{\hat{\W}_x(+\Delta) + \hat{\W}_x(-\Delta)}{2},
%\]
%где $\pm\Delta$ получают путём варьирования поля специально введённого для этого магнитного диполя.
%
%В данном методе делаются как минимум два предположения, с которыми не согласен автор данной работы:
%\begin{enumerate}[(1)]
%	\item возможность измерения вертикального разделения замкнутых орбит пучков на уровне $10^{-12}$~м посредством использования SQUID-магнетометров;\label{itm:Koop-assumption-1}
%	\item ... ????
%\end{enumerate}
%
%Мы не согласны с предположением~\ref{itm:Koop-assumption-1} на том основании, что 
%\begin{enumerate*}[(a)]
%	\item такая точность измерения смещения орбиты не была показана экспериментально, но
%	 \item даже предполагая такую возможность, оценки~\cite{Kawall:BPM} предполагают слабую вертикальную фокусировку и \hl{набег бетатронной фазы 0.1}, что исключает возможность коррекции спин-декогеренциии в такой структуре.
%\end{enumerate*}
