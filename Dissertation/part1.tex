\chapter{Методология измерения ЭДМ и её систематические ошибки} \label{chapt1}

\section{Общее введение в методологию ``Замороженного Спина'' (FS)} \label{sect1_1}
\subsection{Уравнение Т-БМТ}
Уравнение Томаса-БМТ описывает динамику спин-вектора $\vec s$ в
магнитном поле $\vec B$ и электростатическом поле $\vec E$. Его
обобщённая версия, включающая влияние ЭДМ, может быть записана (в
системе центра масс пучка) как:~\cite[стр.~6]{Eremey:Thesis}
\begin{subequations}
  \begin{align}
    \ddt{\vec s} &= \vec s\times \bkt{\vec\W_{MDM} +\vec\W_{EDM}}, \label{eq:TBMT_main}
    \intertext{где МДМ и ЭДМ угловые скорости $\vec\W_{MDM}$ и $\vec\W_{EDM}$ }
    \vec\W_{MDM} &= \frac qm \bkt*{G\vec B - \bkt{G - \frac{1}{\gamma^2-1}}\frac{\vec E\times\vec\beta}{c}},\label{eq:TBMT_MDM} \\
    \vec\W_{EDM} &= \frac qm \frac\eta2 \bkt*{\frac{\vec E}c + \vec\beta\times \vec B}.\label{eq:TBMT_EDM}
  \end{align}
\end{subequations}
В уравнениях выше, $m,~q,~G=(g-2)/2$ есть, соответственно, масса, заряд, и
магнитная аномалия частицы; $\beta = \sfrac{v_0}{c}$,
нормализованная скорость частицы; $\gamma$ её Лоренц-фактор. ЭДМ
множитель $\eta$ определяется уравнением $d = \eta\frac{q}{2mc}s$, где
$d$ --- ЭДМ частицы, а $s$ её спин.

\subsection{Концепция замороженного спина}
Из уравнения~\eqref{eq:TBMT_MDM} можно видеть, что, в отсутствии ЭДМ,
направление вектора спина частицы пучка может быть зафиксировано
относительно её вектора импульса: $\vec\W_{MDM}=\vec 0$; иными словами, можно реализовать
условие замороженности спина (Frozen Spin condition).

Достоинством налагания FS-условия на пучок в накопительном кольце
следующее: в соответствии с
уравнениями~\cref{eq:TBMT_main,eq:TBMT_MDM,eq:TBMT_EDM}, векторы МДМ и
ЭДМ угловых скоростей ортогональны, а потому в общей скорости
прецессии они складываются квадратично, в связи с чем сдвиг частоты
прецессии, связанный с ЭДМ, становится эффектом второго порядка
величины:~\cite[стр.~5]{Mane:SpinWheel}
\[
\w \propto \sqrt{\W_{MDM}^2 + \W_{EDM}^2} \approx \W_{MDM} + \frac{\W_{EDM}^2}{2\W_{MDM}}.
\]
Это обстоятельство значительно ухудшает чувствительность эксперимента.

Однако, заморозив спин в горизонтальной плоскости, единственная
осающаяся МДМ компонента угловой скорости сонаправлена с ЭДМ
компонентой, а значит складывается с ней линейно. Таким образом,
чувствительность значительно улучшается.

\subsection{Реализация условия замороженности спина в накопительном кольце}
Накопительные кольца могут быть классифицированы в три группы:
\begin{enumerate}
\item чисто магнитные (как COSY, NICA, etc),
\item чисто электростатические (Brookhaven AGS Analog Ring),
\item комбинированные.
\end{enumerate}

Ввиду уравнения~\eqref{eq:TBMT_MDM}, условие FS не может быть
выполнено в чисто магнитном кольце.

Для некоторого числа частиц, таких как протон, чья $G>0$, чисто
электростатическое кольцо может быть использовано в рамках FS
методологии ЭДМ эксперимента с пучком на так называемой ``магической''
энергии, определяемой как $\gamma_{mag} = \sqrt{(1+G)/G}$.

Для частиц с $G<0$ (таких как дейтрон),это невозможно, и необходимо
использовать комбинированное кольцо. Для того, чтобы реализовать FS
условие в комбинированном кольце, вводится ~\cite{BNL:Deuteron2008} радиальное электрическое
поле величины
\begin{equation}\label{eq:FS_Er}
  E_r = \frac{GB_yc\beta\gamma^2}{1-G\beta^2\gamma^2}.
\end{equation}

\section{Методы, основанные на FS-методологии} \label{sect1_2}
\subsection{Метод BNL FS}
BNL FS метод, предложенный коллаборацией, занимающейся разработкой
метода измерения ЭДМ в накопительном кольце в Брукхейвенской
Национальной Лаборатории (США) в 2008 году,~\cite{BNL:Deuteron2008}
это метод для комбинированного кольца. Пучок продольно-поляризованных
дейтронов инжектируется в кольцо; с помощью поляриметрии наблюдается
его спин-прецессияв вертикальной и горизонтальной плоскостях; ЭДМ
сигнал --- это изменение вертикальной компоненты поляризации со
временем, выражаемое как:~\cite[стр.~8]{BNL:Deuteron2008}
\begin{equation}
  \Delta P_V = P\frac{\w_{edm}}{\W}\sin\bkt{\W t + \Theta_0},
\end{equation}
где $\W = \sqrt{\w_{edm}^2 + \w_a^2}$, $\w_a,~\w_{edm}$ угловые
скорости вызываемые, соответственно, магнитным и электрическим
дипольными моментами.

Налагая радиальное магнитное поле $E_r$~\eqref{eq:FS_Er}, ожидается
уменьшение компоненты $\w_a$ по крайней мере на фактор $10^9$; ввиду
малости величины гипотезируемой $\w_{edm}$, $\Delta P_V \approx P
\w_{edm} t$, и максимальная величина $\Delta P_V$ возрастает в $10^9$.

Ожидаемая чувствительность эксперимента $10^{-29}~e\cdot cm$ за $10^7$
секунд (6 месяцев) полного времени измерения. На этом уровне
чувствительности, асимметрия сечения $\varepsilon_{LR} \approx 5\cdot
10^{-6}$ для наименьших практически реализуемых значений
$\w_a$.~\cite[стр.~18]{BNL:Deuteron2008} Последнее обстоятельство
ставит серьёзную проблему для поляриметрии.~\cite{Mane:SpinWheel} Один
из вариантов её решения лежит в применении внешнего радиального
магнитного поля и измерении обзей частоты прецессии засчёт МДМ и ЭДМ
вместе. Это основа так называемого метода Spin Wheel, о котором в
следующем разделе.

Единственный известный систематический эффект спиновой динамики
первого порядка это присутствие ненулевой средней вертикальной
компоненты электрического поля $\avg{E_V}$. В этом случае, спин будет
прецессировать вокруг радиального направления с частотой~\cite[стр.~11]{BNL:Deuteron2008}
\[
\w_{syst} \approx \frac{\mu\avg{E_V}}{\beta c\gamma^2}.
\]
Здесь важно рассмотреть два обстоятельства:
\begin{itemize}
\item присутствие $\avg{E_V}\neq 0$ вызвано ошибкой юстировки
  элементов ускорителя;
\item этот систематический эффект меняет знак при инжекции пучка в
  обратном направлении.
\end{itemize}
Последнее обстоятельство является причиной структуры инжекции пучка
использованной в этом методе (сначала по-часовой, потом
против-часовой; CW/CCW). Хотя $\w_{syst}$ меняет знак при смене
направления движения пучка , эта методология тем не менее не учитывает
его \emph{величину}. В разделах~\ref{sec:systematic_error-fake-signal} и~\ref{sec:simulation-fake_signal}, мы
показываем, что при реалистичной величине ошибки установки
спин-ротаторов 100 мкм, частота МДМ прецессии вокруг радиальной оси
находится на уровне 50--100 рад/сек.~\cite{Senichev:FDM} В связи с
этим, невозможно использовать данную методологию в её оригинальном варианте.

\subsection{Spin Wheel концепция}
Озвученные выше проблемы с поляриметрией и высокой скоростью прецессии
решаются в Spin Wheel (SW) модификации, предложенной проф. И. Коопом
(Новосибирский Государственный Университет). Основная идея метода в
следующем: сначала, обеспечивается условие замороженного спина; затем
включается радиальное магнитное поле величины $B_x$, достаточно сильное чтобы
вызвать вращение спина с частотой порядка 1 Гц. Поскольку поле
радиальное, вызванная им МДМ прецессия сонаправлена с ЭДМ, а значит
они складываются линейно: $\w \propto \W_{MDM} + \W_{EDM}$.

ЭДМ вклад вычисляется сравнением циклов с обратными знаками $B_x$:~\cite[стр.~1963]{Koop:IPAC13}
\[
\W_{EDM} = \frac{\W_x(+B_x) + \W_x(-B_x)}{2}.
\]

Внешнее поле также вызовет разделение орбит
пучков.~\cite[стр.~1963]{Koop:IPAC13} Это разделение может быть
измерено на уровне пико-метров SQUID магнетометрами; его предлагается
использовать для калибровки внешнего поля.

Поскольку из-за внешнего поля прецессия вокруг радиальной оси на 10
порядков выше чем в оригинальном предложении, значительно упрощается
задача для поляриметрии. Однако, существуют сомнения в возможности
измерить вызываемое внешним полем разделение орбит даже при помощи SQUIDов.

Также, проблема паразитного поля, вызванного ошибкой юстировки, не решена.

\subsection{Метод Frequency Domain}
Методология Frequency Domain (далее FDM)~\cite{Senichev:FDM} была разработана специально для решения проблемы неточности установки магнитов, и возникающего в связи с этим паразитного МДМ вращения спина. Как было обозначено выше, частота вращения спина в вертикальной плоскости, связанная с магнитным дипольным моментом, при реалистичной ошибке юстировки, находится на уровне 8--16 Гц, что делает невозможным наблюдение медленного нарастания вертикальной компоненты поляризации, связанное с наличием у частицы электрического дипольного момента, как предполагается оригинальным BNL FS методом. В FDM, ЭДМ-эффект вычисляется путём сравнения комбинированной (МДМ + ЭДМ) частоты прецессии, наблюдаемой при циркуляции пучка в прямом и обратном направлениях. Поскольку при смене полярности ведущего поля $\vec B \mapsto -\vec B$, $\vec\beta \mapsto -\vec\beta$, и $\vec E \mapsto \vec E$:
\begin{subequations}
  \begin{align}
    \w_x^{CW/CCW} &= \w_x^{MDM, CW/CCW} + \w_x^{EDM, CW/CCW}, \\
    \w_x^{MDM, CW} &= -\w_x^{MDM, CCW} \equiv \w_x^{MDM}, \label{eq:FDM_CW_CCW_MDM} \\
    \w_x^{EDM, CW} &= \w_x^{EDM, CCW} \equiv \w_x^{EDM},
    \intertext{поэтому, ЭДМ эстиматор}
    \hat\w_x^{EDM} &:= \frac12\bkt{\w_x^{CW} + \w_x^{CCW}} \label{eq:FDM_estimator} \\
                  &= \w_x^{EDM} + \underbrace{\frac12\bkt{\w_x^{MDM, CW} + \w_x^{MDM, CCW}}}_{\varepsilon \to 0}.
  \end{align}
\end{subequations}

Для того, чтобы гарантировать малость $\varepsilon$ по сравнению с требуемой точностью измерений, т.е., что уравнение~\eqref{eq:FDM_CW_CCW_MDM} выполняется достаточно точно, была разработана специальная процедура смены полярности ведущего поля, описанная в разделе~\ref{sec:field_flipping}.

\subsubsection{Смена полярности ведущего поля}\label{sec:field_flipping}
Здесь будет описание процедуры когда я её пойму.

Также отметим, что существуют методы измерения ЭДМ элементарных частиц, не основанные на принципе замороженного спина; например~\cite{COSY:SpinTuneMapping}

\section{Анализ систематических ошибок} \label{sec:systematic_error}
\subsection{Возмущение оси прецессии спина частицы засчёт бетатронного движения}
Motivation for the analysis of this problem is the general concern that a scalar value (effective Lorentz factor) cannot account for the jittering of the beam particles' spin precession axes, which definitely occurs as a result of the betatron motion, during which the particles experience varying RF electro-magnetic fields. That is, the SPAs form a ''spin field'' [Shatunov terminology] about the invariant spin axis defined on the reference (closed) orbit.

This fact invalidates the structural part of our statistical model, used to obtain the spin precession frequency from polarimetry data, introducing the model specification systematic error.

But this error is negligible compared to the plain polarimetry error of 3\%, hence in our model we will assume that the spins of all beam particles precess about the invariant spin axis, and consider only their spin tune variation 
\subsection{Декогеренция спинов частиц пучка в идеальном и неидеальном кольцах}
\subsubsection{Требования к времени когеренции пучка}
\subsubsection{Причины возникновения декогеренции}
\subsubsection{Подавление декогеренции с помощью секступольных полей}
\subsection{Фальш-сигнал}\label{sec:systematic_error-fake_signal}
Motivation for studying the effect. Analytical estimates of the MDM precession frequency about the radial axis. 
