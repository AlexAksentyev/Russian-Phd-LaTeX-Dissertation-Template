\chapter{Методология измерения ЭДМ и её систематические ошибки}

\section{Общее введение в методологию ``Замороженного Спина'' (FS)}
\subsection{Уравнение Т-БМТ}\label{sec:TBMT_introduction}
Уравнение Томаса-БМТ описывает динамику спин-вектора $\vec s$ в
магнитном поле $\vec B$ и электростатическом поле $\vec E$. Его
обобщённая версия, включающая влияние ЭДМ, может быть записана (в
системе центра масс пучка) как:~\cite[стр.~6]{Eremey:Thesis}
\begin{subequations}
  \begin{align}
    \ddt{\vec s} &= \vec s\times \bkt{\vec\W_{MDM} +\vec\W_{EDM}}, \label{eq:TBMT_main}
    \intertext{где МДМ и ЭДМ угловые скорости $\vec\W_{MDM}$ и $\vec\W_{EDM}$ }
    \vec\W_{MDM} &= \frac qm \bkt*{G\vec B - \bkt{G - \frac{1}{\gamma^2-1}}\frac{\vec E\times\vec\beta}{c}},\label{eq:TBMT_MDM} \\
    \vec\W_{EDM} &= \frac qm \frac\eta2 \bkt*{\frac{\vec E}c + \vec\beta\times \vec B}.\label{eq:TBMT_EDM}
  \end{align}
\end{subequations}
В уравнениях выше, $m,~q,~G=(g-2)/2$ есть, соответственно, масса, заряд, и
магнитная аномалия частицы; $\beta = \sfrac{v_0}{c}$,
нормализованная скорость частицы; $\gamma$ её Лоренц-фактор. ЭДМ
множитель $\eta$ определяется уравнением $d = \eta\frac{q}{2mc}s$, где
$d$ --- ЭДМ частицы, а $s$ её спин.

В стандартном формализме принято оперировать с матрицей преобразования (поворота) спина за оборот в кольце $R$:~\cite[стр.~4]{COSY:SpinTuneMapping}
\[
\bold{t}_R = \exp\bkt{-i\pi\nu_s\vec\sigma\cdot\bar n} = \cos\pi\nu_s - i (\vec\sigma\cdot\bar n)\sin\pi\nu_s,
\]

где $\nu_s = \sfrac{\W_s}{\W_{cyc}}$ отношение угловой скорости поворота спин-вектора частицы к её циклотронной частоте, называемое \emph{спин-тюн}, а $\bar n$ определяет направление оси прецессии спина, и называется \emph{инвариантной спиновой осью}.

\subsection{Концепция замороженного спина}
Из уравнения~\eqref{eq:TBMT_MDM} можно видеть, что, в отсутствии ЭДМ,
направление вектора спина частицы пучка может быть зафиксировано
относительно её вектора импульса: $\vec\W_{MDM}=\vec 0$; иными словами, можно реализовать
условие замороженности спина (Frozen Spin condition).

Достоинством налагания FS-условия на пучок в накопительном кольце
следующее: в соответствии с
уравнениями~\cref{eq:TBMT_main,eq:TBMT_MDM,eq:TBMT_EDM}, векторы МДМ и
ЭДМ угловых скоростей ортогональны, а потому в общей скорости
прецессии они складываются квадратично, в связи с чем сдвиг частоты
прецессии, связанный с ЭДМ, становится эффектом второго порядка
величины:~\cite[стр.~5]{Mane:SpinWheel}
\[
\w \propto \sqrt{\W_{MDM}^2 + \W_{EDM}^2} \approx \W_{MDM} + \frac{\W_{EDM}^2}{2\W_{MDM}}.
\]
Это обстоятельство значительно ухудшает чувствительность эксперимента.

Однако, заморозив спин в горизонтальной плоскости, единственная
осающаяся МДМ компонента угловой скорости сонаправлена с ЭДМ
компонентой, а значит складывается с ней линейно. Таким образом,
чувствительность значительно улучшается.

\subsection{Реализация условия замороженности спина в накопительном кольце}\label{sec:FS_in_a_ring}
Накопительные кольца могут быть классифицированы в три группы:
\begin{enumerate}
\item чисто магнитные (как COSY, NICA, etc),
\item чисто электростатические (Brookhaven AGS Analog Ring),
\item комбинированные.
\end{enumerate}

Ввиду уравнения~\eqref{eq:TBMT_MDM}, условие FS не может быть
выполнено в чисто магнитном кольце.

Для некоторого числа частиц, таких как протон, чья $G>0$, чисто
электростатическое кольцо может быть использовано в рамках FS
методологии ЭДМ эксперимента с пучком на так называемой ``магической''
энергии, определяемой как $\gamma_{mag} = \sqrt{(1+G)/G}$.

Для частиц с $G<0$ (таких как дейтрон),это невозможно, и необходимо
использовать комбинированное кольцо. Для того, чтобы реализовать FS
условие в комбинированном кольце, вводится ~\cite{BNL:Deuteron2008} радиальное электрическое
поле величины
\begin{equation}\label{eq:FS_Er}
  E_r = \frac{GB_yc\beta\gamma^2}{1-G\beta^2\gamma^2}.
\end{equation}

\section{Методы, основанные на FS-методологии} \label{sect1_2}
\subsection{Метод BNL FS}
BNL FS метод, предложенный коллаборацией, занимающейся разработкой
метода измерения ЭДМ в накопительном кольце в Брукхейвенской
Национальной Лаборатории (США) в 2008 году,~\cite{BNL:Deuteron2008}
это метод для комбинированного кольца. Пучок продольно-поляризованных
дейтронов инжектируется в кольцо; с помощью поляриметрии наблюдается
его спин-прецессияв вертикальной и горизонтальной плоскостях; ЭДМ
сигнал --- это изменение вертикальной компоненты поляризации со
временем, выражаемое как:~\cite[стр.~8]{BNL:Deuteron2008}
\begin{equation}
  \Delta P_V = P\frac{\w_{edm}}{\W}\sin\bkt{\W t + \Theta_0},
\end{equation}
где $\W = \sqrt{\w_{edm}^2 + \w_a^2}$, $\w_a,~\w_{edm}$ угловые
скорости вызываемые, соответственно, магнитным и электрическим
дипольными моментами.

Налагая радиальное магнитное поле $E_r$~\eqref{eq:FS_Er}, ожидается
уменьшение компоненты $\w_a$ по крайней мере на фактор $10^9$; ввиду
малости величины гипотезируемой $\w_{edm}$, $\Delta P_V \approx P
\w_{edm} t$, и максимальная величина $\Delta P_V$ возрастает в $10^9$.

Ожидаемая чувствительность эксперимента $10^{-29}~e\cdot cm$ за $10^7$
секунд (6 месяцев) полного времени измерения. На этом уровне
чувствительности, асимметрия сечения $\varepsilon_{LR} \approx 5\cdot
10^{-6}$ для наименьших практически реализуемых значений
$\w_a$.~\cite[стр.~18]{BNL:Deuteron2008} Последнее обстоятельство
ставит серьёзную проблему для поляриметрии.~\cite{Mane:SpinWheel} Один
из вариантов её решения лежит в применении внешнего радиального
магнитного поля и измерении обзей частоты прецессии засчёт МДМ и ЭДМ
вместе. Это основа так называемого метода Spin Wheel, о котором в
следующем разделе.

Единственный известный систематический эффект спиновой динамики
первого порядка это присутствие ненулевой средней вертикальной
компоненты электрического поля $\avg{E_V}$. В этом случае, спин будет
прецессировать вокруг радиального направления с частотой~\cite[стр.~11]{BNL:Deuteron2008}
\[
\w_{syst} \approx \frac{\mu\avg{E_V}}{\beta c\gamma^2}.
\]
Здесь важно рассмотреть два обстоятельства:
\begin{itemize}
\item присутствие $\avg{E_V}\neq 0$ вызвано ошибкой юстировки
  элементов ускорителя;
\item этот систематический эффект меняет знак при инжекции пучка в
  обратном направлении.
\end{itemize}
Последнее обстоятельство является причиной структуры инжекции пучка
использованной в этом методе (сначала по-часовой, потом
против-часовой; CW/CCW). Хотя $\w_{syst}$ меняет знак при смене
направления движения пучка, эта методология тем не менее не учитывает
его \emph{величину}. В разделах~\ref{sec:systematic_error-fake_signal} и~\ref{sec:simulation-fake_signal}, 
мы
показываем, что при реалистичной величине ошибки установки
спин-ротаторов 100 мкм, частота МДМ прецессии вокруг радиальной оси
находится на уровне 50--100 рад/сек.~\cite{Senichev:FDM} В связи с
этим, невозможно использовать данную методологию в её оригинальном варианте.

\subsection{Spin Wheel концепция}
Озвученные выше проблемы с поляриметрией и высокой скоростью прецессии
решаются в Spin Wheel (SW) модификации, предложенной проф. И. Коопом
(Новосибирский Государственный Университет). Основная идея метода в
следующем: сначала, обеспечивается условие замороженного спина; затем
включается радиальное магнитное поле величины $B_x$, достаточно сильное чтобы
вызвать вращение спина с частотой порядка 1 Гц. Поскольку поле
радиальное, вызванная им МДМ прецессия сонаправлена с ЭДМ, а значит
они складываются линейно: $\w \propto \W_{MDM} + \W_{EDM}$.

ЭДМ вклад вычисляется сравнением циклов с обратными знаками $B_x$:~\cite[стр.~1963]{Koop:IPAC13}
\[
\W_{EDM} = \frac{\W_x(+B_x) + \W_x(-B_x)}{2}.
\]

Внешнее поле также вызовет разделение орбит
пучков.~\cite[стр.~1963]{Koop:IPAC13} Это разделение может быть
измерено на уровне пико-метров SQUID магнетометрами; его предлагается
использовать для калибровки внешнего поля.

Поскольку из-за внешнего поля прецессия вокруг радиальной оси на 10
порядков выше чем в оригинальном предложении, значительно упрощается
задача для поляриметрии. Однако, существуют сомнения в возможности
измерить вызываемое внешним полем разделение орбит даже при помощи SQUIDов.

Также, проблема паразитного поля, вызванного ошибкой юстировки, не решена.

\subsection{Метод Frequency Domain}\label{sec:FDM_concept}
Методология Frequency Domain (далее FDM)~\cite{Senichev:FDM} была разработана специально для решения проблемы неточности установки магнитов, и возникающего в связи с этим паразитного МДМ вращения спина. Как было обозначено выше, частота вращения спина в вертикальной плоскости, связанная с магнитным дипольным моментом, при реалистичной ошибке юстировки, находится на уровне 8--16 Гц, что делает невозможным наблюдение медленного нарастания вертикальной компоненты поляризации, связанное с наличием у частицы электрического дипольного момента, как предполагается оригинальным BNL FS методом. В FDM, ЭДМ-эффект вычисляется путём сравнения комбинированной (МДМ + ЭДМ) частоты прецессии, наблюдаемой при циркуляции пучка в прямом и обратном направлениях. Поскольку при смене полярности ведущего поля $\vec B \mapsto -\vec B$, $\vec\beta \mapsto -\vec\beta$, и $\vec E \mapsto \vec E$:
\begin{subequations}
  \begin{align}
    \w_x^{CW/CCW} &= \w_x^{MDM, CW/CCW} + \w_x^{EDM, CW/CCW}, \\
    \w_x^{MDM, CW} &= -\w_x^{MDM, CCW} \equiv \w_x^{MDM}, \label{eq:FDM_CW_CCW_MDM} \\
    \w_x^{EDM, CW} &= \w_x^{EDM, CCW} \equiv \w_x^{EDM},
    \intertext{поэтому, ЭДМ эстиматор}
    \hat\w_x^{EDM} &:= \frac12\bkt{\w_x^{CW} + \w_x^{CCW}} \label{eq:FDM_estimator} \\
                  &= \w_x^{EDM} + \underbrace{\frac12\bkt{\w_x^{MDM, CW} + \w_x^{MDM, CCW}}}_{\varepsilon \to 0}.
  \end{align}
\end{subequations}

Для того, чтобы гарантировать малость $\varepsilon$ по сравнению с требуемой точностью измерений, т.е., что уравнение~\eqref{eq:FDM_CW_CCW_MDM} выполняется достаточно точно, была разработана специальная процедура смены полярности ведущего поля, описанная в разделе~\ref{sec:field_flipping-theory}.

\subsubsection{Смена полярности ведущего поля}\label{sec:field_flipping-theory}
Как было описано в разделе~\ref{sec:FDM_concept}, для того, чтобы исключить МДМ-эффект из конечной статистики эксперимента, построенного на основе Frequency Domain методологии в комбинированном накопительном кольце, необходимо произвести смену полярности ведущего магнитного поля. Электростатическое поле $E_r = \frac{GB_yc\beta\gamma^2}{1-G\beta^2\gamma^2}$ (см. раздел~\ref{sec:FS_in_a_ring}) при этом фиксировано.

Частоты прецессии спинов частиц пучка определяются по формуле~\cite[стр.~4]{COSY:SpinTuneMapping}
\[
(\W_x, \W_y, \W_z) = 2\pi\cdot f_{rev} \cdot \nu_s \cdot \bar n,
\]
где $f_{rev}$ есть циклотронная частота частицы, а $\nu_s$ и $\bar n$ --- её спин-тюн и ось стабильного спина, соответственно. В разделе~\ref{sec:decoherence_simulations} мы приведём свидетельства того, что при использовании секступольных полей выравниваются не только спин-тюны частиц, но и направления их осей стабильного спина, в связи с чем в дальнейшем рассмотрении мы будем предполагать что спин-векторы всех частиц в пучке вращаются вокруг $\bar n^{CO}$, определённой на референсной орбите. Таким образом, при смене полярности ведущего поля достаточно восстановить эффективный Лоренц-фактор пучка, для того, чтобы восстановить величину угловой скорости паразитной МДМ прецессии.

Калибровка $\gamma_{eff}$ выполняется напрямую, через восстановление угловой скорости прецессии спина в горизонтальной плоскости:
В начальном состоянии, $\W_x\gg \W_y, \W_z$, и $\bar n^{CO}\approx \hat x$. Используя спин-суппрессор (Вин-фильтр), мы подавляем прецессию вокруг вектора $\hat x$; одновременно с этим, мы отходим от ``замороженного'' значения энергии (это делается для того, чтобы избежать неустойчивого состояния ``заморозки'' спина во всех плоскостях). При изменении энергии пучка, меняется также и величина ведущего поля, за тем, чтобы сохранить референсную орбиту. Горизонтальная прецессия становится доминантной, и $\bar n^{CO} \approx \hat y$. После смены полярности ведущего поля, мы опять подстраиваем его величину таким образом, чтобы восстановить условие замороженности спина в горизонтальной плоскости. Тогда, при выключении спин-суппрессора, и возвращении энергии пучка на изначальный уровень, мы получаем $\bar n^{CO} \approx -\hat x$, $\gamma_{eff}^{CCW} = \gamma_{eff}^{CW}$, то есть, МДМ прецессия происходит с той же угловой скоростью, но в обратном направлении. 

Также отметим, что существуют методы измерения ЭДМ элементарных частиц, не основанные на принципе замороженного спина; например~\cite{COSY:SpinTuneMapping}

\section{Анализ систематических ошибок} \label{sec:systematic_error}
\subsection{Возмущение оси прецессии спина частицы засчёт бетатронного движения}
Motivation for the analysis of this problem is the general concern that a scalar value (effective Lorentz factor) cannot account for the jittering of the beam particles' spin precession axes, which definitely occurs as a result of the betatron motion, during which the particles experience varying RF electro-magnetic fields. That is, the SPAs form a ''spin field'' [Shatunov terminology] about the invariant spin axis defined on the reference (closed) orbit.

This fact invalidates the structural part of our statistical model, used to obtain the spin precession frequency from polarimetry data, introducing the model specification systematic error.

But this error is negligible compared to the plain polarimetry error of 3\%, hence in our model we will assume that the spins of all beam particles precess about the invariant spin axis, and consider only their spin tune variation 
\subsection{Декогеренция спинов частиц пучка в идеальном и неидеальном кольцах}
Когеренцией спина называется мера или качество сохранения поляризации
в изначально поляризованном пучке.~\cite[стр.~205]{Eremey:Thesis}

Когда поляризованный пучок инжектируется в накопительное кольцо, спин
векторы частиц пучка начинают прецессировать вокруг вертикального
(ведущего) поля. Частота прецессии зависит от равновесного уровня
энергии частицы, который различен для частиц пучка.

Это обстоятельство не является проблемой в том случае, когда начальная
поляризация пучка вертикальна; однако метод измерения ЭДМ в
накопительном кольце, основанный
на принципе замороженного спина требует, чтобы вектор поляризации
пучка был сонаправлен с его вектором импульса, т.е. лежал в
горизонтальной плоскости. Таким образом, декогеренция спина есть
внутренняя проблема метода замороженного спина.
\subsubsection{Требования ко времени когеренции пучка}
Время когеренции спина (spin coherence time; SCT) для метода
замороженного спина, выполненного в накопительном кольце с идеально
установленными элементами определяется минимальным детектируемым углом
отклонения вектора поляризации пучка из горизонтальной плоскости
только засчёт ЭДМ. Для уровня чувствительности $10^{-29}~e\cdot cm$
это примерно $5\cdot10^{-6}$.~\cite{BNL:Deuteron2008}

В соответствии с уравнением Т-БМТ,
\[
\W_{EDM,x} = \eta\frac{qE_x}{2mc},
\]
где $\eta$ есть коэффициент пропорциональности между ЭДМ и спином,
равный $10^{-15}$ для дейтрона,для данного уровня чувствительности.~\cite[стр.~206]{Eremey:Thesis}

Для дейтронного BNL FS кольца, $E_x = 12$
МВ/м,~\cite[стр.~19]{BNL:Deuteron2008} так что $\W_{EDM,x}\approx
10^{-9}$ рад/сек. Таким образом получаем, что для того, чтобы достичь
детектируемый уровень отклонения вектора поляризации на 1 мкрад требуется SCT порядка 1000 секунд.~\cite[стр.~207]{Eremey:Thesis}
\subsubsection{Происхождение декогеренции}
Декогеренция спина в пучке вызвана разницей угловых скоростей
прецессии спинов частиц, которая, в свою очередь, вызвана разницей
длин орбит и импульсов частиц. Это можно видеть исходя из следующих
соображений.

Когда частица со спином входит в область магнитного поля, её спин-вектор
начинает поворачиваться вокруг вектора магнитного поля с угловой
скоростью определяемой уравнением Т-БМТ~\eqref{eq:TBMT_MDM}:
\begin{equation*}
	\vec\W_{MDM} = \frac qm G\vec B.
\end{equation*}
На выходе из области, вектор спина повёрнут на угол
\begin{equation*}
	\theta = \Delta t\cdot\W_{MDM} = \frac Lv \cdot \frac qm GB\cdot \frac{\gamma_0}{\gamma_0} = \frac{L\gamma_0 GB}{B\rho} = \frac L\rho\gamma_0\cdot G,
\end{equation*}
где $L$ есть длина пути внутри области с магнитным полем, и $B\rho =
p/q$ магнитная жёсткость.

В простой модели рассмотренной выше, влияние орбитальной динамики на
спиновую динамику вырадено через $\gamma_0 L/\rho$ (эффективный
Лоренц-фактор). В случае референсной частицы, $\gamma_0L/\rho =
\gamma_0\alpha$, $\alpha$ угол поворота вектора импульса,в то время
как для частицы участвующей в бетатронном движении, эффективный
Лоренц-фактор больше. В следующих разделах мы выразим связь между
спиновой и орбитальной динамиками частицы в накопительном кольце в
более общих терминах.

\paragraph{Сдвиг равновесного значения импульса частицы.}
Продольная динамика заряженной частицы на референсной орбите в
накопительном кольце описывается системой уравнений
\begin{equation*}
	\begin{cases}
		\ddt{\varphi} &= -\w_{RF}\eta\delta, \\
		\ddt{\delta} &= \frac{q V_{RF}\w_{RF}}{2\pi h\beta^2E}\sin\varphi.
	\end{cases}
\end{equation*}
В уравнениях выше: $\varphi$ отклонение фазы частицы от референсной
$\varphi_0 = 0$; $\delta = \frac{\Delta p}{p_0}$ относительное
отклонение импульса от $p_0$ референсной частицы; $V_{RF}$, $\w_{RF}$
амплитуда и частоты колебаний ВЧ поля; $\eta = \alpha_0 - \gamma^{-2}$
слип-фактор, где $\alpha_0$ --- коэффициент сжатия орбиты, определяемый
через $\sfrac{\Delta L}{L} = \alpha_0\delta$, $L$ длина орбиты; $h$
гармоническое число; $E$ полная энергия ускоряемой частицы. $\w_{RF} =
2\pi h f_{rev}$, где $f_{rev}=T_{rev}^{-1}$ --- частота оборота пучка.

Решения этой системы формируют семейство эллипсов в плоскости
$(\varphi, \delta)$, центрированных на $(0,0)$. Однако, если
рассмотреть частицу, участвующую в бетатронных колебаниях, и
использовать разложение Тейлора более высокого порядка для
коэффициента сжатия орбиты $\alpha = \alpha_0 + \alpha_1\delta$,
первое уравнение системы превратится в:~\cite[стр.~2579]{Senichev:IPAC13}
\[
\ddt{\varphi} = -\w_{RF} \bkt*{\bkt{\frac{\Delta L}{L}}_\beta + \bkt{\alpha_0 + \gamma^{-2}}\delta + \bkt{\alpha_1 - \alpha_0\gamma^{-2} + \gamma^{-4}}\delta^2},
\]
где $\bkt{\frac{\Delta L}{L}}_\beta =
\frac{\pi}{2L}\bkt*{\varepsilon_xQ_x + \varepsilon_yQ_y},$ есть
удлинение орбиты, связанное с бетатронным движением; $\varepsilon_x$ и
$\varepsilon_y$ --- горизонтальный и вертикальный эмиттансы пучка, и
$Q_x$ и $Q_y$ горизонтальный и вертикальный тюны.~\cite[стр.~2580]{Senichev:IPAC13}

Решения модифицированной системы более не центрированы на одной и той
же точке. Удлинение орбиты и отклонение импульса вызывают сдвиг
равновесного значения импульса частицы~\cite[стр.~2581]{Senichev:IPAC13}
\begin{equation}\label{eq:EquLevMom_shift}
\Delta\delta_{eq} = \frac{\gamma_0^2}{\gamma_0^2\alpha_0 - 1}\bkt*{\frac{\delta_m^2}{2}\bkt{\alpha_1 - \alpha_0\gamma^{-2} + \gamma_0^{-4}} + \bkt{\frac{\Delta L}{L}}_\beta},
\end{equation}
где $\delta_m$ --- амплитуда синхротронных колебаний.

\paragraph{Понятие эффективного Лоренц-фактора.}
Равновесное значение энергии, связанное со сдвигом импульса~\eqref{eq:EquLevMom_shift}, и называемое \emph{эффективным Лоренц-фактором}, есть~\cite{Senichev:FDM}
\begin{equation}\label{eq:EffectiveGamma}
\gamma_{eff} = \gamma_0 + \beta_0^2\gamma_0\cdot\Delta\delta_{eq},
\end{equation}
где $\gamma_0$, $\beta_0$ --- Лоренц-фактор референсной частицы и
нормализованное значение
скорости. Уравнения~\eqref{eq:EquLevMom_shift}
и~\eqref{eq:EffectiveGamma} определяют связь между спиновой и
орбитальной динамиками частицы.

Из уравнения для спин-тюна частицы в магнитном поле $\nu_s = \gamma_{eff}\cdot G$ следует, что спин-тюны двух частиц с одинаковыми эффективными Лоренц-факторами равны, независимо от их траекторий в ускорителе. Этот принцип используется при использовании секступольных полей для подавления спиновой декогеренции, а также при смене полярности ведущего магнитного поля кольца.

\subsubsection{Подавление декогеренции с помощью секступольных полей}
Чтобы минимизировать декогеренцию спина, связанную с бетатронным
движением и отклонением импульса, могут быть использованы
секступольные (или октупольные) поля~\cite[стр.~212]{Eremey:Thesis}

Секступоль силы
\[
S_{sext} = \frac{1}{B\rho} \pddx{B_y}[x][2],
\]
где $B\rho$ магнитная жёсткость, влияет на коэффициент сжатия орбиты
первого порядка как~\cite[стр.~2581]{Senichev:IPAC13}
\begin{align}
	\Delta \alpha_{1,sext} &= -\frac{S_{sext}D_0^3}{L}, \label{eq:Sext_compaction_effect}
	\intertext{и одновременно на длину орбиты как}
	\bkt{\frac{\Delta L}{L}}_{sext} &= \mp \frac{S_{sext}D_0\beta_{x,y}\varepsilon_{x,y}}{L}, \label{eq:Sext_OL_effect}
\end{align}
где $D(s,\delta) = D_0(s) + D_1(s)\delta$ обозначает функцию дисперсии.

В следующих разделах мы будем называть декогеренцию, связанную си
горизонтальными/вертикальными бетатронными, и синхротронными
колебаниями соответственно X-/Y-, и D-декогеренцией. 

Из уравнений~\labelcref{eq:Sext_compaction_effect,eq:Sext_OL_effect} можно
видеть, что для подавления декогеренции необходимы три семейства
секступолей, помещённых в максимумы функций: $\beta_x$, $\beta_y$ для подавления
X-,Y-декогеренции, и $D_0$ для D-декогеренции.

\subsection{Фальш-сигнал}\label{sec:systematic_error-fake_signal}
Систематические ошибки, вызванные физическими неидеальностями
ускорителя, включая неточность юстировки оптических элементов,
вызывают фальш-сигнал ЭДМ.~\cite[стр.~230]{Eremey:Thesis} Особенно в
этом отношении проблематичны наклоны элементов вокруг оптической оси, поскльку они
индуцируют паразитные горизонтальные компоненты магнитного поля $B_x$
и $B_z$, которые обе вращают спин в вертикальной плоскости; той, в которой измеряется ЭДМ.

Ю. Сеничевым были сделаны~\cite{Senichev:FDM} аналитические оценки МДМ частоты прецессии спина
вокруг радиальной оси. Из уравнения Т-БМТ, и выражения силы Лоренца,
скорость МДМ прецессии вокруг радиальной оси есть
\begin{equation}
\SD{\W_x^{MDM}} = \frac{q}{m\gamma}\frac{G+1}{\gamma}\frac{\SD{B_x}}{\sqrt{n}},
\end{equation}
где $n$ есть число наклонённых спин-ротаторов, и $\SD{B_x} = B_y
\SD{\delta h}/L$, при стандартном отклонении ошибки юстировки
$\SD{\delta h}$. При величине ошибки $\SD{\delta h} = 100$ мкм, и
длине дефлектора $L=1$ м, $\SD{\W_x^{MDM}} \approx 100$ рад/сек.~\cite{Senichev:FDM}

Мы изучили спиновую динамику в структурах с замороженным и
квази-замороженным спином в присутствии наклонов оптических элементов
с помощью кода COSY INFINITY. Наши симуляции согласуются с оценками,
представленными выше.

\paragraph{Имплементация паразитного поля.}\label{sec:error_field_implementation}
Имплементируя неидеальности полей, мы следовали рекомендациям
изложенным в~\cite[стр.~235]{Eremey:Thesis}. Малое возмущение
магнитного поля, в первом приближении, действует как маленький пропорциональный поворот
спин-вектора. Поэтому мы имплементировали наклон E+B элемента как
домножение соответствующей матрицы поворота на его спиновую матрицу
перехода, ``спин-кик.''

В соответствии с уравнением~\eqref{eq:TBMT_MDM}, изменение МДМ частоты
прецессии, ассоциированное с введённым паразитным полем $(B_x, 0, B_z)$ есть
\begin{align*}
	\Delta\W_{MDM} &= \frac qm (B_x, 0, B_z),
	\intertext{поэтому угол спин-кика равен}
	\Theta_{kick} &= t_0\Delta\W_{MDM},
\end{align*}
где $t_0 = L/v_0$ пролётное время референсной частицы через элемент.
